\documentclass[%
  aspectratio=169,
  9pt,
%   t,
  USenglish,
%english,
%   dark,
  light,
  mathserif,
%   serif, 
  professionalfont,
%  handout,
%  affiliationinhead,
  affiliationintitlepagehead,
  titlegraphic,
  %% The following options would violate the CD rules!
   affiliation,
%   uselogos,
  % navigationbar,
  %progressbar,
%   seprules,
%   titleinhead,
]{beamer}

%
%\usepackage{tikz}
%\usepackage{pgfplots}
%\usepgfplotslibrary{groupplots}



\usepackage{times}
\usepackage{epsfig}
\usepackage{graphicx}
\usepackage{amsmath}
\usepackage{amssymb}
\usepackage[utf8]{inputenc}
\usepackage{booktabs}
\setlength{\tabcolsep}{5pt}
\usepackage{subcaption}

\usepackage{appendix}
\usepackage{tummath}
\usepackage{tumtensors} % experimental
\usepackage{tumcolors}

\usepackage{pgfmath}
\newcommand\randmin{}
\newcommand\randmax{}
\newcommand\randmultof{}
\newcommand\setrand[4]%
{\def\randmin{#1}%
	\def\randmax{#2}%
	\def\randmultof{#3}%
	\pgfmathsetseed{#4}%
}
\newcommand\nextrand
{\pgfmathparse{int(int((rnd*(\randmax-\randmin+1)+\randmin)/\randmultof)*\randmultof)}%
	\xdef\thisrand{\pgfmathresult}%
}

% Include other packages here, before hyperref.

% If you comment hyperref and then uncomment it, you should delete
% egpaper.aux before re-running latex.  (Or just hit 'q' on the first latex
% run, let it finish, and you should be clear).
%\usepackage[breaklinks=true,bookmarks=false]{hyperref}


\usepackage[capitalize]{cleveref}
\usepackage[square,sort,comma,numbers]{natbib}

%% this hack seems to be nececessary due to incompatibilities of cvpr template and tikz... -> https://tex.stackexchange.com/questions/398223/tikz-gives-error-command-everyshipouthook-already-defined
\makeatletter
\@namedef{ver@everyshi.sty}{}
\makeatother
%% hackend

%{r\tikzsetnextfilenameawinput}

\newcommand{\rawtimeseries}[1]{

\begin{tikzpicture}[baseline=-2em, inner sep=0]

	\begin{axis}[
		thin,
		width=6cm,
		hide axis,
		height=3cm,
		ymin=0, ymax=1.4,
		no marks,  
		draw opacity=.8,
		smooth=0.01
	]
		 
	\addplot[b1color] table [x=t, y=B1, col sep=comma, forget plot] {images/example/#1};
	\addplot[b9color] table [x=t, y=B9, col sep=comma, forget plot] {images/example/#1};
	\addplot[b10color] table [x=t, y=B10, col sep=comma] {images/example/input.csv};
	
	\addplot[b11color] table [x=t, y=B11, col sep=comma, forget plot] {images/example/#1};
	\addplot[b12color] table [x=t, y=B12, col sep=comma] {images/example/#1};
	
	\addplot[b5color] table [x=t, y=B5, col sep=comma, forget plot] {images/example/#1};
	\addplot[b6color] table [x=t, y=B6, col sep=comma, forget plot] {images/example/#1};
	\addplot[b7color] table [x=t, y=B7, col sep=comma, forget plot] {images/example/#1};
	\addplot[b8color] table [x=t, y=B8, col sep=comma, forget plot] {images/example/#1};
	\addplot[b8Acolor] table [x=t, y=B8A, col sep=comma] {images/example/#1};
		
	\addplot[b2color] table [x=t, y=B2, col sep=comma, forget plot] {images/example/#1};
	\addplot[b3color] table [x=t, y=B3, col sep=comma, forget plot] {images/example/#1};
	\addplot[b4color] table [x=t, y=B4, col sep=comma] {images/example/#1};

	\end{axis}
	
\end{tikzpicture}
	
}

\usepackage{tikz}
\usepackage{pgfplots}
\usetikzlibrary{positioning, calc,arrows,arrows.meta, fit}
%\usetikzlibrary{arrows.meta,calc,decorations.markings,math,arrows.meta}
\usepgfplotslibrary{groupplots}
\usepgfplotslibrary{fillbetween}
\usepgfplotslibrary{statistics} % provides boxplots
\usepackage{xfrac}

\newcommand{\tp}{tp}
\newcommand{\tn}{tn}
\newcommand{\fp}{fp}
\newcommand{\fn}{fn}


\usepackage{tumcolors}
\usepackage{tummath}
\newcommand{\yhat}{\hat{\V{y}}}
\newcommand{\ycorrect}{\hat{y}^+}
\newcommand{\thetadelta}{\V{\Theta}_\delta}
\newcommand{\biasdelta}{b_\delta}
\newcommand{\biasclass}{\V{b}_\text{c}}
\newcommand{\thetaclass}{\V{\Theta}_\text{c}}
\newcommand{\thetafeat}{\V{\Theta}_\text{feat}}
\newcommand{\fclass}{f_\text{c}}
\newcommand{\fdelta}{f_\delta}
\newcommand{\ffeat}{f_\text{feat}}
\newcommand{\f}{f}

\newcommand{\rvtime}{T_c} 
\newcommand{\xuptot}{\M{X}_{\rightarrow t}} 
\newcommand{\deltauptot}{\delta_{\rightarrow t}} 
\newcommand{\tstop}{\ensuremath{t_\text{stop}}}
\newcommand{\meantstop}{\ensuremath{\bar{t}_\text{stop}}}
\usepackage[super]{nth}
\usepackage{mathtools}

\definecolor{evalcolor}{HTML}{3F3F3F}
\definecolor{traincolor}{HTML}{B98951}
\definecolor{validcolor}{HTML}{3F4BBE}

\colorlet{colortrain}{tumblue}
\colorlet{colorinfer}{tumblack}

\colorlet{earlinesscolor}{tumblue}
\colorlet{accuracycolor}{tumorange}

\colorlet{stdcolor}{tumbluelight}
\colorlet{mediancolor}{tumorange}
\colorlet{meancolor}{tumblue}

%\colorlet{b1color}{tumdiagramaubergine}
%\colorlet{b2color}{tumdiagramnavyblue}
%\colorlet{b3color}{tumdiagramturquoise}
%\colorlet{b4color}{tumdiagramgreen}
%\colorlet{b5color}{tumdiagramlimegreen}
%\colorlet{b6color}{tumdiagramyellow}
%\colorlet{b7color}{tumdiagramsand}
%\colorlet{b8color}{tumdiagramredorange}
%\colorlet{b8Acolor}{tumdiagramred}
%\colorlet{b9color}{tumblack}
%\colorlet{b10color}{tumblue}
%\colorlet{b11color}{tumdiagramdarkred}
%\colorlet{b12color}{tumorange}

% atmospheric bands
%\colorlet{b1color}{tumaubergine}%tumdiagramaubergine
%\colorlet{b9color}{tumblack}%tumblack
%\colorlet{b10color}{tumblack}%tumblue

%visisble bands
\colorlet{b2color}{tumblue}%tumdiagramnavyblue
\colorlet{b3color}{tumgreen}%tumdiagramturquoise
\colorlet{b4color}{tumdarkred}%tumdiagramgreen

% near infrared bands
\colorlet{b5color}{tumlimegreen}%tumdiagramlimegreen
\colorlet{b6color}{tumturquoise}%tumdiagramyellow
\colorlet{b7color}{tumblack}%tumdiagramsand
\colorlet{b8color}{tumredorange}%tumdiagramredorange
\colorlet{b8Acolor}{tumred}%tumdiagramred

% SWIR bands
\colorlet{b11color}{tumaubergine}%tumdiagramdarkred
\colorlet{b12color}{tumorange}%tumorange

\colorlet{epsilon0color}{tumorange}
\colorlet{epsilon1color}{tumblue}
\colorlet{epsilon10color}{tumblack}

\colorlet{meadowcolor}{tumbluemedium}
\colorlet{wbarleycolor}{tumbluedark}
\colorlet{corncolor}{tumorange}
\colorlet{wheatcolor}{tumgreen}
\colorlet{sbarleycolor}{tumred}
\colorlet{clovercolor}{tumturquoise}
\colorlet{triticalecolor}{tumsand}

\tikzstyle{rnn}=[draw,circle, inner sep=.1em]
\tikzstyle{norm}=[rounded corners,draw]
\tikzstyle{annot}=[rounded corners, fill=tumblue!20]
\tikzstyle{infer}=[-stealth, shorten >=.0em, shorten <=.0em, colorinfer]
\tikzstyle{loss}=[fill=tumblue!10, rounded corners, font=\small]
\tikzstyle{grad}=[colortrain]

\newcommand{\ptoffset}{\varepsilon}

\tikzstyle{test} = [thick]
\tikzstyle{train} = [thin, dotted]

\usepackage[inline]{enumitem}
\setenumerate{label=(\roman*),itemsep=3pt,topsep=3pt}

\setlength{\belowcaptionskip}{-10pt}
%\usepackage{titlesec}
%\titlespacing{\section}{0pt}{10pt}{3pt}

\usetikzlibrary{external}
\tikzexternalize[prefix=tikz/]
%\tikzexternalize
\tikzexternaldisable

\usepackage[eulergreek]{sansmath}
\pgfplotsset{
	y tick label style={/pgf/number format/.cd,%
		scaled y ticks = false,
		set thousands separator={},
		fixed},
	x tick label style={/pgf/number format/.cd,%
		scaled x ticks = false,
		set decimal separator={,},
		fixed},
	tick label style = {font=\sansmath\sffamily},
	every axis label = {
		font=\sansmath\sffamily},
	every axis/.append style={
		axis lines=left, 
		enlargelimits, 
		thick},
	legend style = {font=\sansmath\sffamily, draw=none, rounded corners, fill opacity=.5, text opacity=1},
	label style = {font=\sansmath\sffamily},
	grid style={line width=.1pt, draw=gray!10},
	major grid style={line width=.2pt,draw=tumgraylight},
}

%\let\tempone\itemize
%\let\temptwo\enditemize
%\renewenvironment{itemize}{\tempone\addtolength{\itemsep}{-.5\baselineskip}}{\temptwo}

\tikzstyle{circ} = [circle, draw=white, fill=tumblue, inner sep=1pt]
\newcommand{\fcn}{
	\begin{tikzpicture}[scale=0.2, rotate=0, baseline=-.25em, inner sep=1pt]
	\node[circ](a0) at (0,-1){};
	\node[circ](a1) at (0,0){};
	\node[circ](a2) at (0,1){};
	
	\node[circ](b0) at (1,-0.5){};
	\node[circ](b1) at (1,0.5){};
	
	\draw[-] (a0) -- (b0);
	\draw[-] (a1) -- (b0);
	\draw[-] (a2) -- (b0);
	
	\draw[-] (a0) -- (b1);
	\draw[-] (a1) -- (b1);
	\draw[-] (a2) -- (b1);
	
	\end{tikzpicture}
}


\newcommand{\earth}{
	\begin{tikzpicture}[baseline=-.25em, inner sep=0]
	\node{\includegraphics[width=8mm]{images/icons/earth}};
	\end{tikzpicture}
}

\newcommand{\sat}{
	\begin{tikzpicture}[baseline=-.25em, inner sep=0]
	\node[rotate=270,anchor=center]{\includegraphics[width=8mm]{images/icons/sat2}};
	\end{tikzpicture}
}

\newcommand{\hidden}[1]{
	\begin{tikzpicture}[scale=.1, baseline=-.25em]	
	%\draw[step=1.0,black,thin] (0,0) grid (#1,1);
	\foreach \i in {1,...,#1}{
		\node[circle, draw=white, fill=tumbluelight, inner sep=1pt] at (\i,0){};
	}
	\end{tikzpicture}
}

\newcommand{\drawvector}[1]{
	\begin{tikzpicture}[scale=.1, baseline=-.25em]	
	%\draw[step=1.0,black,thin] (0,0) grid (#1,1);
	\foreach \i in {1,...,#1}{
		\node[circ] at (\i,0){};
	}
	\end{tikzpicture}
}
\tikzstyle{proba} = [circle, draw=tumgray, inner sep=2.5pt, fill=tumorange]
\newcommand{\drawprobas}[5]{
	\begin{tikzpicture}[scale=.3, baseline=-.25em]	

	\node[proba, fill=tumblue!#1] at (0,-2){};
	\node[proba, fill=tumblue!#2] at (0,-1){};
	\node[proba, fill=tumblue!#3] at (0,-0){};
	\node[proba, fill=tumblue!#4] at (0,1){};
	\node[proba, fill=tumblue!#5] at (0,2){};
	\end{tikzpicture}
}

\newcommand{\vegetationsmodell}{
	\begin{tikzpicture}[scale=0.5, rotate=0, baseline=-.25em]
	\node[proba](a0) at (0,-1){};
	\node[proba](a1) at (0,0){};
	\node[proba](a2) at (0,1){};
	
	\node[proba](b0) at (1,-0.5){};
	\node[proba](b1) at (1,0.5){};
	
	\draw[-] (a0) -- (b0);
	\draw[-] (a1) -- (b0);
	\draw[-] (a2) -- (b0);
	
	\draw[-] (a0) -- (b1);
	\draw[-] (a1) -- (b1);
	\draw[-] (a2) -- (b1);
	
	\node[fit=(a0)(a2)(b1),draw,rounded corners](node name){};
	
	\end{tikzpicture}
}
\usetheme{TUM}

%{r\tikzsetnextfilenameawinput}

\newcommand{\rawtimeseries}[1]{

\begin{tikzpicture}[baseline=-2em, inner sep=0]

	\begin{axis}[
		thin,
		width=6cm,
		hide axis,
		height=3cm,
		ymin=0, ymax=1.4,
		no marks,  
		draw opacity=.8,
		smooth=0.01
	]
		 
	\addplot[b1color] table [x=t, y=B1, col sep=comma, forget plot] {images/example/#1};
	\addplot[b9color] table [x=t, y=B9, col sep=comma, forget plot] {images/example/#1};
	\addplot[b10color] table [x=t, y=B10, col sep=comma] {images/example/input.csv};
	
	\addplot[b11color] table [x=t, y=B11, col sep=comma, forget plot] {images/example/#1};
	\addplot[b12color] table [x=t, y=B12, col sep=comma] {images/example/#1};
	
	\addplot[b5color] table [x=t, y=B5, col sep=comma, forget plot] {images/example/#1};
	\addplot[b6color] table [x=t, y=B6, col sep=comma, forget plot] {images/example/#1};
	\addplot[b7color] table [x=t, y=B7, col sep=comma, forget plot] {images/example/#1};
	\addplot[b8color] table [x=t, y=B8, col sep=comma, forget plot] {images/example/#1};
	\addplot[b8Acolor] table [x=t, y=B8A, col sep=comma] {images/example/#1};
		
	\addplot[b2color] table [x=t, y=B2, col sep=comma, forget plot] {images/example/#1};
	\addplot[b3color] table [x=t, y=B3, col sep=comma, forget plot] {images/example/#1};
	\addplot[b4color] table [x=t, y=B4, col sep=comma] {images/example/#1};

	\end{axis}
	
\end{tikzpicture}
	
}

%%%%%%%%%%%%%%%%%%%%%%%%%%%%%%%%%%%%%%%%%%%%%%%%%%%%%%%%%%%%%%%%%%%%%%%%%%%%%%%%%%%%%%%

%Convolutional-Recurrent Networks for Multi-temporal Classification
\title{Cloud-Robust Classification of Remote Sensing Time Series}
%\subtitle{NeurIPS 2018 Workshop on Spatiotemporal Modeling and Decision-making}
\subtitle{$\Phi$-week 2019}
\author[M. Rußwurm, M. Körner]{Marc Rußwurm, Marco Körner}
\institute[TUM]{Technical University of Munich\\Chair of Remote Sensing Technology\\Computer Vision Research Group\\\url{www.lmf.bgu.tum.de/vision}}

\date{13th September 2019, ESA ESRIN, Frascati, Italy}

\begin{document}

\begin{frame}[t]
  \titlepage
\end{frame}

%\begin{frame}
%	\frametitle{Outline}
%	
%	First, introduce
%	\begin{description}
%		\item time series data and
%		\item the application   identification 
%		\item address data pre-processing
%		\item and end-to-end learning of deep neural networks
%	\end{description}
%
%	Then, qualitatively demonstrate the mechanisms of
%	\begin{description}
%		\item Gated Recurrence 
%		\item Self-attention
%	\end{description}
%
%	Finally, show quantitative results on 
%	\begin{description}
%		\item preprocessed and
%		\item raw Sentinel 2 Time Series
%	\end{description}
%\end{frame}


{{\setbeamertemplate{background canvas}{\includegraphics[width=\paperwidth]{images/cloudfree}}
\begin{frame}[plain]
%\frametitle{When we think of satellite images we picture this}
%%\centering\includegraphics[width=.75\textwidth]{images/montreal_satellite}
%\includegraphics[width=\textwidth]{images/cloudfree}
\end{frame}
}

{\setbeamertemplate{background canvas}{\includegraphics[width=\paperwidth]{images/clouds}}
	\begin{frame}[plain]
%		\frametitle{... however, ususally }
			
		
	%		\includegraphics[width=\textwidth]{images/cloud_airplane}
			
	\end{frame}
	
}

{\setbeamercolor{background canvas}{bg=tumbluedark}
	\begin{frame}[plain]
	
	\vspace{8em}
	\begin{center}
		\Huge\color{tumwhite}
		How should we deal with $\includegraphics[width=2em]{images/icons/cloud2}^\ast$?
	\end{center}\color{white}
	\vspace{2em}
	\raggedleft \Large$^\ast$ ...and other noise in the data
	
	\vfill
	\vspace{6em}
	\raggedleft{\small \color{tumgray}
	Icons made by Smashicons from www.flaticon.com
	}
\end{frame}
}


\begin{frame}<presentation:1>
\frametitle{Cloud coverage}
\centering

\def\imagewidth{1.5cm}


\visible<1>{\includegraphics[width=\imagewidth]{images/activations/16494/x/x-0.png}}
\visible<1>{\includegraphics[width=\imagewidth]{images/activations/16494/x/x-1.png}}
\visible<1>{\includegraphics[width=\imagewidth]{images/activations/16494/x/x-2.png}}
\visible<1>{\includegraphics[width=\imagewidth]{images/activations/16494/x/x-3.png}}
\visible<1>{\includegraphics[width=\imagewidth]{images/activations/16494/x/x-4.png}}
\visible<1,2>{\includegraphics[width=\imagewidth]{images/activations/16494/x/x-5.png}}
\visible<1>{\includegraphics[width=\imagewidth]{images/activations/16494/x/x-6.png}}
\visible<1>{\includegraphics[width=\imagewidth]{images/activations/16494/x/x-7.png}}
\visible<1,2>{\includegraphics[width=\imagewidth]{images/activations/16494/x/x-8.png}}
\visible<1>{\includegraphics[width=\imagewidth]{images/activations/16494/x/x-9.png}}
\visible<1>{\includegraphics[width=\imagewidth]{images/activations/16494/x/x-10.png}}
\visible<1>{\includegraphics[width=\imagewidth]{images/activations/16494/x/x-11.png}}
\visible<1,2>{\includegraphics[width=\imagewidth]{images/activations/16494/x/x-12.png}}
\visible<1,2>{\includegraphics[width=\imagewidth]{images/activations/16494/x/x-13.png}}
\visible<1,2>{\includegraphics[width=\imagewidth]{images/activations/16494/x/x-14.png}}
\visible<1,2>{\includegraphics[width=\imagewidth]{images/activations/16494/x/x-15.png}}
\visible<1,2>{\includegraphics[width=\imagewidth]{images/activations/16494/x/x-16.png}}
\visible<1>{\includegraphics[width=\imagewidth]{images/activations/16494/x/x-18.png}}
\visible<1>{\includegraphics[width=\imagewidth]{images/activations/16494/x/x-19.png}}
\visible<1,2>{\includegraphics[width=\imagewidth]{images/activations/16494/x/x-20.png}}
\visible<1,2>{\includegraphics[width=\imagewidth]{images/activations/16494/x/x-21.png}}
\visible<1>{\includegraphics[width=\imagewidth]{images/activations/16494/x/x-22.png}}
\visible<1>{\includegraphics[width=\imagewidth]{images/activations/16494/x/x-23.png}}
\visible<1>{\includegraphics[width=\imagewidth]{images/activations/16494/x/x-24.png}}
\visible<1>{\includegraphics[width=\imagewidth]{images/activations/16494/x/x-25.png}}
\visible<1>{\includegraphics[width=\imagewidth]{images/activations/16494/x/x-26.png}}
\visible<1,2>{\includegraphics[width=\imagewidth]{images/activations/16494/x/x-27.png}}
\visible<1>{\includegraphics[width=\imagewidth]{images/activations/16494/x/x-28.png}}
\visible<1,2>{\includegraphics[width=\imagewidth]{images/activations/16494/x/x-29.png}}
\visible<1>{\includegraphics[width=\imagewidth]{images/activations/16494/x/x-30.png}}
\visible<1>{\includegraphics[width=\imagewidth]{images/activations/16494/x/x-31.png}}
\visible<1,2>{\includegraphics[width=\imagewidth]{images/activations/16494/x/x-32.png}}
\visible<1>{\includegraphics[width=\imagewidth]{images/activations/16494/x/x-33.png}}
%	
\end{frame}

\newcommand{\xtvector}{
	\begin{tikzpicture}[baseline=-1.9em,yscale=-5]
		\node[draw=tumgraylight, circle, fill=b2color, text=white, text opacity=1, font=\small, inner sep=.1em](d) at (0,0){};
		\node[draw=tumgraylight, circle, fill=b3color, text=white, text opacity=1, font=\small, inner sep=.1em](d) at (0,1){};
		\node[draw=tumgraylight, circle, fill=b4color, text=white, text opacity=1, font=\small, inner sep=.1em](d) at (0,2){};
		\node[draw=tumgraylight, circle, fill=b5color, text=white, text opacity=1, font=\small, inner sep=.1em](d) at (0,3){};
		\node[draw=tumgraylight, circle, fill=b6color, text=white, text opacity=1, font=\small, inner sep=.1em](d) at (0,4){};
		\node[draw=tumgraylight, circle, fill=b7color, text=white, text opacity=1, font=\small, inner sep=.1em](d) at (0,5){};
		\node[draw=tumgraylight, circle, fill=b8color, text=white, text opacity=1, font=\small, inner sep=.1em](d) at (0,6){};
		\node[draw=tumgraylight, circle, fill=b8Acolor, text=white, text opacity=1, font=\small, inner sep=.1em](d) at (0,7){};
		\node[draw=tumgraylight, circle, fill=b11color, text=white, text opacity=1, font=\small, inner sep=.1em](d) at (0,8){};
		\node[draw=tumgraylight, circle, fill=b12color, text=white, text opacity=1, font=\small, inner sep=.1em](d) at (0,9){};
	\end{tikzpicture}
}



\begin{frame}
	\frametitle{Looking at single pixels}
	\framesubtitle{Sentinel 2 (raw)}
	
	
	
	
	
	\begin{tikzpicture}[baseline=-2em, inner sep=0]
		
		\begin{axis}[
		width=\textwidth,
	%	hide axis,
		height=5.5cm,
		ymin=0, ymax=1.2,
		%no marks,  
		draw opacity=.8,
		smooth=0.001,
		legend style={at={(1,1.3)},line width=2pt, draw opacity=1},
		legend columns=5,
		ylabel={reflectance},
		xlabel={time $t$ {\small (January to December 2018)}}
		]
		
		
			\addplot[b2color, tsmark] table [x=t, y=B02, col sep=comma] {images/example/12-71456800_raw.csv};
			\addplot[b3color, tsmark] table [x=t, y=B03, col sep=comma] {images/example/12-71456800_raw.csv};
			\addplot[b4color, tsmark] table [x=t, y=B04, col sep=comma] {images/example/12-71456800_raw.csv};
			
			\only<2-3>{
				 \coordinate(c1) at (axis cs:15,1.1);
				 \coordinate(c2) at (axis cs:10,.8);
				 \coordinate(c3) at (axis cs:0,.6);
				 \coordinate(c4) at (axis cs:40,.6);
				 \coordinate(c5) at (axis cs:46,.6);
%				 \coordinate(c6) at (axis cs:0,.6);
				 
				 \coordinate(c8) at (axis cs:24,.9);
				 \coordinate(c9) at (axis cs:27,.8);
				 \coordinate(c7) at (axis cs:29,.9);
				 
				 
				 \node[inner sep=.5em](annotclouds) at (axis cs:32,1.2){clouds};
				 \draw[-stealth] (annotclouds) -- (c7);
				 \draw[-stealth] (annotclouds) -- (c8);
				 \draw[-stealth] (annotclouds) -- (c9);
				 \draw[-stealth] (annotclouds) -- (c4);
				 \draw[-stealth] (annotclouds) -- (c5);
				 \draw[-stealth] (annotclouds) -- (c1);
%				 \draw[-stealth] (annotclouds) -- (c6);
				 
			}
			\only<3-3>{
				
				\node[inner sep=.5em](annotground) at (axis cs:47,1.2){ground};
				\coordinate(g1) at (axis cs:4,.2);
				\coordinate(g2) at (axis cs:8,.2);
				\coordinate(g3) at (axis cs:13,.2);
				\coordinate(g4) at (axis cs:18,.2);
				\coordinate(g5) at (axis cs:23,.2);
				\coordinate(g6) at (axis cs:31,.2);
				\coordinate(g7) at (axis cs:35,.2);
				\coordinate(g8) at (axis cs:41,.2);
				\coordinate(g9) at (axis cs:47,.2);
				
%				\draw[-stealth] (annotground) -- (g1);
%				\draw[-stealth] (annotground) -- (g2);
%				\draw[-stealth] (annotground) -- (g3);
%				\draw[-stealth] (annotground) -- (g4);
%				\draw[-stealth] (annotground) -- (g5);
				\draw[-stealth] (annotground) -- (g6);
				\draw[-stealth] (annotground) -- (g7);
				\draw[-stealth] (annotground) -- (g8);
				\draw[-stealth] (annotground) -- (g9);
				
			}
			
			\only<1->{
			\addplot[b5color, tsmark] table [x=t, y=B05, col sep=comma] {images/example/12-71456800_raw.csv};
			\addplot[b6color, tsmark] table [x=t, y=B06, col sep=comma] {images/example/12-71456800_raw.csv};
			\addplot[b7color, tsmark] table [x=t, y=B07, col sep=comma] {images/example/12-71456800_raw.csv};
			\addplot[b8color, tsmark] table [x=t, y=B08, col sep=comma] {images/example/12-71456800_raw.csv};
			\addplot[b8Acolor, tsmark] table [x=t, y=B8A, col sep=comma] {images/example/12-71456800_raw.csv};
			
			\addplot[b11color, tsmark] table [x=t, y=B11, col sep=comma] {images/example/12-71456800_raw.csv};
			\addplot[b12color, tsmark] table [x=t, y=B12, col sep=comma] {images/example/12-71456800_raw.csv};
			}
		
			\only<1-3>
			\legend{B02 (blue),B03 (green),B04 (red),B05,B06,B07,B08,B8A,B11,B12}
			
		\end{axis}
		
	\end{tikzpicture}
	
\end{frame}


\begin{frame}
\frametitle{Introducing Notation}


\begin{tikzpicture}[baseline=-2em, inner sep=0]

\begin{axis}[
width=\textwidth,
%	hide axis,
height=5.5cm,
ymin=0, ymax=1.2,
%no marks,  
draw opacity=.8,
smooth=0.001,
legend style={at={(.65,1.1)},line width=2pt, draw opacity=1},
legend columns=3,
ylabel={reflectance},
xlabel={time $t$}
]

\addplot[b2color, tsmark, opacity=.6] table [x=t, y=B02, col sep=comma] {images/example/12-71456800_raw.csv};
\addplot[b3color, tsmark, opacity=.6] table [x=t, y=B03, col sep=comma] {images/example/12-71456800_raw.csv};
\addplot[b4color, tsmark, opacity=.6] table [x=t, y=B04, col sep=comma] {images/example/12-71456800_raw.csv};
\addplot[b5color, tsmark, opacity=.6] table [x=t, y=B05, col sep=comma] {images/example/12-71456800_raw.csv};
\addplot[b6color, tsmark, opacity=.6] table [x=t, y=B06, col sep=comma] {images/example/12-71456800_raw.csv};
\addplot[b7color, tsmark, opacity=.6] table [x=t, y=B07, col sep=comma] {images/example/12-71456800_raw.csv};
\addplot[b8color, tsmark, opacity=.6] table [x=t, y=B08, col sep=comma] {images/example/12-71456800_raw.csv};
\addplot[b8Acolor, tsmark, opacity=.6] table [x=t, y=B8A, col sep=comma] {images/example/12-71456800_raw.csv};
\addplot[b11color, tsmark, opacity=.6] table [x=t, y=B11, col sep=comma] {images/example/12-71456800_raw.csv};
\addplot[b12color, tsmark, opacity=.6] table [x=t, y=B12, col sep=comma] {images/example/12-71456800_raw.csv};

\node[draw=tumbluedark, rounded corners, inner sep=0, minimum width=.8em, minimum height=3em](xt) at (axis cs:32,0.22){};
\node[font=\Large] (annotxt) at (axis cs:38,1.05){$\V{x}_t = 
	\begin{pmatrix} \rho_\text{B02} \\ \vdots \\ \rho_\text{B12} \end{pmatrix} = 
	\left(\xtvector\right)$};
\draw[-stealth] (annotxt) -- (xt);

%
%\only<1>{
%	\legend{B02 (blue),B03 (green),B04 (red),B05,B06,B07,B08,B8A,B11,B12}
%}

\end{axis}

\end{tikzpicture}

\end{frame}

\begin{frame}
\frametitle{Introducing Notation}

\vspace{-2em}

\begin{tikzpicture}
\node[](X){$\left(
\begin{tikzpicture}[baseline=-4em,yscale=-.25, xscale=.4]
\foreach \x in {0,...,22}{
	\node[draw=tumgraylight, circle, fill=b2color, text=white, text opacity=1, font=\small, inner sep=.2em](d) at (\x,0){};
	\node[draw=tumgraylight, circle, fill=b3color, text=white, text opacity=1, font=\small, inner sep=.2em](d) at (\x,1){};
	\node[draw=tumgraylight, circle, fill=b4color, text=white, text opacity=1, font=\small, inner sep=.2em](d) at (\x,2){};
	\node[draw=tumgraylight, circle, fill=b5color, text=white, text opacity=1, font=\small, inner sep=.2em](d) at (\x,3){};
	\node[draw=tumgraylight, circle, fill=b6color, text=white, text opacity=1, font=\small, inner sep=.2em](d) at (\x,4){};
	\node[draw=tumgraylight, circle, fill=b7color, text=white, text opacity=1, font=\small, inner sep=.2em](d) at (\x,5){};
	\node[draw=tumgraylight, circle, fill=b8color, text=white, text opacity=1, font=\small, inner sep=.2em](d) at (\x,6){};
	\node[draw=tumgraylight, circle, fill=b8Acolor, text=white, text opacity=1, font=\small, inner sep=.2em](d) at (\x,7){};
	\node[draw=tumgraylight, circle, fill=b11color, text=white, text opacity=1, font=\small, inner sep=.2em](d) at (\x,8){};
	\node[draw=tumgraylight, circle, fill=b12color, text=white, text opacity=1, font=\small, inner sep=.2em](d) at (\x,9){};
}
\end{tikzpicture}\right) \Huge \in \mathbb{R}^{D \times T}$};

\node[draw, rounded corners, xshift=1.85em, minimum width=1em, minimum height=8.5em, label={[name=featurevector]above:$\V{x}_t$}] at (X){};

\node[left=.5em of X, font=\Huge](Xmatr){$\M{X} = $};


\node[above=4em of Xmatr, xshift=2em] (annotmatrix){Input Matrix};
\node[above=1em of featurevector] (annotfeat){feature vector at time $t$};

\draw[-stealth] (annotfeat) -- (featurevector);
\draw[-stealth] (annotmatrix) -- (Xmatr);

\node[below=5em of Xmatr, font=\Huge](yvect){$\V{y} = $};

\node[right=.1em of yvect](gt){$\begin{pmatrix}
	{y}_i\\ \vdots \\ {y}_C
	\end{pmatrix} = \left(\drawprobas{0}{0}{0}{100}{0}\right) \text{where}~y_i \in \{0,1\}$};

\node[right=1em of gt, font=\Huge](yvect2){$\yhat =$};
\node[right=.1em of yvect2](gt2){$\begin{pmatrix}
	\hat{y}_i\\ \vdots \\ \hat{y}_C
	\end{pmatrix} = \left(\drawprobas{10}{30}{10}{90}{10}\right) \text{where}~\hat{y}_i \in [0,1]$ and $C$ classes};

\node[below=2em of yvect, xshift=2em] (annoty){ground truth};

\node[below=2em of yvect2, xshift=2em] (annoty2){prediction};
\draw[-stealth] (annoty) -- (yvect);
\draw[-stealth] (annoty2) -- (yvect2);

%	\node[below right=of Xmatr] (annotmatrix){Input Matrix};
\end{tikzpicture}
\end{frame}

\begin{frame}<presentation:1-6>
\frametitle{Data Preprocessing}
%	
%	[format] in this presentation we have the special opportunity to to compare commercially pre-processed satellite imagery with raw imagery on crop type classification in Bavaria Germany.
%	
\begin{tikzpicture}[node distance=.1em]
\node[draw=black, rounded corners, minimum height=3cm, minimum width=4.5cm, label=below:preprocessing, font=\Large\bfseries](gaf){%
	\only<1>{\includegraphics[width=2cm]{images/icons/gears}}%
	\only<2>{$f_{\Mweight_\text{sel}}$}%
	\only<3>{$f_{\Mweight_\text{sel}}\left(f_{\Mweight_\text{atm}}\right)$}%
	\only<4>{$f_{\Mweight_\text{sel}}\left(f_{\Mweight_\text{atm}}\left(f_{\Mweight_\text{cl}}\right)\right)$}%
	\only<5>{$f_{\Mweight_\text{sel}}\left(f_{\Mweight_\text{atm}}\left(f_{\Mweight_\text{cl}}\left(f_{\text{int}}\right)\right)\right)$}%
	\only<6>{$\Mweight_\text{preprocessing}$}%
	\only<7>{\includegraphics[width=2cm]{images/GAF_logo}}%
};
\node[right=1.5em of gaf, inner sep=0](raw){\rawtimeseriestwo{12-71456800_raw.csv}};
\node[font=\huge,left=0em of raw, inner sep=0](bopen){$\Bigg($};
\node[font=\huge,right=0em of raw, inner sep=0](bopen){$\Bigg)$};
\visible<1->{
	\node[right=2em of raw, font=\huge](equals){$=$};
	\node[right=of equals, yshift=-1em]{\rawtimeseriestwo{12-71456800.csv}};
}
\end{tikzpicture}

\only<1-6>{
	\begin{rdescription}
		\item[$f_{\Mweight_\text{sel}}(\M{X})$]<2-> temporal selection (not considering winter period) where $\Mweight_\text{sel} = \{t_\text{start}, t_\text{end}\}$
		\item[$f_{\Mweight_\text{atm}}(\M{X})$]<3-> atmospheric correction ($\M{X}_\text{top-of-atmosphere} \rightarrow \M{X}_\text{bottom-of-atmosphere}$)
		\item[$f_{\Mweight_\text{cl}}(\M{X})$]<4-> cloud/cloudshadow classification (F-Mask, MAJA, CNNs, Cloud Clustering (go FDL!))
		\item[$f_{\text{int}}(\M{X})$]<5-> temporal interpolation to generate equal sample times
		\item[$f_{\Mweight_\text{\dots}}$]<6-> many more...	
	\end{rdescription}
}
\only<7>{
	\vspace{2em}
	\centering\Large In this case: Preprocessing Engine of \includegraphics[width=4em]{images/GAF_logo}
}

\end{frame}


{\setbeamercolor{background canvas}{bg=white}
	\begin{frame}[plain]
	
	\vspace{8em}
	\begin{center}
		\Huge\color{tumbluedark}
		Let's look at Deep Neural Networks
	\end{center}\color{white}
	
\end{frame}
}


\begin{frame}

\frametitle{End-to-End Learning of Deep Neural Networks}
\centering

\Huge
\begin{equation*}
\yhat = f_\Mweight \left(\M{X}\right)
\end{equation*}

\end{frame}



%
%\begin{frame}
%\LARGE
%\centering
%\textbf{Filter this temporal noise by pre-classificaition?}
%\end{frame}

%\begin{frame}
%		\input{images/scl.tikz}
%\end{frame}


%


%\begin{frame}<presentation:1>
%
%\frametitle{Introducing a model to pre-classify clouds?}
%\LARGE
%\centering\figcloudfilteringpipeline
%
%\end{frame}

%\begin{frame}
%\frametitle{Clouds classification works very well...}
%
%\includegraphics[width=\textwidth]{images/Li18_clouds}
%
%\texttt{\small Li, Z., Shen, H., Cheng, Q., Liu, Y., You, S., \& He, Z. (2018). Deep learning based cloud detection for remote sensing images by the fusion of multi-scale convolutional features. arXiv preprint arXiv:1810.05801.}
%\end{frame}
%
%\begin{frame}<presentation:2>
%
%\frametitle{Identifying clouds is rarely the main objective!}
%\LARGE
%\centering\figcloudfilteringpipeline
%
%\end{frame}

\begin{frame}
\frametitle{End-to-End Learning of Deep Neural Networks}
\centering

\begin{tikzpicture}[node distance=.1em]
\node[minimum width=1cm, minimum height=1.5cm, draw,rounded corners](veg) at (0,0){\vegetationsmodell};
\coordinate[below=1em of veg](labelreference);
\node(annotveg) at (labelreference){$f_{{\Mweight}}$};
\visible<1>{
\node[above=2em of veg, font=\small, xshift=4em, text width=15em](annotf){differentiable non-linear function \\ width randomly initialized weights $\Mweight$};
\draw[-stealth] (annotf) -- (veg);
}
\node[right=1em of veg, inner sep=0](input){%
%		$\left(
		\only<1-8>{%
			\rawtimeseriestwo{12-71456800.csv}%
		}%
		\only<9-12>{%
			\begin{tikzpicture}[baseline=-1.5em, xscale=0.3, yscale=-.3]
				\foreach \x in {0,...,15}{
					\node[draw=tumgraylight, circle, fill=b2color, text=white, text opacity=1, font=\small, inner sep=2.5pt](d) at (\x,0){};
					\node[draw=tumgraylight, circle, fill=b3color, text=white, text opacity=1, font=\small, inner sep=2.5pt](d) at (\x,1){};
					\node[draw=tumgraylight, circle, fill=b4color, text=white, text opacity=1, font=\small, inner sep=2.5pt](d) at (\x,2){};
					\node[draw=tumgraylight, circle, fill=b5color, text=white, text opacity=1, font=\small, inner sep=2.5pt](d) at (\x,3){};
					\node[draw=tumgraylight, circle, fill=b6color, text=white, text opacity=1, font=\small, inner sep=2.5pt](d) at (\x,4){};
				}
			\end{tikzpicture}%
		}%
		\only<13-14>{
			\begin{tikzpicture}
			\node[draw, rounded corners, minimum width=1cm, minimum height=1.5cm](preproc){\includegraphics[width=.8cm]{images/icons/gears}};
			\node[right=1.5em of preproc](input){\rawtimeseriestwo{12-71456800_raw.csv}};
			\node[font=\huge,left=0em of input, inner sep=0](bopen){$\Bigg($};
			\node[font=\huge, right=0em of input, inner sep=0](bopen){$\Bigg)$};
			\end{tikzpicture}
		}
		\only<15>{
			\rawtimeseriestwo{12-71456800_raw.csv}
		}
};% \rawtimeseries{prep77770412.csv}
\only<13,14>{
	\node[right=1.5em of annotveg]{$f_{{\Mweight}_\text{preproc}}$};
}


\node(annotinput) at (labelreference -| input){$\M{X}$};
\node[font=\huge,left=0em of input, inner sep=0](bopen){$\Bigg($};
\node[font=\huge, right=0em of input, inner sep=0](bopen){$\Bigg)$};
\node[left= of veg, font=\huge](equals){$=$};
\only<-5>{\node[left= of equals](probas){\drawprobas{10}{30}{10}{20}{10}};}
\visible<5>{\node[left= of equals](probas){\drawprobas{20}{30}{50}{20}{30}};}
\visible<6>{\node[left= of equals](probas){\drawprobas{30}{40}{30}{50}{20}};}
\visible<7>{\node[left= of equals](probas){\drawprobas{20}{20}{20}{80}{10}};}
\visible<8->{\node[left= of equals](probas){\drawprobas{10}{30}{10}{100}{10}};}
\node(annotprobas) at (labelreference -| probas){$\yhat$};
\visible<2->{
	\node[left= 5em of probas](gt){\drawprobas{0}{0}{0}{100}{0}};
	\node(annotgt) at (labelreference -| gt){$\V{y}$};
}
\visible<3->{
	\draw[stealth-stealth] (gt) -- node[midway,above](loss){$\mathcal{L}(\V{y},\yhat)$} (probas);
}


\visible<4->{\draw[-stealth] (loss)  to [out=60,in=135,looseness=1] node[midway,above]{${\Mweight} \leftarrow \Mweight - \frac{\partial \mathcal{L}}{\partial {\Mweight}}$} (veg);}


\only<9-12>{
	\node[above=10em of annotinput, font=\Large\bfseries, text=white, fill=tumbluedark, rounded corners](e1){End};
	\node[above=10em of annotgt, font=\Large\bfseries, text=white, fill=tumbluedark, rounded corners](e2){End};
	\node[above=10em of annotveg, text=white, fill=tumbluedark, font=\Large\bfseries, rounded corners](to){to};
	\draw[thick, tumbluedark] (e1) -- (to) -- (e2);
}

\only<10>{
	\node[below=of annotinput]{Satellite Time Series};
	\node[below=of annotgt]{Crop Types};
}

\only<11>{
	\node[below=of annotinput]{Images (single Band)};
	\node[below=of annotgt]{Cats and Dogs};
}
\only<12>{
	\node[below=of annotinput]{Text and Language};
	\node[below=of annotgt]{Sentiment};
}

\only<14>{
	\coordinate(fpreproc) at ($ (veg)+(3.5em,0) $);
	
	\draw[-stealth, dotted, red, thick] (loss)  to [out=-60,in=-135,looseness=1] node[midway,below]{\xcancel{$\frac{\partial \mathcal{L}}{\partial {\Mweight}_\text{preproc}}$}} ($ (fpreproc)+(0,-2.3em) $);
	
	\node[above right=6em of fpreproc](a){$\Mweight_\text{sel}$: start/end of vegetation period};
	\node[below right=6em of fpreproc](b){$\Mweight_\text{atm}$: atmospheric parameters};
	\node[below right=7em and 2em of fpreproc, text width=14em](c){$\Mweight_\text{cl}$: cloud classifier trained on different training set};
	\draw[-stealth, shorten >=4em] (a.south west) -- (fpreproc);
	\draw[-stealth, shorten >=4em] (b.north west) -- (fpreproc);
	\draw[-stealth, shorten >=4em] (c.north west) -- (fpreproc);
}

\end{tikzpicture}
\end{frame}



%{\setbeamercolor{background canvas}{bg=tumorange}
%	\begin{frame}[plain]
%	\vfill
%	\begin{center}
%		\Huge\color{white}
%		Let's run an experiment...
%	\end{center}
%	\vfill
%\end{frame}
%}

\begin{frame}
	\frametitle{Crop Type Dataset northern Bavaria}
	
	\begin{columns}
		\Large
		\column{.5\textwidth}
		Common project with \includegraphics[height=5mm]{images/GAF_logo}
		
		\vspace{1em}
		\begin{itemize}
			\item crop type labels by the \\ \textbf{Bavarian Ministry of Agriculture}
			\item \textbf{49k} field \textbf{parcels} of 2018
			\item \textbf{34 crop categories}
		\end{itemize}
		\column{.5\textwidth}
		\includegraphics[width=\textwidth]{images/holl}
		\small
		Parcels colored by crop type (40 km by 40 km)
		
	\end{columns}
	
\end{frame}

\begin{frame}<presentation:7>
\frametitle{Two Datasets: Raw and Preprocessed from the same Examples}
%	
%	[format] in this presentation we have the special opportunity to to compare commercially pre-processed satellite imagery with raw imagery on crop type classification in Bavaria Germany.
%	
\begin{tikzpicture}[node distance=.1em]
\node[draw=black, rounded corners, minimum height=3cm, minimum width=4.5cm, label=below:preprocessing, font=\Large\bfseries](gaf){%
	\only<1>{\includegraphics[width=2cm]{images/icons/gears}}%
	\only<2>{$f_{\Mweight_\text{sel}}$}%
	\only<3>{$f_{\Mweight_\text{sel}}\left(f_{\Mweight_\text{atm}}\right)$}%
	\only<4>{$f_{\Mweight_\text{sel}}\left(f_{\Mweight_\text{atm}}\left(f_{\Mweight_\text{cl}}\right)\right)$}%
	\only<5>{$f_{\Mweight_\text{sel}}\left(f_{\Mweight_\text{atm}}\left(f_{\Mweight_\text{cl}}\left(f_{\text{int}}\right)\right)\right)$}%
	\only<6>{$\Mweight_\text{preprocessing}$}%
	\only<7>{\includegraphics[width=2cm]{images/GAF_logo}}%
};
\node[right=1.5em of gaf, inner sep=0](raw){\rawtimeseriestwo{12-71456800_raw.csv}};
\node[font=\huge,left=0em of raw, inner sep=0](bopen){$\Bigg($};
\node[font=\huge,right=0em of raw, inner sep=0](bopen){$\Bigg)$};
\visible<1->{
	\node[right=2em of raw, font=\huge](equals){$=$};
	\node[right=of equals, yshift=-1em](pre){\rawtimeseriestwo{12-71456800.csv}};
}
\only<7>{
\node[above=3em of raw, xshift=-2em](annotraw){raw data};
\node[above=4em of pre, xshift=2em](annotpre){preprocessed data};
\draw[-stealth] (annotraw) -- (raw);
\draw[-stealth] (annotpre) -- (pre);
}
\end{tikzpicture}

\only<1-6>{
	\begin{rdescription}
		\item[$f_{\Mweight_\text{sel}}(\M{X})$]<2-> temporal selection (not considering winter period) where $\Mweight_\text{sel} = \{t_\text{start}, t_\text{end}\}$
		\item[$f_{\Mweight_\text{atm}}(\M{X})$]<3-> atmospheric correction ($\M{X}_\text{top-of-atmosphere} \rightarrow \M{X}_\text{bottom-of-atmosphere}$)
		\item[$f_{\Mweight_\text{cl}}(\M{X})$]<4-> cloud/cloudshadow classification (F-Mask, MAJA, CNNs, Cloud Clustering (go FDL!))
		\item[$f_{\text{int}}(\M{X})$]<5-> temporal interpolation to generate equal sample times
		\item[$f_{\Mweight_\text{\dots}}$]<6-> many more...	
	\end{rdescription}
}
\only<7>{
	\vspace{2em}
	\centering\Large In this case: Preprocessing Engine of \includegraphics[width=4em]{images/GAF_logo}
}

\end{frame}

%\begin{frame}
%	\frametitle{Dataset}
%
%	
%	\begin{tikzpicture}[node distance=.5em]
%	\node(raw1){\rawtimeseriestwo{12-71456800_raw.csv}};
%	\node[right=of raw1](pre1){\rawtimeseriestwo{12-71456800.csv}};
%	
%	\node[below=of raw1](raw2){\rawtimeseriestwo{27-71460091_raw.csv}};
%	\node[right=of raw2](pre2){\rawtimeseriestwo{27-71460091.csv}};
%	
%	\node[above=of raw1]{Raw Sentinel 2 Data};
%	\node[above=of pre1]{\includegraphics[width=1cm]{images/GAF_logo}-preprocessed Sentinel 2 Data};
%	
%	\node[left=of raw1]{meadow};
%	\node[left=of raw2]{wheat};
%	\end{tikzpicture}
%	
%%
%%\begin{tabular}{lcc}
%%	\toprule
%%	Datasets & Raw Sentinel 2 Data & \includegraphics[width=1cm]{images/GAF_logo}-preprocessed Sentinel 2 Data \\
%%	\cmidrule(lr){2-2}\cmidrule(lr){3-3}
%%	{\vspace{1em}meadow} &  & \rawtimeseriestwo{12-71456800.csv} \\
%%	{wheat} & \rawtimeseriestwo{27-71460091_raw.csv} & \rawtimeseriestwo{27-71460091.csv} \\
%%	{32 more} & \rawtimeseriestwo{1-71470174_raw.csv} & \rawtimeseriestwo{1-71470174.csv} \\
%%	\bottomrule
%%\end{tabular}
%
%%\begin{columns}[t]
%%	
%%	\column{.5\textwidth}
%%	Raw Sentinel 2 Data
%%	
%%	\rawtimeseriestwo{12-71456800_raw.csv}
%%	
%%	\column{.5\textwidth}
%%	\includegraphics[width=2cm]{images/GAF_logo}-preprocessed Sentinel 2 Data
%%	
%%	\rawtimeseriestwo{12-71456800.csv}
%%	
%%\end{columns}
%
%%	\begin{tikzpicture}[baseline=-2em, inner sep=0]
%%
%%\begin{axis}[
%%thin,
%%width=6cm,
%%%hide axis,
%%height=3cm,
%%ymin=0, ymax=1.4,
%%no marks,  
%%draw opacity=.8,
%%smooth=0.01
%%]
%%%
%%%
%%\addplot[b11color] table [x=t, y=B11, col sep=comma, forget plot] {images/example/12-71456800_raw.csv};
%%\addplot[b12color] table [x=t, y=B12, col sep=comma] {images/example/12-71456800_raw.csv};
%%
%%\addplot[b5color] table [x=t, y=B05, col sep=comma, forget plot] {images/example/12-71456800_raw.csv};
%%\addplot[b6color] table [x=t, y=B06, col sep=comma, forget plot] {images/example/12-71456800_raw.csv};
%%\addplot[b7color] table [x=t, y=B07, col sep=comma, forget plot] {images/example/12-71456800_raw.csv};
%%\addplot[b8color] table [x=t, y=B08, col sep=comma, forget plot] {images/example/12-71456800_raw.csv};
%%\addplot[b8Acolor] table [x=t, y=B8A, col sep=comma] {images/example/12-71456800_raw.csv};
%%
%%\addplot[b2color] table [x=t, y=B02, col sep=comma, forget plot] {images/example/12-71456800_raw.csv};
%%\addplot[b3color] table [x=t, y=B03, col sep=comma, forget plot] {images/example/12-71456800_raw.csv};
%%\addplot[b4color] table [x=t, y=B04, col sep=comma] {images/example/12-71456800_raw.csv};
%%
%%\end{axis}
%%
%%\end{tikzpicture}
%\end{frame}


\begin{frame}
\frametitle{Four state-of-the-art deep Models for Time Series Classification}

\centering\begin{tabular}{lcccc}
	\toprule
	& LSTM-RNN$^1$ & Transformer$^1$ & MS-ResNet$^3$ & TempCNN$^4$ \\
	\cmidrule(lr){2-2}\cmidrule(lr){3-3}\cmidrule(lr){4-4}\cmidrule(lr){5-5}
	Mechanism & Recurrence & Self-Attention & Convolution & Convolution \\
	Parameters & 100k & 600k & 2M & 433k \\
	\bottomrule
\end{tabular}

\vspace{4em}

{\footnotesize\raggedright

$^1$ \textbf{Hochreiter, S., \& Schmidhuber, J. (1997)}. Long short-term memory. Neural computation, 9(8), 1735-1780.

$^2$ \textbf{Vaswani}, A., Shazeer, N., Parmar, N., Uszkoreit, J., Jones, L., Gomez, A. N., \& Polosukhin, I. \textbf{(2017)}. Attention is all you need. In Advances in neural information processing systems (pp. 5998-6008).

$^3$ \textbf{Wang}, F., Han, J., Zhang, S., He, X., \& Huang, D. \textbf{(2018)}. Csi-net: Unified human body characterization and action recognition. arXiv preprint arXiv:1810.03064.

$^4$ \textbf{Pelletier, C.}, Webb, G. I., \& Petitjean, F. \textbf{(2019)}. Temporal convolutional neural network for the classification of satellite image time series. Remote Sensing, 11(5), 523.

}

\end{frame}


\begin{frame}
\frametitle{Preprocessed versus Raw Data}

%\begin{tabular}{rrlrlrlrl}
%\toprule
%& \multicolumn{2}{c}{MS-ResNet$^1$} & \multicolumn{2}{c}{RNN (LSTM)$^2$} & \multicolumn{2}{c}{Transformer$^3$} & \multicolumn{2}{c}{TempCNN$^4$} \\
%& acc. & $\kappa$ & acc. & $\kappa$ & acc. & $\kappa$ & acc. & $\kappa$ \\
%\cmidrule(lr){2-3}\cmidrule(lr){4-5} \cmidrule(lr){6-7}\cmidrule(lr){8-9}
%\small pre & 0.80 $\pm$ 0.0035 & 0.76 $\pm$ 0.0042 & 0.80 $\pm$ 0.0019 & 0.76 $\pm$ 0.0028 & 0.85 $\pm$ 0.0039 & 0.81 $\pm$ 0.0046 & 0.84 $\pm$ 0.0012 & 0.80 $\pm$ 0.0016 \\
%\small raw & 0.80 $\pm$ 0.0026 & 0.75 $\pm$ 0.0032 & 0.83 $\pm$ 0.0038 & 0.80 $\pm$ 0.0043 & 0.83 $\pm$ 0.0038 & 0.80 $\pm$ 0.0043 & 0.80 $\pm$ 0.0025 & 0.75 $\pm$ 0.0032 \\
%
%%preprocessed & \textbf{0.8484} & \textbf{0.8150} & 0.8059 & 0.7616 & \textbf{0.8116} & \textbf{0.7680} & \textbf{0.8351} & \textbf{0.7971} \\
%%raw 		 & 0.8331 & 0.7971 & \textbf{0.8048} & \textbf{0.7611} & 0.7859 & 0.7420 & 0.7944 & 0.7462 \\
%\midrule
%& \\
%$\Delta$ & & & & \\
%\bottomrule
%\end{tabular}
{
\centering

\begin{tabular}{rccccc}
	\toprule
	\textbf{accuracy} & RNN (LSTM)$^2$ & Transformer$^3$ & MS-ResNet$^1$ & TempCNN$^4$ \\
	\cmidrule(lr){2-2}\cmidrule(lr){3-3} \cmidrule(lr){4-4}\cmidrule(lr){5-5}
	preprocessed & \textbf{.804} $^{\pm.0031}$ & .804 $^{\pm.0011}$ & \textbf{.849}$^{\pm .0041}$ & \textbf{.836} $^{\pm .0012}$ \\
	raw & .801 $^{\pm .0026}$ & \textbf{.842} $^{\pm .0043}$ & {.836} $^{\pm .0033}$ & .799 $^{\pm .0027}$ \\
	\midrule
	$\Delta$ & .003 $^{\pm .0041}$ & -.038 $^{\pm .0045}$ & .013 $^{\pm .0055}$ & .038 $^{\pm .0029}$ \\
	
	\bottomrule
\end{tabular}
\vspace{1em}

\begin{tabular}{rccccc}
	\toprule
	\textbf{kappa} & RNN (LSTM)$^2$ & Transformer$^3$ & MS-ResNet$^1$ & TempCNN$^4$ \\
	\cmidrule(lr){2-2}\cmidrule(lr){3-3} \cmidrule(lr){4-4}\cmidrule(lr){5-5}
	preprocessed & \textbf{.759} $^{\pm .0037}$ & .759 $^{\pm .0017}$ & \textbf{.816} $^{\pm .0048}$ & \textbf{.799} $^{\pm .0015}$ \\
	raw & .756 $^{\pm .0037}$ & \textbf{.808} $^{\pm .0052}$ & .800 $^{\pm .0039}$ & .750 $^{\pm .0036}$ \\
	\midrule
	$\Delta$ & .003 $^{\pm .0048}$ & -.049 $^{\pm .0054}$ & .016 $^{\pm .0060}$ & .049 $^{\pm .0036}$ \\
	
	\bottomrule
\end{tabular}

}
\vspace{1em}

\begin{columns}[t]
	\column{.5\textwidth}
	
	\textbf{Experiments}:
	\begin{itemize}
		\item mean $\pm$ standard deviation of 10 models trained from different random initialization
	\end{itemize}
	
	\column{.5\textwidth}
	
	\textbf{Findings}:
	\begin{itemize}
		\item preprocessed data seems slightly better except for transformer
		\item the difference $\Delta$ is only 0-5\%
	\end{itemize}
	

\end{columns}


\end{frame}



\begin{frame}
\frametitle{Understand Neural Networks}
\framesubtitle{with three Analyses}

\centering\begin{tikzpicture}
\node[minimum width=1cm, minimum height=1.5cm, draw,rounded corners](veg) at (0,0){\vegetationsmodell};
\node[below left=of veg]{Gradient Backprop to X};
\node[below=of veg]{Gated Recurrence};
\node[below right=of veg]{Self-Attention};
\end{tikzpicture}

\end{frame}



%{\setbeamercolor{background canvas}{bg=tumbluedark}
%	\begin{frame}[plain]
%	
%	\begin{center}
%		\Huge\color{tumwhite}
%		Do cloudy observations have influence on the prediction?
%	\end{center}\color{white}
%	
%\end{frame}
%}
%
%{\setbeamercolor{background canvas}{bg=tumorange}
%\begin{frame}[plain]
%\vfill
%\begin{center}
%	\Huge\color{white}
%	Let's look at...
%\end{center}
%\vfill
%\end{frame}
%}
%
%{\setbeamercolor{background canvas}{bg=tumbluedark}
%	\begin{frame}[plain]
%	
%	\begin{center}
%		\vspace{3em}
%		\Huge\color{tumwhite}
%		...backpropagated gradients $\frac{\partial{\yhat}}{\partial\M{X}}$ from the prediction $\yhat$ to the input $\M{X}$
%		
%		\vspace{3em}
%		\raggedleft{\Large using a LSTM-RNN model}
%	\end{center}\color{white}
%	
%\end{frame}
%}

%{\setbeamercolor{background canvas}{bg=white}
%\begin{frame}[plain]
%
%\vspace{8em}
%\begin{center}
%	\Huge\color{tumbluedark}
%	with \textbf{gradients}
%\end{center}\color{white}
%
%\end{frame}
%}

\begin{frame}<presentation:4->
\only<1-3>{\frametitle{Usually we use gradients to adjust $\Mweight$...}}
\only<4->{\frametitle{Input feature importance analysis through gradient backpropagation}}

\centering
\begin{tikzpicture}
\node[font=\Huge](grad){
\only<1-3>{
$\frac{\partial \mathcal{L}(\V{y},f_\Mweight(\M{X}))}{\partial \Mweight}$
}
\only<4->{
\Huge$\frac{\partial \max(\yhat)}{\partial \M{X}}$
}
};

\node[above left=of grad, text width=10em](annotdx){
\only<2,3>{how do we have to change the network weights $\Mweight$...}
\only<4,5>{if we change the input $\M{X}$...} 
};
\node[below right=of grad, text width=10em](annotdy){
\only<3>{... to change (minimize) the loss $\mathcal{L}$?}
\only<5>{... how would the highest predicted score $\max(\yhat)$ change?}
};

\visible<2,3,4,5>{\draw[-stealth, shorten >= 1em, rounded corners] (annotdx) -| ($ (annotdx)!0.5!(grad) $) |- ($ (grad)+(-.5em, -1em) $);}
\visible<3,5>{\draw[-stealth, shorten >= 1em, rounded corners] (annotdy) -| ($ (annotdy)!0.5!(grad) $) |- ($ (grad)+(2.5em, 1em) $);}

\end{tikzpicture}
\end{frame}

\begin{frame}
\frametitle{Can be implemented in four lines}


%\Large $\frac{\partial \max(\yhat)}{\partial \M{X}}$ implementation

\includegraphics[width=\textwidth]{images/dydx_code}

\end{frame}


\begin{frame}
\frametitle{Gradients from $\max(\yhat)$ to \M{X}}

\vspace{-3em}
\begin{tikzpicture}

\def\root{images/rnn_examples/5}


\begin{groupplot}[
group style = {
group size = 1 by 2,
xlabels at=edge bottom,
xticklabels at=edge bottom,
vertical sep=0pt,
},
width=\textwidth,
%		hide axis,
enlargelimits=.1,
height=4cm,
xlabel=observation time $t$,
legend style={at={(1,1.8)},line width=2pt, draw opacity=1},
legend columns=4,
%ymin=-.2, ymax=.2,
%no marks,
]
\nextgroupplot[draw opacity=.8, smooth=0.01, ylabel=$\M{X}$]

\addplot[b2color, mark=*,mark size=.5pt] table [x=t, y=B2, col sep=comma] {\root/x.csv};
\addplot[b3color, mark=*,mark size=.5pt] table [x=t, y=B3, col sep=comma] {\root/x.csv};
\addplot[b4color, mark=*,mark size=.5pt] table [x=t, y=B4, col sep=comma] {\root/x.csv};

\addplot[b5color, mark=*,mark size=.5pt] table [x=t, y=B5, col sep=comma] {\root/x.csv};
\addplot[b6color, mark=*,mark size=.5pt] table [x=t, y=B6, col sep=comma] {\root/x.csv};
\addplot[b7color, mark=*,mark size=.5pt] table [x=t, y=B7, col sep=comma] {\root/x.csv};
\addplot[b8color, mark=*,mark size=.5pt] table [x=t, y=B8, col sep=comma] {\root/x.csv};
\addplot[b8Acolor, mark=*,mark size=.5pt] table [x=t, y=B8A, col sep=comma] {\root/x.csv};

\addplot[b11color, mark=*,mark size=.5pt] table [x=t, y=B11, col sep=comma] {\root/x.csv};
\addplot[b12color, mark=*,mark size=.5pt] table [x=t, y=B12, col sep=comma] {\root/x.csv};


\legend{B02 (blue),B03 (green),B04 (red),B05,B06,B07,B08,B8A,B11,B12}

\node[anchor=west](annotclouds) at (axis cs:-3,1){clouds};
\draw[-stealth](annotclouds) -- (axis cs:3,.75);

\node[anchor=west](annotnoclouds) at (axis cs:30,1){no clouds};
\draw[-stealth](annotnoclouds) -- (axis cs:33,.35);


\nextgroupplot[draw opacity=.8, smooth=0.01, ylabel=$\frac{\partial \max(\yhat)}{\partial \V{X}}$, ymin=-3, ymax=3]
\addplot[b11color, mark=*,mark size=.5pt] table [x=t, y=B11, col sep=comma, forget plot] {\root/dydx.csv};
\addplot[b12color, mark=*,mark size=.5pt] table [x=t, y=B12, col sep=comma] {\root/dydx.csv};

\addplot[b5color, mark=*,mark size=.5pt] table [x=t, y=B5, col sep=comma, forget plot] {\root/dydx.csv};
\addplot[b6color, mark=*,mark size=.5pt] table [x=t, y=B6, col sep=comma, forget plot] {\root/dydx.csv};
\addplot[b7color, mark=*,mark size=.5pt] table [x=t, y=B7, col sep=comma, forget plot] {\root/dydx.csv};
\addplot[b8color, mark=*,mark size=.5pt] table [x=t, y=B8, col sep=comma, forget plot] {\root/dydx.csv};
\addplot[b8Acolor, mark=*,mark size=.5pt] table [x=t, y=B8A, col sep=comma] {\root/dydx.csv};

\addplot[b2color, mark=*,mark size=.5pt] table [x=t, y=B2, col sep=comma, forget plot] {\root/dydx.csv};
\addplot[b3color, mark=*,mark size=.5pt] table [x=t, y=B3, col sep=comma, forget plot] {\root/dydx.csv};
\addplot[b4color, mark=*,mark size=.5pt] table [x=t, y=B4, col sep=comma] {\root/dydx.csv};

\node[anchor=west](annotcloudsgrad) at (axis cs:-3,2.5){no gradients};
\draw[-stealth](annotcloudsgrad) -- (axis cs:3,0);

\node[anchor=west](annotnoclouds) at (axis cs:35,2.5){gradients};
\draw[-stealth](annotnoclouds) -- (axis cs:33,1.8);

\end{groupplot}
\end{tikzpicture}
\end{frame}


\begin{frame}<presentation:0>
\begin{tikzpicture}

\frametitle{Gradients from $\max(\yhat)$ to \M{X}}
\framesubtitle{Example 2}

\def\root{images/rnn_examples/6}


\begin{groupplot}[
group style = {
group size = 1 by 2,
xlabels at=edge bottom,
xticklabels at=edge bottom,
vertical sep=0pt,
},
width=\textwidth,
%		hide axis,
enlargelimits=.1,
height=4cm,
xlabel=observation time $t$,
%ymin=-.2, ymax=.2,
%no marks,
]
\nextgroupplot[draw opacity=.8, smooth=0.01, ylabel=$\M{X}$]
\addplot[b11color, mark=*,mark size=.5pt] table [x=t, y=B11, col sep=comma, forget plot] {\root/x.csv};
\addplot[b12color, mark=*,mark size=.5pt] table [x=t, y=B12, col sep=comma] {\root/x.csv};

\addplot[b5color, mark=*,mark size=.5pt] table [x=t, y=B5, col sep=comma, forget plot] {\root/x.csv};
\addplot[b6color, mark=*,mark size=.5pt] table [x=t, y=B6, col sep=comma, forget plot] {\root/x.csv};
\addplot[b7color, mark=*,mark size=.5pt] table [x=t, y=B7, col sep=comma, forget plot] {\root/x.csv};
\addplot[b8color, mark=*,mark size=.5pt] table [x=t, y=B8, col sep=comma, forget plot] {\root/x.csv};
\addplot[b8Acolor, mark=*,mark size=.5pt] table [x=t, y=B8A, col sep=comma] {\root/x.csv};

\addplot[b2color, mark=*,mark size=.5pt] table [x=t, y=B2, col sep=comma, forget plot] {\root/x.csv};
\addplot[b3color, mark=*,mark size=.5pt] table [x=t, y=B3, col sep=comma, forget plot] {\root/x.csv};
\addplot[b4color, mark=*,mark size=.5pt] table [x=t, y=B4, col sep=comma] {\root/x.csv};

\nextgroupplot[draw opacity=.8, smooth=0.01, ylabel=$\frac{\partial \max(\yhat)}{\partial \V{X}}$]
\addplot[b11color, mark=*,mark size=.5pt] table [x=t, y=B11, col sep=comma, forget plot] {\root/dydx.csv};
\addplot[b12color, mark=*,mark size=.5pt] table [x=t, y=B12, col sep=comma] {\root/dydx.csv};

\addplot[b5color, mark=*,mark size=.5pt] table [x=t, y=B5, col sep=comma, forget plot] {\root/dydx.csv};
\addplot[b6color, mark=*,mark size=.5pt] table [x=t, y=B6, col sep=comma, forget plot] {\root/dydx.csv};
\addplot[b7color, mark=*,mark size=.5pt] table [x=t, y=B7, col sep=comma, forget plot] {\root/dydx.csv};
\addplot[b8color, mark=*,mark size=.5pt] table [x=t, y=B8, col sep=comma, forget plot] {\root/dydx.csv};
\addplot[b8Acolor, mark=*,mark size=.5pt] table [x=t, y=B8A, col sep=comma] {\root/dydx.csv};

\addplot[b2color, mark=*,mark size=.5pt] table [x=t, y=B2, col sep=comma, forget plot] {\root/dydx.csv};
\addplot[b3color, mark=*,mark size=.5pt] table [x=t, y=B3, col sep=comma, forget plot] {\root/dydx.csv};
\addplot[b4color, mark=*,mark size=.5pt] table [x=t, y=B4, col sep=comma] {\root/dydx.csv};

\end{groupplot}
\end{tikzpicture}
\end{frame}


\begin{frame}<presentation:0>
\begin{tikzpicture}

\frametitle{Gradients from $\max(\yhat)$ to \M{X}}
\framesubtitle{Example 3}

\def\root{images/rnn_examples/7}


\begin{groupplot}[
group style = {
group size = 1 by 2,
xlabels at=edge bottom,
xticklabels at=edge bottom,
vertical sep=0pt,
},
width=\textwidth,
%		hide axis,
enlargelimits=.1,
height=4cm,
xlabel=observation time $t$,
%ymin=-.2, ymax=.2,
%no marks,
]
\nextgroupplot[draw opacity=.8, smooth=0.01, ylabel=$\M{X}$]
\addplot[b11color, mark=*,mark size=.5pt] table [x=t, y=B11, col sep=comma, forget plot] {\root/x.csv};
\addplot[b12color, mark=*,mark size=.5pt] table [x=t, y=B12, col sep=comma] {\root/x.csv};

\addplot[b5color, mark=*,mark size=.5pt] table [x=t, y=B5, col sep=comma, forget plot] {\root/x.csv};
\addplot[b6color, mark=*,mark size=.5pt] table [x=t, y=B6, col sep=comma, forget plot] {\root/x.csv};
\addplot[b7color, mark=*,mark size=.5pt] table [x=t, y=B7, col sep=comma, forget plot] {\root/x.csv};
\addplot[b8color, mark=*,mark size=.5pt] table [x=t, y=B8, col sep=comma, forget plot] {\root/x.csv};
\addplot[b8Acolor, mark=*,mark size=.5pt] table [x=t, y=B8A, col sep=comma] {\root/x.csv};

\addplot[b2color, mark=*,mark size=.5pt] table [x=t, y=B2, col sep=comma, forget plot] {\root/x.csv};
\addplot[b3color, mark=*,mark size=.5pt] table [x=t, y=B3, col sep=comma, forget plot] {\root/x.csv};
\addplot[b4color, mark=*,mark size=.5pt] table [x=t, y=B4, col sep=comma] {\root/x.csv};

\nextgroupplot[draw opacity=.8, smooth=0.01, ylabel=$\frac{\partial \max(\yhat)}{\partial \V{X}}$]
\addplot[b11color, mark=*,mark size=.5pt] table [x=t, y=B11, col sep=comma, forget plot] {\root/dydx.csv};
\addplot[b12color, mark=*,mark size=.5pt] table [x=t, y=B12, col sep=comma] {\root/dydx.csv};

\addplot[b5color, mark=*,mark size=.5pt] table [x=t, y=B5, col sep=comma, forget plot] {\root/dydx.csv};
\addplot[b6color, mark=*,mark size=.5pt] table [x=t, y=B6, col sep=comma, forget plot] {\root/dydx.csv};
\addplot[b7color, mark=*,mark size=.5pt] table [x=t, y=B7, col sep=comma, forget plot] {\root/dydx.csv};
\addplot[b8color, mark=*,mark size=.5pt] table [x=t, y=B8, col sep=comma, forget plot] {\root/dydx.csv};
\addplot[b8Acolor, mark=*,mark size=.5pt] table [x=t, y=B8A, col sep=comma] {\root/dydx.csv};

\addplot[b2color, mark=*,mark size=.5pt] table [x=t, y=B2, col sep=comma, forget plot] {\root/dydx.csv};
\addplot[b3color, mark=*,mark size=.5pt] table [x=t, y=B3, col sep=comma, forget plot] {\root/dydx.csv};
\addplot[b4color, mark=*,mark size=.5pt] table [x=t, y=B4, col sep=comma] {\root/dydx.csv};

\end{groupplot}
\end{tikzpicture}
\end{frame}

{\setbeamercolor{background canvas}{bg=black}
	\begin{frame}[plain]
	\vfill
	\begin{columns}
		\column{.25\textwidth}
		\color{tumwhite}
		
		\Huge
		Gated Recurrence
		
		\column{.75\textwidth}
		\raggedleft
		\tikzstyle{rec} = [draw=white, fill=black, rounded corners, inner sep=2em]
		\begin{tikzpicture}[node distance=2em]
			\node[rec](rec){};
			\draw[-stealth, white, very thick] ($ (rec)+(0,5em) $) -- (rec) node[at start, above]{$\V{x}_t$};
			\draw[-stealth, white, very thick] (rec) -- ($ (rec)+(0,-5em) $) node[at end, below]{$\V{h}_t$};
			
			\draw[-stealth, white, very thick] (rec.south) to[in=270, out=250, looseness=2] ($ (rec.west)+(-1.5em,0) $) to[in=110, out=90, looseness=2] (rec.north) node[at start, left, xshift=-4em]{$\V{h}_{t-1}$};
			
			\node[rec, right=3em of rec](unrolled0){};
			\draw[-stealth, white, very thick] ($ (unrolled0)+(0,5em) $) -- (unrolled0) node[at start, above]{$\V{x}_1$};
			\draw[-stealth, white, very thick] (unrolled0) -- ($ (unrolled0)+(0,-5em) $) node[at end, below]{$\V{h}_1$};;
			
			\coordinate(sep) at ($ (rec)!0.5!(unrolled0) $);
			
			\draw[dotted, thick] ($ (sep)+(0,-5em) $) -- ($ (sep)+(0,5em) $);

			\node[rec, right=of unrolled0](unrolled1){};
			\draw[-stealth, white, very thick] ($ (unrolled1)+(0,5em) $) -- (unrolled1) node[at start, above]{$\V{x}_2$};
			\draw[-stealth, white, very thick] (unrolled1) -- ($ (unrolled1)+(0,-5em) $) node[at end, below]{$\V{h}_2$};
			
			\node[rec, right=of unrolled1](unrolled2){};
			\draw[-stealth, white, very thick] ($ (unrolled2)+(0,5em) $) -- (unrolled2) node[at start, above]{$\V{x}_3$};
			\draw[-stealth, white, very thick] (unrolled2) -- ($ (unrolled2)+(0,-5em) $) node[at end, below]{$\V{h}_3$};;
			
			\node[rec, right=of unrolled2](unrolled3){};
			\draw[-stealth, white, very thick] ($ (unrolled3)+(0,5em) $) -- (unrolled3) node[at start, above]{$\V{x}_4$};
			\draw[-stealth, white, very thick] (unrolled3) -- ($ (unrolled3)+(0,-5em) $) node[at end, below]{$\V{h}_4$};
			
			\draw[-stealth, white, very thick] (unrolled0) -- (unrolled1) node[near end, above]{$\V{h}_1$};
			\draw[-stealth, white, very thick] (unrolled1) -- (unrolled2) node[near end, above]{$\V{h}_2$};
			\draw[-stealth, white, very thick] (unrolled2) -- (unrolled3) node[near end, above]{$\V{h}_3$};
			
			
		\end{tikzpicture}
		
	\end{columns}
	\vfill
\end{frame}
}

\newcommand{\gates}[6]{
	\tikzstyle{gate} = [inner sep=0em, draw, rounded corners, minimum width=1.5em, minimum height=1.5em]
	\begin{tikzpicture}[xscale=0.75, yscale=-0.66]
		\node[gate,#2](i) at (0,0){$\V{i}_{#1}$};
		\node[gate,#3](f) at (0,1){$\V{f}_{#1}$};
		\node[gate,#4](g) at (1,0){$\V{g}_{#1}$};
		\node[gate,#5](o) at (1,1){$\V{o}_{#1}$};
		\node[gate,#6](c) at (0.5,2){$\V{c}_{#1}$};
		
		\node[font=\tiny, inner sep=0, minimum width=0, minimum height=0, rounded corners=0](thetai) at (i.south east){$\Mweight_i$};
		\node[font=\tiny, inner sep=0, minimum width=0, minimum height=0, rounded corners=0](thetai) at (f.south east){$\Mweight_f$};
		\node[font=\tiny, inner sep=0, minimum width=0, minimum height=0, rounded corners=0](thetai) at (g.south east){$\Mweight_g$};
		\node[font=\tiny, inner sep=0, minimum width=0, minimum height=0, rounded corners=0](thetai) at (o.south east){$\Mweight_o$};
	\end{tikzpicture}
}

\begin{frame}

%1 gates
%2 input gate
%3 modulation gate
%4 forget gate
%5 output gate
%6 cell state
%7 output

\frametitle{Long Short Term Memory (LSTM) Recurrent Network}
\tikzstyle{focused} = [fill=tumbluedark, text=white, draw, rounded corners, thin]
\tikzstyle{unfocused} = [fill=white, draw, rounded corners, thin]
\tikzstyle{background} = [fill=white, draw, rounded corners, thin, opacity=0.2]

\tikzstyle{rec} = [draw=tumbluedark, fill=none, rounded corners, inner sep=0em, minimum width=5em, minimum height=6em]
\tikzstyle{conn} = [-stealth, tumbluedark, very thick]

\begin{columns}
	
\column{.5\textwidth}
	
\begin{tikzpicture}[node distance=3em]

\node[rec, right=3em of rec, background](unrolled0){
	\only<1-5,7->{\gates{1}{unfocused}{unfocused}{unfocused}{unfocused}{unfocused}}
	\only<6>{\gates{1}{unfocused}{unfocused}{unfocused}{unfocused}{focused}}
};
\node[above=of unrolled0, background](x0){$\V{x}_1$};
\draw[conn, background] (x0) -- (unrolled0);
\draw[conn, background] (unrolled0) -- ($ (unrolled0)+(0,-5em) $);
\node[below=of unrolled0, background](h0){$\V{h}_1$};

\node[rec, right=of unrolled0](unrolled1){%i f g o c
	\only<1>{\gates{2}{background}{background}{background}{background}{unfocused}}%
	\only<2>{\gates{2}{focused}{unfocused}{unfocused}{unfocused}{unfocused}}%
	\only<3>{\gates{2}{unfocused}{unfocused}{focused}{unfocused}{unfocused}}%
	\only<4>{\gates{2}{unfocused}{focused}{unfocused}{unfocused}{unfocused}}%
	\only<5>{\gates{2}{unfocused}{unfocused}{unfocused}{focused}{unfocused}}%
	\only<6>{\gates{2}{focused}{focused}{focused}{unfocused}{focused}}%
	\only<7>{\gates{2}{unfocused}{unfocused}{unfocused}{focused}{focused}}%
};
\visible<2-5>{\node[above=of unrolled1, focused](x1){$\V{x}_2$};}
\visible<1,6->{\node[above=of unrolled1, unfocused](x1){$\V{x}_2$};}
\draw[conn] (x1) -- (unrolled1);
\draw[conn] (unrolled1) -- ($ (unrolled1)+(0,-5em) $);

\visible<7>{\node[below=of unrolled1, focused](h0){$\V{h}_2$};}
\visible<1-6>{\node[below=of unrolled1, unfocused](h0){$\V{h}_2$};}
%													  i f g o c
%\phantom{
%	\node[rec, right=of unrolled1](unrolled2){\gates{3}{unfocused}{unfocused}{unfocused}{unfocused}{unfocused}};
%	\draw[conn] ($ (unrolled2)+(0,5em) $) -- (unrolled2) node[at start, above]{$\V{x}_3$};
%	\draw[conn] (unrolled2) -- ($ (unrolled2)+(0,-5em) $) node[at end, below]{$\V{h}_3$};;
%}
\draw[conn] (unrolled0) -- (unrolled1);
\visible<1,5->{
	\node[unfocused, yshift=1em] at ($ (unrolled0)!0.5!(unrolled1) $){$\V{h}_1$};
}
\visible<2-5>{
	\node[focused, yshift=1em] at ($ (unrolled0)!0.5!(unrolled1) $){$\V{h}_1$};
}
\visible<1-5,7->{
	\node[unfocused, yshift=-1em] at ($ (unrolled0)!0.5!(unrolled1) $){$\V{c}_1$};
}
\visible<6>{
	\node[focused, yshift=-1em] at ($ (unrolled0)!0.5!(unrolled1) $){$\V{c}_1$};
}
%\visible<2-4>{
%	\draw[conn] (unrolled0) -- (unrolled1) node[midway, above, focused]{$\V{h}_1$} node[midway, below, focused]{$\V{c}_1$};
%}

%\visible<1-6>{\draw[conn] (unrolled1) -- (unrolled2) node[midway, above, unfocused]{$\V{h}_2$};}
%\visible<7>{\draw[conn] (unrolled1) -- (unrolled2) node[midway, above, focused]{$\V{h}_2$};}


\end{tikzpicture}

\column{.5\textwidth}

Long Short-Term Memory Neural Network (LSTM):
\begin{equation*}
	\V{h}_t, \V{c}_t \leftarrow \V{x}_t ,\V{h}_{t-1}, \V{c}_{t-1}
\end{equation*}
controlled by internal Gates:

\visible<2->{$\V{i_t} = \sigma\left(\V{x}_t\Mweight_{xi} + \V{h}_{t-1}\Mweight_{hi}\right)$}
\visible<3->{$\V{g_t} = \tanh\left(\V{x}_t\Mweight_{xg} + \V{h}_{t-1}\Mweight_{hg}\right)$}
\visible<4->{$\V{f_t} = \sigma\left(\V{x}_t\Mweight_{xf} + \V{h}_{t-1}\Mweight_{hf}\right)$}
\visible<5->{$\V{o_t} = \sigma\left(\V{x}_t\Mweight_{xo} + \V{h}_{t-1}\Mweight_{ho}\right)$}
\vspace{1em}

\visible<6->{$\V{c_t} = \V{f}_t*\V{c}_{t-1} + \V{i}_t \odot \V{g}_t$}

\visible<7->{$\V{h_t} = \V{o}_t \odot \tanh(\V{c_t})$}

\end{columns}

\end{frame}

%
%\begin{frame}<presentation:3>
%\frametitle{Employ ConvRNNs for Vegetation Land Cover Classification directly}
%%	\input{images/seqencnetwork.tikz}
%%	\figseqencnetwork
%%\tikzsetnextfilename{network}

\tikzstyle{operator} = [draw, circle, fill=tumbluemedium, draw=tumbluemedium, inner sep=0, text=white]
%\tikzstyle{function} = [draw, rectangle, fill=tumbluemedium, draw=tumbluemedium, text=white]
\tikzstyle{gate} = [fill=tumivory,draw,rounded corners]

\tikzstyle{dummy} = [inner sep=0]
\tikzstyle{flow} = [rounded corners]
\tikzstyle{endflow} = [-stealth,flow]
\tikzstyle{beginflow} = [stealth-,flow]

\tikzstyle{bigpassbox} = [opacity=.2, rounded corners, draw=none]

% defaultvalue -> might be replaced later
\colorlet{tensorcolor}{forwardcolor}

\tikzstyle{bigbox} = [rectangle, draw=tumivory, thick, fill=tumgraylight, rounded corners, 
inner sep=.5ex]

\tikzstyle{wireframe} = [draw=tumgray]
\tikzstyle{image} = [inner sep=0, fill=none, minimum size=\imagewidth]

\tikzset{pic shift/.store in=\shiftcoord,
	pic shift={(0,0)},
	pics/seqlstmfw/.style={
		code={
		\begin{scope}[shift={\shiftcoord},xscale=1.3,yscale=.8]
			
			\node[dummy] (bl) at (0,0){}; % bottom left
			\node[dummy] (tr) at (1,1){}; % top right
			
			\node[dummy] (br) at ($ (bl -| tr) $){}; % bottom right
			\node[dummy] (tl) at ($ (bl |- tr) $){}; % top left
			
			\node[fit=(bl) (tr),bigbox] (-C) {};
			
			% input coordinate for rounded draw lines -> slightly right of tl
			\coordinate (-input) at (0.1,1); % top left
			
			% output coordinate for rounded draw lines -> slightly left of br
			\coordinate (-coutput) at (0.9,0); % bottom right
			\coordinate (-cinput) at (0.1,0); % bottom left
			\coordinate (-houtput) at (0.9,1); % bottom right
			
%			% gate distance
			\def\d{1/6}
			
			% gate heights
			\def\h{1/3}
			
			\coordinate (f)  at bl+(1*\d,0);
			\coordinate (i)  at bl+(2*\d,0);
			\coordinate (j)  at bl+(3*\d,0);
			\coordinate (o)  at bl+(4*\d,0);
			\coordinate (out) at bl+(5*\d,0);
			
			\coordinate (gates) at (0,2*\h);
			
			%\node[above=of tl](xt){$x_{t}$};
			%\node[left=of tl](htminus1){$h_{t-1}$};
			
			%\node[below=of br](ct){$c_{t}$};
			
			\node[gate](fgate) at ($ (gates -| f) $){};
			\node[gate](igate) at ($ (gates -| i) $){};
			\node[gate](jgate) at ($ (gates -| j) $){};
			\node[gate](ogate) at ($ (gates -| o) $){};
			
%			\coordinate (htminus1) at bl+(-.5,0);
%			\coordinate (ht) at bl+(-.5,0);
%			
			% forget gate
			\node[operator](fmult) at ($ (bl -| fgate) $) {};
			\draw[endflow] (-input) -| (fgate) -- (fmult); 
			
%			%j
			\node[operator](jmult) at ([shift={(0,-1*\h)}]jgate) {};
			\node[operator](cadd) at ($ (bl -| jgate) $) {};
			\draw[endflow] (-input) -| (jgate) -- (jmult);
			\draw[endflow] (jmult) -- (cadd); 			

%			%i	
			\draw[endflow] (-input) -| (igate) |- (jmult); 
%
%%			% outpu
			\node[operator](outtanh) at ([shift={(0,1*\h)}]out) {};
%			
%			%o 
			\draw[endflow] (tl) -| (ogate) |- (outtanh);
			\draw[flow] (outtanh) |- (-houtput);
%			
%			% output flow
			\draw[endflow] (cadd) -| (outtanh);
			\draw[flow] (-cinput) -- (fmult) -- (cadd) -- (-coutput);
%			
			
			% debug
%			\node at (gates) {\tiny{gates}};
%			\node at (-input) {\tiny{-input}};
%			\node at (-coutput) {\tiny{-coutput}};
%			\node at (-houtput) {\tiny{-houtput}};
%			\node at (f) {\tiny{f}};
%			\node at (i) {\tiny{i}};
%			\node at (j) {\tiny{j}};
%			\node at (o) {\tiny{o}};
%			\node at (tl) {\tiny{tl}};
%			\node at (br) {\tiny{br}};
%			\node at (bl) {\tiny{bl}};
%			\node at (tr) {\tiny{tr}};
%			\node at (out) {\tiny{out}};
			
		\end{scope}
		}
	}
}

\tikzset{pic shift/.store in=\shiftcoord,
	pic shift={(0,0)},
	pics/seqlstmbw/.style={
		code={
			\begin{scope}[shift={\shiftcoord},xscale=1.3,yscale=-.8]
				
				\node[dummy] (bl) at (0,0){}; % bottom left
				\node[dummy] (tr) at (1,1){}; % top right
				
				\node[dummy] (br) at ($ (bl -| tr) $){}; % bottom right
				\node[dummy] (tl) at ($ (bl |- tr) $){}; % top left
				
				\node[fit=(bl) (tr),bigbox] (-C) {};
				
				% input coordinate for rounded draw lines -> slightly right of tl
				\coordinate (-input) at (0.1,1); % top left
				
				% output coordinate for rounded draw lines -> slightly left of br
				\coordinate (-coutput) at (0.9,0); % bottom right
				\coordinate (-cinput) at (0.1,0); % bottom left
				\coordinate (-houtput) at (0.9,1); % top right
				
				%			% gate distance
				\def\d{1/6}
				
				% gate heights
				\def\h{1/3}
				
				\coordinate (f)  at bl+(1*\d,0);
				\coordinate (i)  at bl+(2*\d,0);
				\coordinate (j)  at bl+(3*\d,0);
				\coordinate (o)  at bl+(4*\d,0);
				\coordinate (out) at bl+(5*\d,0);
				
				\coordinate (gates) at (0,2*\h);
				
				%\node[above=of tl](xt){$x_{t}$};
				%\node[left=of tl](htminus1){$h_{t-1}$};
				
				%\node[below=of br](ct){$c_{t}$};
				
				\node[gate](fgate) at ($ (gates -| f) $){};
				\node[gate](igate) at ($ (gates -| i) $){};
				\node[gate](jgate) at ($ (gates -| j) $){};
				\node[gate](ogate) at ($ (gates -| o) $){};
				
				%			\coordinate (htminus1) at bl+(-.5,0);
				%			\coordinate (ht) at bl+(-.5,0);
				%			
				% forget gate
				\node[operator](fmult) at ($ (bl -| fgate) $) {};
				\draw[endflow] (-input) -| (fgate) -- (fmult); 
				
				%			%j
				\node[operator](jmult) at ([shift={(0,-1*\h)}]jgate) {};
				\node[operator](cadd) at ($ (bl -| jgate) $) {};
				\draw[endflow] (-input) -| (jgate) -- (jmult);
				\draw[endflow] (jmult) -- (cadd); 			
				
				%			%i	
				\draw[endflow] (-input) -| (igate) |- (jmult); 
				%
				%%			% outpu
				\node[operator](outtanh) at ([shift={(0,1*\h)}]out) {};
				%			
				%			%o 
				\draw[endflow] (tl) -| (ogate) |- (outtanh);
				\draw[flow] (outtanh) |- (-houtput);
				%			
				%			% output flow
				\draw[endflow] (cadd) -| (outtanh);
				\draw[flow] (-cinput) -- (fmult) -- (cadd) -- (-coutput);
				%			
				
				% debug
				%			\node at (gates) {\tiny{gates}};
				%			\node at (-input) {\tiny{-input}};
				%			\node at (-coutput) {\tiny{-coutput}};
				%			\node at (-houtput) {\tiny{-houtput}};
				%			\node at (f) {\tiny{f}};
				%			\node at (i) {\tiny{i}};
				%			\node at (j) {\tiny{j}};
				%			\node at (o) {\tiny{o}};
				%			\node at (tl) {\tiny{tl}};
				%			\node at (br) {\tiny{br}};
				%			\node at (bl) {\tiny{bl}};
				%			\node at (tr) {\tiny{tr}};
				%			\node at (out) {\tiny{out}};
				
			\end{scope}
		}
	}
}

\newcommand{\tensorcube}[4]{
	\def\w{#1}
	\def\h{#2}
	\def\d{#3}
%	\def\img{#4}
	
	\begin{scope}[perspective3d]
		
		% bw back frame
		\begin{scope}[canvas is yz plane at x=\d]
			\node[transform shape,image, minimum size=\w, anchor=south west, fill=none, wireframe](back){};
		\end{scope}
		
		% front image
		\begin{scope}[canvas is yz plane at x=0]
			\node[transform shape,image, anchor=south west, opacity=1](front){
				%\includegraphics[width=\w]{\img}
				#4
				};
		\end{scope}
		
		% front frame
		\begin{scope}[canvas is yz plane at x=0]
			\node[transform shape,image, minimum size=\w, anchor=south west, fill=none, wireframe](front){};
		\end{scope}
		
		% fill right side
		\fill[tensorcolor,opacity=.5, wireframe] (front.south east) -- (back.south east) -- (back.north east) -- (front.north east);
		% fill top side
		\fill[tensorcolor,opacity=.5,opacity=0.6, wireframe] (front.north east) -- (back.north east) -- (back.north west) -- (front.north west);
		% fill right side
		
		
		%\draw[] (front.south west) -- (back.south west);
		
	\end{scope}
}

\newcommand{\concatstates}[4]{
	\def\w{#1}
	\def\h{#2}
	\def\d{#3}
	\def\img{#4}
	
	\begin{scope}[perspective3d]
		
		% bw back frame
		\begin{scope}[canvas is yz plane at x=2*\d]
			\node[transform shape,image, minimum size=\w, anchor=south west, fill=none, wireframe](back){};
		\end{scope}
		
		% middle frame
		\begin{scope}[canvas is yz plane at x=\d]
			\node[transform shape,image, minimum size=\w, anchor=south west, fill=none, wireframe](middle){};
		\end{scope}
		
		% front image
		\begin{scope}[canvas is yz plane at x=0]
			\node[transform shape,image, anchor=south west](front){\includegraphics[width=\w]{\img}};
		\end{scope}
		
		% front frame
		\begin{scope}[canvas is yz plane at x=0]
			\node[transform shape,image, minimum size=\w, anchor=south west, fill=none,wireframe](front){};
		\end{scope}
		
		% fill right side
		\fill[backwardcolor,opacity=.5, wireframe] (front.south east) -- (middle.south east) -- (middle.north east) -- (front.north east);
		% fill top side
		\fill[backwardcolor,opacity=.5,opacity=0.6, wireframe] (front.north east) -- (middle.north east) -- (middle.north west) -- (front.north west);
		% fill right side
		
		\fill[forwardcolor,opacity=.5, wireframe] (middle.south east) -- (back.south east) -- (back.north east) -- (middle.north east);
		% fill top side
		\fill[forwardcolor,opacity=.5,opacity=0.6, wireframe] (middle.north east) -- (back.north east) -- (back.north west) -- (middle.north west);
		
		%\draw[] (front.south west) -- (back.south west);
		
		
	\end{scope}
}

\begin{tikzpicture}[scale=1, node distance=1em]%,show background rectangle,background rectangle/.style={draw=red}]

%\matrix (m) [matrix of nodes, ampersand replacement=\&]{

\def\d{1.6}
\def\encoderheight{0.4}
\def\decoderheight{-0.4}
\def\imagewidth{8mm}
\def\stateimagewidth{8mm}
\def\classimagewidth{8mm}

\def\rgbone{images/network/48px/rgb1}
\def\rgbtwo{images/network/48px/rgb2}
\def\rgbthree{images/network/48px/rgb3}
\def\rgbfour{images/network/48px/rgb4}
\def\prediction{images/network/48px/prediction}
\def\groundtruth{images/network/48px/ground_truth}
\def\activation{images/network/48px/maize}
\def\state{images/network/48px/state}


%\draw [bigpassbox, fill=forwardcolor](1,.25) rectangle (10,4);

\node[bigpassbox, fill=forwardcolor, rectangle, minimum width=7cm,minimum height=3cm, anchor=south west, label={[shift={(2em,-3.5ex)}]north:jan $\rightarrow$ dec}] (forwardbox) at (1,.15) {};
%\draw [bigpassbox, fill=backwardcolor](1,-.25) rectangle (10,-4);

\draw pic (fw1) at (\d,\encoderheight) {seqlstmfw};% \&
\node[above=of fw1tl, label=above:{$\VInput_{0}$}](xfw1){\includegraphics[width=\imagewidth]{\rgbone}};%images/network/time1_x

\draw pic (fw2) at (2*\d,\encoderheight) {seqlstmfw};
\node[above=of fw2-input, label=above:{$\VInput_{1}$}](xfw2){\includegraphics[width=\imagewidth]{\rgbtwo}}; % images/network/time11_x

\draw pic (fw3) at (3*\d,\encoderheight) {seqlstmfw};
\node[above=of fw3-input](xfw3){$\dots$};

\draw pic (fw4) at (4*\d,\encoderheight) {seqlstmfw};
\node[above=of fw4-input, label=above:{$\VInput_T$}](xfw4){\includegraphics[width=\imagewidth]{\rgbfour}}; % images/network/time30_x

\node[left=of fw1-input](enczerostateh){$\V{0}$};
\node[left=of fw1-cinput](enczerostatec){$\V{0}$};
\draw[endflow] (enczerostateh) -- (fw1-input);
\draw[endflow] (enczerostatec) -- (fw1-cinput);

% draw connections from input to cells
\draw[flow] (xfw1) |- (fw1-input);
\draw[flow] (xfw2) |- (fw2-input);
\draw[flow] (xfw3) |- (fw3-input);
\draw[flow] (xfw4) |- (fw4-input);

% draw hidden connections between cells
\draw[endflow] (fw1-houtput) -- (fw2-input);
\draw[endflow] (fw2-houtput) -- (fw3-input);
\draw[endflow] (fw3-houtput) -- (fw4-input);

% draw hidden connections between cells
\draw[endflow] (fw1-coutput) -- (fw2-cinput);
\draw[endflow] (fw2-coutput) -- (fw3-cinput);
\draw[endflow] (fw3-coutput) -- (fw4-cinput);

\node[right= 2.5em of fw4-coutput](cfw){$\VCellState_T^\text{fw}$};

\draw[endflow] (fw4-coutput) -- (cfw);

\visible<2->{

\node[bigpassbox, fill=backwardcolor, rectangle, minimum width=7cm,minimum height=3cm, anchor=north west, label={[shift={(2em,3.5ex)}]south:dec $\rightarrow$ jan}] (forwardbox) at (1,-.15) {};

\draw pic (bw1) at (1*\d,\decoderheight) {seqlstmbw};% \&
\node[below=of bw1tl, label=below:{$\VInput_T$}](xbw1){\includegraphics[width=\imagewidth]{\rgbfour}}; % images/network/time30_x

\draw pic (bw2) at (2*\d,\decoderheight) {seqlstmbw};
\node[below=of bw2tl, label=below:{$\VInput_{T-1}$}](xbw2){\includegraphics[width=\imagewidth]{\rgbthree}}; %images/network/time20_x

\draw pic (bw3) at (3*\d,\decoderheight) {seqlstmbw};
\node[below=of bw3tl](xbw3){$\dots$};

\draw pic (bw4) at (4*\d,\decoderheight) {seqlstmbw};
\node[below=of bw4tl, label=below:{$\VInput_0$}](xbw4){\includegraphics[width=\imagewidth]{\rgbone}};

%\node[anchor=center](state) at ($(enc4-coutput)!0.5!(dec1-cinput)$){context state vector $\VCellState_T$};

\node[left=of bw1-input](deczerostateh){$\V{0}$};
\node[left=of bw1-cinput](deczerostatec){$\V{0}$};

\draw[endflow] (deczerostateh) -- (bw1-input);
\draw[endflow] (deczerostatec) -- (bw1-cinput);

% draw state
%\draw[flow] (enc4-coutput) -- ++(.2,0) |- (state);
%\draw[beginflow] (dec1-cinput) -- ++(-.2,0) |- (state);

% draw connections from cells to output
\draw[flow] (xbw1) |- (bw1-input);
\draw[flow] (xbw2) |- (bw2-input);
\draw[flow] (xbw3) |- (bw3-input);
\draw[flow] (xbw4) |- (bw4-input);

\draw[endflow] (bw1-houtput) -- (bw2-input);
\draw[endflow] (bw2-houtput) -- (bw3-input);
\draw[endflow] (bw3-houtput) -- (bw4-input);

\draw[endflow] (bw1-coutput) -- (bw2-cinput);
\draw[endflow] (bw2-coutput) -- (bw3-cinput);
\draw[endflow] (bw3-coutput) -- (bw4-cinput);

\node[right= 2.5em of bw4-coutput](cbw){$\VCellState_T^\text{bw}$};

%\node[right= 6em of fw4-coutput,yshift=.5em, anchor=center,label=left:{$\VCellState^{fw}$}](cfw){
%	%\includegraphics[width=\imagewidth]{images/network/42_state}
%	\begin{tikzpicture}
%	    \colorlet{tensorcolor}{forwardcolor}
%		\tensorcube{\imagewidth}{\imagewidth}{.75}{images/network/42_state}
%	\end{tikzpicture}
%	};
%	
%\node[right= 6em of bw4-coutput, yshift=.5em, anchor=center,label=left:{$\VCellState^{bw}$}](cbw){
%	%\includegraphics[width=\imagewidth]{images/network/72_state}	
%	\begin{tikzpicture}
%	    \colorlet{tensorcolor}{backwardcolor}
%		\tensorcube{\imagewidth}{\imagewidth}{.75}{images/network/72_state}
%	\end{tikzpicture}
%	};

%\draw[endflow] (fw4-coutput) -- +(.25,0);
%\draw[endflow] (bw4-coutput) -- +(.25,0);

\draw[endflow] (bw4-coutput) -- (cbw);

}

\visible<3->{
	
	\node[bigpassbox, fill=classcolor, rectangle, minimum width=5cm,minimum height=3.5cm, anchor=center, label={[shift={(-1ex,-3.5ex)}]north:classification}] (classbox) at (12,0) {};
	
	\node[] (c) at ($(cfw)!0.5!(cbw)$) {};
	
	%\begin{scope}[perspective3d, xshift=10cm]
	%\node[transform shape,image]() at (.5,0,0){\includegraphics[width=\imagewidth]{images/network/72_state}};
	%\node[transform shape,image]() at (.25,0,0){\includegraphics[width=\imagewidth]{images/network/72_state}};
	%\node[transform shape,image]() at (0,0,0){\includegraphics[width=\imagewidth]{images/network/72_state}};
	%\end{scope}
	
	%\node(concatstates) at (6.2*\d,0){
	\node(concatstates) at ($(classbox)+(-1.15cm,0)$){
	\begin{tikzpicture}
	\concatstates{\stateimagewidth}{\stateimagewidth}{1}{\state}
	\end{tikzpicture}
	};
	
	\draw[endflow] (cfw) -- (concatstates);
	\draw[endflow] (cbw) -- (concatstates);
	
	%\path[draw] (0,0) -- (0,1) -- (1,1) -- (1,0) -- (0,0);
	
	\node[right=of concatstates](conv){
		\begin{tikzpicture}
		    \colorlet{tensorcolor}{classcolor}
		    % static
			\tensorcube{\stateimagewidth}{\stateimagewidth}{.75}{\includegraphics[width=\stateimagewidth]{\activation}} %images/network/48px/72_state
	%		\tensorcube{\stateimagewidth}{\stateimagewidth}{.75}{
	%		\animategraphics[poster=16,width=\stateimagewidth,autoplay,loop]{3}{images/network/48px/prediction_scores/prediction_scores-}{0}{16}
	%		}
		\end{tikzpicture}
	};
	
	\node[rectangle, minimum width=4mm, minimum height=4mm, fill=tumgray, draw=tumgray, opacity=.5](inconvrect) at ($ (concatstates) + (-3mm,-3mm) $){};
	%\node[below=0mm of inconvrect](){\tiny \color{white} $\kclass$};
	
	\node[rectangle, minimum width=2mm, minimum height=2mm, fill=tumgray, draw=tumgray, opacity=.5, inner sep=0](outconvrect) at (conv){};
	
	\draw[tumgray] (inconvrect.north east) -- (outconvrect.north west) -- (outconvrect.south west) -- (inconvrect.south east);
	%\draw[tumblack] (inconvrect.north west) -- (outconvrect.north west);
	%\draw[tumblack] (inconvrect.south east) -- (outconvrect.south west);
	%\draw[tumblack] (inconvrect.south west) -- (outconvrect.south west);
	
	\node[below= 3em of conv, label={south:prediction}](pred){\includegraphics[width=\classimagewidth]{\prediction}}; %$\hat{\V{y}}$
	\node[above= 3em of conv, label={north:Ground Truth}](ground){\includegraphics[width=\classimagewidth]{\groundtruth}}; % $\V{y}}$
	
	\draw[endflow] (conv) -- (pred) node[midway, right] {\small $argmax$}; 
	\draw[<->] (ground) -- (conv) node[midway, right] {\small $H(\V{y},\hat{\V{y}})$}; 
}
%};
\end{tikzpicture}

%%	
%%	%\tikzsetnextfilename{lstm}

\tikzstyle{operator} = [draw, circle, fill=tumbluemedium, draw=tumbluemedium, inner sep=0, text=white]
%\tikzstyle{function} = [draw, rectangle, fill=tumbluemedium, draw=tumbluemedium, text=white]
\tikzstyle{gate} = [] %fill=tumivory,draw,rounded corners=1pt, inner sep=2pt, minimum width=11mm, minimum height=11mm
\tikzstyle{io} = []%fill=tumwhite,draw,rounded corners=1pt, inner sep=2pt, minimum width=11mm, minimum height=11mm

\tikzstyle{dummy} = [inner sep=0]
\tikzstyle{flow} = [rounded corners]
\tikzstyle{endflow} = [-stealth,flow]

\tikzstyle{perspective3drnn}=[
x={(0.5cm,0.5cm)}, y={(1cm,0cm)}, z={(0cm,1cm)}]
\tikzstyle{wireframe} = [draw=tumgray]

\def\image{47-30}

\colorlet{boxcolor}{tumgraylight}
\tikzstyle{bigbox} = [rectangle, draw=tumivory, thick, fill=boxcolor, rounded corners, 
inner xsep=0ex, inner ysep=2ex]

\tikzstyle{annot} = [fill=white, fill opacity=.5, text opacity=1, rounded corners, text=black, xshift=-1.5mm, yshift=-1mm]

\newcommand{\concatstates}[3]{
	\def\w{1cm}%
	\def\h{1cm}%
	\def\d{#1}%
	\def\img{#2}%
	\def\fillcolor{#3}%
	%
	\begin{tikzpicture}[perspective3drnn, rounded corners=0]
		
		% bw back frame
		\begin{scope}[canvas is yz plane at x=2*\d]
			\node[transform shape,inner sep=0, minimum size=\w, anchor=south west, fill=none, wireframe](back){};
		\end{scope}
		
		% front image
		\begin{scope}[canvas is yz plane at x=0]
			\node[inner sep=0, anchor=south west](front){\includegraphics[width=\w]{\img}};
		\end{scope}
		
		% front frame
		\begin{scope}[canvas is yz plane at x=0]
			\node[transform shape,inner sep=0, minimum size=\w, anchor=south west, fill=none,wireframe](front){};
		\end{scope}
		
		\fill[\fillcolor,opacity=.5, wireframe] (front.south east) -- (back.south east) -- (back.north east) -- (front.north east);
		% fill top side
		\fill[\fillcolor,opacity=.5,opacity=0.6, wireframe] (front.north east) -- (back.north east) -- (back.north west) -- (front.north west);
		
	\end{tikzpicture}
}

\tikzset{pic shift/.store in=\shiftcoord,
	pic shift={(0,0)},
	lstmanim/.pic = {
		\begin{scope}[shift={\shiftcoord},xscale=6,yscale=2.5]
			
			\node[dummy] (bl) at (0,0){}; % bottom left
			\node[dummy] (tr) at (1,1){}; % top right
			
			\node[dummy] (br) at ($ (bl -| tr) $){}; % bottom right
			\node[dummy] (tl) at ($ (bl |- tr) $){}; % top left
			
			\node[fit=(bl) (tr),bigbox] (-C) {};
			
			% input coordinate for rounded draw lines -> slightly right of tl
			\coordinate (-input) at (0.1,1); % top left
			
			% output coordinate for rounded draw lines -> slightly left of br
			\coordinate (-coutput) at (0.98,0); % bottom right
			\coordinate (-houtput) at (0.98,1); % bottom right
			
			%			% gate distance
			\def\d{1/5}
			
			% gate heights
			\def\h{1/3}
			
			\coordinate (f)  at bl+(0.7*\d,0);
			\coordinate (i)  at bl+(1.8*\d,0);
			\coordinate (j)  at bl+(2.9*\d,0);
			\coordinate (o)  at bl+(4*\d,0);
			\coordinate (out) at bl+(4.7*\d,0);
			
			\coordinate (gates) at (0,2*\h);
			
			%\node[above=of tl](xt){$x_{t}$};
			%\node[left=of tl](htminus1){$h_{t-1}$};
			
			%\node[below=of br](ct){$c_{t}$};
			
			\node[gate](fgate) at ($ (gates -| f) $){
%				\includegraphics[width=1cm]{images/lstm/f}
				\concatstates{.2}{images/lstm/f}{tumbluelight}
				};
			\node[annot] at (fgate){$\VForgetGate_{t}$};
			
			\node[gate](igate) at ($ (gates -| i) $){
%				\includegraphics[width=1cm]{images/lstm/i}
				\concatstates{.2}{images/lstm/i}{tumbluelight}
				};
			\node[annot] at (igate){$\VInputGate_{t}$};
			
			\node[gate](jgate) at ($ (gates -| j) $){
%				\includegraphics[width=1cm]{images/lstm/j}
				\concatstates{.2}{images/lstm/j}{tumbluelight}
				};
			\node[annot] at (jgate){$\VModulationGate_{t}$};
			
			\node[gate](ogate) at ($ (gates -| o) $){
%				\includegraphics[width=1cm]{images/lstm/o}
				\concatstates{.2}{images/lstm/o}{tumbluelight}
				};
			\node[annot] at (ogate){$\VOutputGate_{t}$};
			
			%			\coordinate (htminus1) at bl+(-.5,0);
			%			\coordinate (ht) at bl+(-.5,0);
			%			
			% forget gate
			\node[operator](fmult) at ($ (bl -| fgate) $) {$ \odot $};
			\draw[flow] (-input) -| (fgate);
			\draw[endflow] (fgate) -- (fmult); 
			
			%			%j
			\node[operator](jmult) at ([shift={(0,-1.3*\h)}]jgate) {$ \odot $};
			\node[operator](cadd) at ($ (bl -| jgate) $) {$ + $};
			\draw[flow] (-input) -| (jgate);
			\draw[endflow] (jgate) -- (jmult);
			\draw[endflow] (jmult) -- (cadd); 			
			
			%			%i	
			\draw[flow] (-input) -| (igate);
			\draw[endflow] (igate) |- (jmult); 
			%
			%%			% outpu
			\node[operator](outtanh) at ($ (jmult -| out) $) {$\odot$};
			%			
			%			%o 
			\draw[flow] (tl) -| (ogate);
			\draw[endflow] (ogate) |- (outtanh);
			\draw[flow] (outtanh) |- (-houtput);
			%			
			%			% output flow
			\draw[endflow] (cadd) -| (outtanh);
			\draw[flow] (fmult) -- (cadd) -- (-coutput);
			%			
			
		\end{scope}
	}
}

\tikzset{pic shift/.store in=\shiftcoord,
	pic shift={(0,0)},
	fgate/.pic = {
		\begin{scope}[shift={\shiftcoord},xscale=1,yscale=1]
			
			\node[dummy] (tl_a) at (0,0){}; % bottom left
			\node[dummy] (br_a) at (1,1){}; % top right
			
			\node[fit=(br_a) (tr_a),gate,inner sep=0] (-C) {};
			
			\node[draw] (conv) at (0.5,0){$conv$}; % bottom left
			\node[draw] (bn) at (0.5,.5){$bn$}; % bottom left
			\node[draw] (sigmoid) at (0.5,1){$\sigma$}; % bottom left
				
		\end{scope}
	}
}

\newcommand{\grutwo}{
\begin{tikzpicture}[scale=1, node distance=2em]%,show background rectangle,background rectangle/.style={draw=red}]
\draw pic (GRU) at (0,0) {gru};
\node[io,left=of GRU-hinput](gru_htminus1){$\VHidden_{t-1}$};
\draw[rounded corners] (gru_htminus1) -| (GRU-hinputflow);
\node[io,above=of GRU-xinput](gru_xt){$\VInput_{t}$};

\draw[rounded corners] (gru_xt) |- (GRU-xinputflow);

\node[io,right=of GRU-houtput](gru_ht){$\VHidden_{t}$};
\draw[rounded corners] (GRU-houtput)--(gru_ht);
\end{tikzpicture}
}

\newcommand{\lstmanimtwo}{
	\begin{tikzpicture}[scale=1, node distance=1em]%,show background rectangle,background rectangle/.style={draw=red}]
	
	
	\draw pic (LSTM) at (0,0) {lstmanim};
	\node[io,left=of LSTMtl](htminus1){
%		\includegraphics[width=1cm]{images/lstm/o}
		\concatstates{.2}{images/lstm/h_}{tumblue}
	};
	\node[annot] at (htminus1){$\VHidden_{t-1}$};

	\draw[endflow] (htminus1) -- (LSTM-input);
	\node[io,right=of LSTMbr](ct){%
%		\includegraphics[width=1cm]{images/lstm/c}
		\concatstates{.2}{images/lstm/c}{tumblue}
	}; % $c_{t}$
	\node[annot] at (ct){$\VCellState_{t}$};

	\draw[endflow] (LSTM-coutput)--(ct);
	\node[io,left=of LSTMbl](ctminus1){%
		%\includegraphics[width=1cm]{images/lstm/c}
		\concatstates{.2}{images/lstm/c_}{tumblue}
	}; % 
	\node[annot] at (ctminus1){$\VCellState_{t-1}$};

	\draw[endflow] (ctminus1)--(LSTMfmult);
	\node[io,right=of LSTMtr](ht){
		%\includegraphics[width=1cm]{images/lstm/h}
		\concatstates{.2}{images/lstm/h}{tumblue}
	};
	\node[annot] at (ht){$\VHidden_{t}$};

	\draw[endflow] (LSTM-houtput)--(ht);
	
    \draw[endflow] (ct) -- ($ (ct)+(0,-1) $) -| (ctminus1);
	\draw[endflow] (ht) -- ($ (ht)+(0,1) $) -| (htminus1);
	
    \node[io,xshift=1ex,above=2em of LSTMtl,label=left:$\VInput_{t}$](xt){%
		%\includegraphics[width=1cm]{images/lstm/x}
		\concatstates{.1}{images/lstm/x}{tumorange}
	};%$x_{t}$

	\draw[rounded corners] (xt) |- (LSTM-input);
	

	\node[right=of ct, yshift=0em ](concatstates){
%		\begin{tikzpicture}
		\concatstates{.2}{images/lstm/cT}{tumblue}
%		\end{tikzpicture}
	};

	\node[right=of concatstates](conv){
		\concatstates{.1}{images/examples/16494/rape}{tumblue}
	};

	\coordinate (center) at ($ (concatstates) + (1mm,-3mm) $);

	\node[rectangle, minimum width=4mm, minimum height=4mm, fill=tumblue, draw=tumblue, opacity=.5, rounded corners=0, semithick](inconvrect) at (center){};
	
	\node[rectangle, minimum width=2mm, minimum height=2mm, fill=tumblue, draw=tumblue, opacity=.5, inner sep=0, semithick, rounded corners=0, xshift=2mm, yshift=-2.5mm](outconvrect) at (conv){};
	
	\draw[tumblue, semithick, rounded corners=0, fill=tumbluedark, opacity=.2] (inconvrect.north east) -- (outconvrect.north west) -- (outconvrect.south west) -- (inconvrect.south east);
	
	
	\begin{scope}[node distance=.5em]
	\node[inner sep=0, label={below:\tiny class A}, above= 2em of concatstates, xshift=0](a){\includegraphics[width=1cm]{images/examples/16494/maize}};
	\node[inner sep=0, label={below:\tiny class B}, right=of a](b){\includegraphics[width=1cm]{images/examples/16494/meadow}};
	\node[inner sep=0, label={below:\tiny class C}, right=of b](c){\includegraphics[width=1cm]{images/examples/16494/peas}};
%	\node[inner sep=0, label={below:\tiny peas}, right=of c](d){\includegraphics[width=1cm]{images/examples/16494/peas}};
%	\node[inner sep=0, label={below:\tiny rapeseed}, right=of d](e){\includegraphics[width=1cm]{images/examples/16494/rape}};
	\end{scope}
	
	\node[fit=(a)(b)(c), draw=tumgray, inner sep=1em, rounded corners, label={above:predictions}](predictions){};
	
	\draw[endflow, draw=tumgray] (conv) -- (predictions);
	
	\draw[endflow] (ct) -- node[midway, above]{$\VCellState_{T}$} (concatstates);
	
	
	\end{tikzpicture}
}
%%	\lstmanimtwo
%\end{frame}


\begin{frame}

\begin{tikzpicture}
%
%	\frametitle{Example}
%		
\def\i{20}
\def\root{images/rnn_examples/8}


\begin{groupplot}[
group style = {
	group size = 1 by 7,
	xlabels at=edge bottom,
	xticklabels at=edge bottom,
	vertical sep=0pt,
},
width=\textwidth,
%		hide axis,
enlargelimits=.1,
height=2.5cm,
%		ymin=0, ymax=1.4,
no marks,
]
\nextgroupplot[draw opacity=.8, smooth=0.01, thin]
\addplot[b11color, mark=*,mark size=.5pt] table [x=t, y=B11, col sep=comma, forget plot] {\root/x.csv};
\addplot[b12color, mark=*,mark size=.5pt] table [x=t, y=B12, col sep=comma] {\root/x.csv};

\addplot[b5color, mark=*,mark size=.5pt] table [x=t, y=B5, col sep=comma, forget plot] {\root/x.csv};
\addplot[b6color, mark=*,mark size=.5pt] table [x=t, y=B6, col sep=comma, forget plot] {\root/x.csv};
\addplot[b7color, mark=*,mark size=.5pt] table [x=t, y=B7, col sep=comma, forget plot] {\root/x.csv};
\addplot[b8color, mark=*,mark size=.5pt] table [x=t, y=B8, col sep=comma, forget plot] {\root/x.csv};
\addplot[b8Acolor, mark=*,mark size=.5pt] table [x=t, y=B8A, col sep=comma] {\root/x.csv};

\addplot[b2color, mark=*,mark size=.5pt] table [x=t, y=B2, col sep=comma, forget plot] {\root/x.csv};
\addplot[b3color, mark=*,mark size=.5pt] table [x=t, y=B3, col sep=comma, forget plot] {\root/x.csv};
\addplot[b4color, mark=*,mark size=.5pt] table [x=t, y=B4, col sep=comma] {\root/x.csv};

\nextgroupplot[cycle list name=featurecolorlist, draw opacity=.8, smooth=0.01, thin, ylabel=$\V{i}$]
\foreach \i in {0,1,...,31}{
	\addplot table [x=t, y=\i, col sep=comma] {\root/i.csv};
}
\nextgroupplot[cycle list name=featurecolorlist, draw opacity=.8, smooth=0.01, thin, ylabel=$\V{f}$]
\foreach \i in {0,1,...,31}{
	\addplot table [x=t, y=\i, col sep=comma] {\root/f.csv};
}
\nextgroupplot[cycle list name=featurecolorlist, draw opacity=.8, smooth=0.01, thin, ylabel=$\V{g}$]
\foreach \i in {0,1,...,31}{
	\addplot table [x=t, y=\i, col sep=comma] {\root/g.csv};
}
\nextgroupplot[cycle list name=featurecolorlist, draw opacity=.8, smooth=0.01, thin, ylabel=$\V{o}$]
\foreach \i in {0,1,...,31}{
	\addplot table [x=t, y=\i, col sep=comma] {\root/o.csv};
}
\nextgroupplot[cycle list name=featurecolorlist, draw opacity=.8, smooth=0.01, thin, ylabel=$\V{h}$]
\foreach \i in {0,1,...,31}{
	\addplot table [x=t, y=\i, col sep=comma] {\root/h.csv};
}
\nextgroupplot[cycle list name=featurecolorlist, draw opacity=.8, smooth=0.01, thin, ylabel=$\V{c}$]
\foreach \i in {0,1,...,31}{
	\addplot table [x=t, y=\i, col sep=comma] {\root/c.csv};
}

\end{groupplot}

\end{tikzpicture}
\end{frame}

\begin{frame}
\begin{tikzpicture}
%
%	\frametitle{Example}
%		
\def\i{20}
\def\root{images/rnn_examples/5}


\begin{groupplot}[
group style = {
group size = 1 by 7,
xlabels at=edge bottom,
xticklabels at=edge bottom,
vertical sep=0pt,
},
width=\textwidth,
%		hide axis,
enlargelimits=.1,
height=2.5cm,
%		ymin=0, ymax=1.4,
no marks,
]
\nextgroupplot[draw opacity=.8, smooth=0.01, thin]
\addplot[b11color, mark=*,mark size=.5pt] table [x=t, y=B11, col sep=comma, forget plot] {\root/x.csv};
\addplot[b12color, mark=*,mark size=.5pt] table [x=t, y=B12, col sep=comma] {\root/x.csv};

\addplot[b5color, mark=*,mark size=.5pt] table [x=t, y=B5, col sep=comma, forget plot] {\root/x.csv};
\addplot[b6color, mark=*,mark size=.5pt] table [x=t, y=B6, col sep=comma, forget plot] {\root/x.csv};
\addplot[b7color, mark=*,mark size=.5pt] table [x=t, y=B7, col sep=comma, forget plot] {\root/x.csv};
\addplot[b8color, mark=*,mark size=.5pt] table [x=t, y=B8, col sep=comma, forget plot] {\root/x.csv};
\addplot[b8Acolor, mark=*,mark size=.5pt] table [x=t, y=B8A, col sep=comma] {\root/x.csv};

\addplot[b2color, mark=*,mark size=.5pt] table [x=t, y=B2, col sep=comma, forget plot] {\root/x.csv};
\addplot[b3color, mark=*,mark size=.5pt] table [x=t, y=B3, col sep=comma, forget plot] {\root/x.csv};
\addplot[b4color, mark=*,mark size=.5pt] table [x=t, y=B4, col sep=comma] {\root/x.csv};

\nextgroupplot[cycle list name=featurecolorlist, draw opacity=.1, smooth=0.01, thin, ylabel=$\V{i}$]

\addplot table [x=t,  y=0, col sep=comma] {\root/i.csv};
\addplot table [x=t,  y=1, col sep=comma] {\root/i.csv};
\addplot table [x=t,  y=2, col sep=comma] {\root/i.csv};
\addplot table [x=t,  y=3, col sep=comma] {\root/i.csv};
\addplot table [x=t,  y=4, col sep=comma] {\root/i.csv};
\addplot table [x=t,  y=5, col sep=comma] {\root/i.csv};
\addplot table [x=t,  y=6, col sep=comma] {\root/i.csv};
\addplot table [x=t,  y=7, col sep=comma] {\root/i.csv};
\addplot table [x=t,  y=8, col sep=comma] {\root/i.csv};
\addplot table [x=t,  y=9, col sep=comma] {\root/i.csv};
\addplot table [x=t, y=10, col sep=comma] {\root/i.csv};
\addplot table [x=t, y=11, col sep=comma] {\root/i.csv};
\addplot table [x=t, y=12, col sep=comma] {\root/i.csv};
\addplot table [x=t, y=13, col sep=comma] {\root/i.csv};
\addplot table [x=t, y=14, col sep=comma] {\root/i.csv};
\addplot table [x=t, y=15, col sep=comma] {\root/i.csv};
\addplot table [x=t, y=16, col sep=comma] {\root/i.csv};
\addplot table [x=t, y=17, col sep=comma] {\root/i.csv};
\addplot table [x=t, y=18, col sep=comma] {\root/i.csv};
\addplot table [x=t, y=19, col sep=comma] {\root/i.csv};
\addplot table [x=t, y=20, col sep=comma] {\root/i.csv};
\addplot table [x=t, y=21, col sep=comma] {\root/i.csv};
\addplot table [x=t, y=22, col sep=comma] {\root/i.csv};
\addplot table [x=t, y=23, col sep=comma] {\root/i.csv};
\addplot table [x=t, y=24, col sep=comma] {\root/i.csv};
\addplot table [x=t, y=25, col sep=comma] {\root/i.csv};
\addplot table [x=t, y=26, col sep=comma] {\root/i.csv};
\addplot table [x=t, y=27, col sep=comma] {\root/i.csv};
\addplot table [x=t, y=28, col sep=comma] {\root/i.csv};
\addplot table [x=t, y=29, col sep=comma] {\root/i.csv};
\addplot[draw opacity=1] table [x=t, y=30, col sep=comma] {\root/i.csv};
\addplot table [x=t, y=31, col sep=comma] {\root/i.csv};

\nextgroupplot[cycle list name=featurecolorlist, draw opacity=.2, smooth=0.01, thin, ylabel=$\V{f}$]

\addplot table [x=t,  y=0, col sep=comma] {\root/f.csv};
\addplot table [x=t,  y=1, col sep=comma] {\root/f.csv};
\addplot table [x=t,  y=2, col sep=comma] {\root/f.csv};
\addplot table [x=t,  y=3, col sep=comma] {\root/f.csv};
\addplot table [x=t,  y=4, col sep=comma] {\root/f.csv};
\addplot table [x=t,  y=5, col sep=comma] {\root/f.csv};
\addplot table [x=t,  y=6, col sep=comma] {\root/f.csv};
\addplot table [x=t,  y=7, col sep=comma] {\root/f.csv};
\addplot table [x=t,  y=8, col sep=comma] {\root/f.csv};
\addplot table [x=t,  y=9, col sep=comma] {\root/f.csv};
\addplot table [x=t, y=10, col sep=comma] {\root/f.csv};
\addplot table [x=t, y=11, col sep=comma] {\root/f.csv};
\addplot table [x=t, y=12, col sep=comma] {\root/f.csv};
\addplot table [x=t, y=13, col sep=comma] {\root/f.csv};
\addplot table [x=t, y=14, col sep=comma] {\root/f.csv};
\addplot table [x=t, y=15, col sep=comma] {\root/f.csv};
\addplot table [x=t, y=16, col sep=comma] {\root/f.csv};
\addplot table [x=t, y=17, col sep=comma] {\root/f.csv};
\addplot table [x=t, y=18, col sep=comma] {\root/f.csv};
\addplot table [x=t, y=19, col sep=comma] {\root/f.csv};
\addplot table [x=t, y=20, col sep=comma] {\root/f.csv};
\addplot table [x=t, y=21, col sep=comma] {\root/f.csv};
\addplot table [x=t, y=22, col sep=comma] {\root/f.csv};
\addplot table [x=t, y=23, col sep=comma] {\root/f.csv};
\addplot table [x=t, y=24, col sep=comma] {\root/f.csv};
\addplot table [x=t, y=25, col sep=comma] {\root/f.csv};
\addplot table [x=t, y=26, col sep=comma] {\root/f.csv};
\addplot table [x=t, y=27, col sep=comma] {\root/f.csv};
\addplot table [x=t, y=28, col sep=comma] {\root/f.csv};
\addplot table [x=t, y=29, col sep=comma] {\root/f.csv};
\addplot[draw opacity=1] table [x=t, y=30, col sep=comma] {\root/f.csv};
\addplot table [x=t, y=31, col sep=comma] {\root/f.csv};

\nextgroupplot[cycle list name=featurecolorlist, draw opacity=.2, smooth=0.01, thin, ylabel=$\V{g}$]

\addplot table [x=t,  y=0, col sep=comma] {\root/g.csv};
\addplot table [x=t,  y=1, col sep=comma] {\root/g.csv};
\addplot table [x=t,  y=2, col sep=comma] {\root/g.csv};
\addplot table [x=t,  y=3, col sep=comma] {\root/g.csv};
\addplot table [x=t,  y=4, col sep=comma] {\root/g.csv};
\addplot table [x=t,  y=5, col sep=comma] {\root/g.csv};
\addplot table [x=t,  y=6, col sep=comma] {\root/g.csv};
\addplot table [x=t,  y=7, col sep=comma] {\root/g.csv};
\addplot table [x=t,  y=8, col sep=comma] {\root/g.csv};
\addplot table [x=t,  y=9, col sep=comma] {\root/g.csv};
\addplot table [x=t, y=10, col sep=comma] {\root/g.csv};
\addplot table [x=t, y=11, col sep=comma] {\root/g.csv};
\addplot table [x=t, y=12, col sep=comma] {\root/g.csv};
\addplot table [x=t, y=13, col sep=comma] {\root/g.csv};
\addplot table [x=t, y=14, col sep=comma] {\root/g.csv};
\addplot table [x=t, y=15, col sep=comma] {\root/g.csv};
\addplot table [x=t, y=16, col sep=comma] {\root/g.csv};
\addplot table [x=t, y=17, col sep=comma] {\root/g.csv};
\addplot table [x=t, y=18, col sep=comma] {\root/g.csv};
\addplot table [x=t, y=19, col sep=comma] {\root/g.csv};
\addplot table [x=t, y=20, col sep=comma] {\root/g.csv};
\addplot table [x=t, y=21, col sep=comma] {\root/g.csv};
\addplot table [x=t, y=22, col sep=comma] {\root/g.csv};
\addplot table [x=t, y=23, col sep=comma] {\root/g.csv};
\addplot table [x=t, y=24, col sep=comma] {\root/g.csv};
\addplot table [x=t, y=25, col sep=comma] {\root/g.csv};
\addplot table [x=t, y=26, col sep=comma] {\root/g.csv};
\addplot table [x=t, y=27, col sep=comma] {\root/g.csv};
\addplot table [x=t, y=28, col sep=comma] {\root/g.csv};
\addplot table [x=t, y=29, col sep=comma] {\root/g.csv};
\addplot[draw opacity=1] table [x=t, y=30, col sep=comma] {\root/g.csv};
\addplot table [x=t, y=31, col sep=comma] {\root/g.csv};

\nextgroupplot[cycle list name=featurecolorlist, draw opacity=.2, smooth=0.01, thin, ylabel=$\V{o}$]

\addplot table [x=t,  y=0, col sep=comma] {\root/o.csv};
\addplot table [x=t,  y=1, col sep=comma] {\root/o.csv};
\addplot table [x=t,  y=2, col sep=comma] {\root/o.csv};
\addplot table [x=t,  y=3, col sep=comma] {\root/o.csv};
\addplot table [x=t,  y=4, col sep=comma] {\root/o.csv};
\addplot table [x=t,  y=5, col sep=comma] {\root/o.csv};
\addplot table [x=t,  y=6, col sep=comma] {\root/o.csv};
\addplot table [x=t,  y=7, col sep=comma] {\root/o.csv};
\addplot table [x=t,  y=8, col sep=comma] {\root/o.csv};
\addplot table [x=t,  y=9, col sep=comma] {\root/o.csv};
\addplot table [x=t, y=10, col sep=comma] {\root/o.csv};
\addplot table [x=t, y=11, col sep=comma] {\root/o.csv};
\addplot table [x=t, y=12, col sep=comma] {\root/o.csv};
\addplot table [x=t, y=13, col sep=comma] {\root/o.csv};
\addplot table [x=t, y=14, col sep=comma] {\root/o.csv};
\addplot table [x=t, y=15, col sep=comma] {\root/o.csv};
\addplot table [x=t, y=16, col sep=comma] {\root/o.csv};
\addplot table [x=t, y=17, col sep=comma] {\root/o.csv};
\addplot table [x=t, y=18, col sep=comma] {\root/o.csv};
\addplot table [x=t, y=19, col sep=comma] {\root/o.csv};
\addplot table [x=t, y=20, col sep=comma] {\root/o.csv};
\addplot table [x=t, y=21, col sep=comma] {\root/o.csv};
\addplot table [x=t, y=22, col sep=comma] {\root/o.csv};
\addplot table [x=t, y=23, col sep=comma] {\root/o.csv};
\addplot table [x=t, y=24, col sep=comma] {\root/o.csv};
\addplot table [x=t, y=25, col sep=comma] {\root/o.csv};
\addplot table [x=t, y=26, col sep=comma] {\root/o.csv};
\addplot table [x=t, y=27, col sep=comma] {\root/o.csv};
\addplot table [x=t, y=28, col sep=comma] {\root/o.csv};
\addplot table [x=t, y=29, col sep=comma] {\root/o.csv};
\addplot[draw opacity=1] table [x=t, y=30, col sep=comma] {\root/o.csv};
\addplot table [x=t, y=31, col sep=comma] {\root/o.csv};

\nextgroupplot[cycle list name=featurecolorlist, draw opacity=.2, smooth=0.01, thin, ylabel=$\V{h}$]

\addplot table [x=t,  y=0, col sep=comma] {\root/h.csv};
\addplot table [x=t,  y=1, col sep=comma] {\root/h.csv};
\addplot table [x=t,  y=2, col sep=comma] {\root/h.csv};
\addplot table [x=t,  y=3, col sep=comma] {\root/h.csv};
\addplot table [x=t,  y=4, col sep=comma] {\root/h.csv};
\addplot table [x=t,  y=5, col sep=comma] {\root/h.csv};
\addplot table [x=t,  y=6, col sep=comma] {\root/h.csv};
\addplot table [x=t,  y=7, col sep=comma] {\root/h.csv};
\addplot table [x=t,  y=8, col sep=comma] {\root/h.csv};
\addplot table [x=t,  y=9, col sep=comma] {\root/h.csv};
\addplot table [x=t, y=10, col sep=comma] {\root/h.csv};
\addplot table [x=t, y=11, col sep=comma] {\root/h.csv};
\addplot table [x=t, y=12, col sep=comma] {\root/h.csv};
\addplot table [x=t, y=13, col sep=comma] {\root/h.csv};
\addplot table [x=t, y=14, col sep=comma] {\root/h.csv};
\addplot table [x=t, y=15, col sep=comma] {\root/h.csv};
\addplot table [x=t, y=16, col sep=comma] {\root/h.csv};
\addplot table [x=t, y=17, col sep=comma] {\root/h.csv};
\addplot table [x=t, y=18, col sep=comma] {\root/h.csv};
\addplot table [x=t, y=19, col sep=comma] {\root/h.csv};
\addplot table [x=t, y=20, col sep=comma] {\root/h.csv};
\addplot table [x=t, y=21, col sep=comma] {\root/h.csv};
\addplot table [x=t, y=22, col sep=comma] {\root/h.csv};
\addplot table [x=t, y=23, col sep=comma] {\root/h.csv};
\addplot table [x=t, y=24, col sep=comma] {\root/h.csv};
\addplot table [x=t, y=25, col sep=comma] {\root/h.csv};
\addplot table [x=t, y=26, col sep=comma] {\root/h.csv};
\addplot table [x=t, y=27, col sep=comma] {\root/h.csv};
\addplot table [x=t, y=28, col sep=comma] {\root/h.csv};
\addplot table [x=t, y=29, col sep=comma] {\root/h.csv};
\addplot[draw opacity=1] table [x=t, y=30, col sep=comma] {\root/h.csv};
\addplot table [x=t, y=31, col sep=comma] {\root/h.csv};

\nextgroupplot[cycle list name=featurecolorlist, draw opacity=.2, smooth=0.01, thin, ylabel=$\V{c}$]
%
\addplot table [x=t,  y=0, col sep=comma] {\root/c.csv};
\addplot table [x=t,  y=1, col sep=comma] {\root/c.csv};
\addplot table [x=t,  y=2, col sep=comma] {\root/c.csv};
\addplot table [x=t,  y=3, col sep=comma] {\root/c.csv};
\addplot table [x=t,  y=4, col sep=comma] {\root/c.csv};
\addplot table [x=t,  y=5, col sep=comma] {\root/c.csv};
\addplot table [x=t,  y=6, col sep=comma] {\root/c.csv};
\addplot table [x=t,  y=7, col sep=comma] {\root/c.csv};
\addplot table [x=t,  y=8, col sep=comma] {\root/c.csv};
\addplot table [x=t,  y=9, col sep=comma] {\root/c.csv};
\addplot table [x=t, y=10, col sep=comma] {\root/c.csv};
\addplot table [x=t, y=11, col sep=comma] {\root/c.csv};
\addplot table [x=t, y=12, col sep=comma] {\root/c.csv};
\addplot table [x=t, y=13, col sep=comma] {\root/c.csv};
\addplot table [x=t, y=14, col sep=comma] {\root/c.csv};
\addplot table [x=t, y=15, col sep=comma] {\root/c.csv};
\addplot table [x=t, y=16, col sep=comma] {\root/c.csv};
\addplot table [x=t, y=17, col sep=comma] {\root/c.csv};
\addplot table [x=t, y=18, col sep=comma] {\root/c.csv};
\addplot table [x=t, y=19, col sep=comma] {\root/c.csv};
\addplot table [x=t, y=20, col sep=comma] {\root/c.csv};
\addplot table [x=t, y=21, col sep=comma] {\root/c.csv};
\addplot table [x=t, y=22, col sep=comma] {\root/c.csv};
\addplot table [x=t, y=23, col sep=comma] {\root/c.csv};
\addplot table [x=t, y=24, col sep=comma] {\root/c.csv};
\addplot table [x=t, y=25, col sep=comma] {\root/c.csv};
\addplot table [x=t, y=26, col sep=comma] {\root/c.csv};
\addplot table [x=t, y=27, col sep=comma] {\root/c.csv};
\addplot table [x=t, y=28, col sep=comma] {\root/c.csv};
\addplot table [x=t, y=29, col sep=comma] {\root/c.csv};
\addplot[draw opacity=1] table [x=t, y=30, col sep=comma] {\root/c.csv};
\addplot table [x=t, y=31, col sep=comma] {\root/c.csv};

\end{groupplot}

\end{tikzpicture}

\end{frame}


\begin{frame}
\begin{tikzpicture}
%
%	\frametitle{Example}
%		
\def\i{20}
\def\root{images/rnn_examples/5}


\begin{groupplot}[
group style = {
group size = 1 by 7,
xlabels at=edge bottom,
xticklabels at=edge bottom,
vertical sep=0pt,
},
width=\textwidth,
%		hide axis,
enlargelimits=.1,
height=2.5cm,
%		ymin=0, ymax=1.4,
no marks,
]
\nextgroupplot[draw opacity=.8, smooth=0.01, thin]
\addplot[b11color, mark=*,mark size=.5pt] table [x=t, y=B11, col sep=comma, forget plot] {\root/x.csv};
\addplot[b12color, mark=*,mark size=.5pt] table [x=t, y=B12, col sep=comma] {\root/x.csv};

\addplot[b5color, mark=*,mark size=.5pt] table [x=t, y=B5, col sep=comma, forget plot] {\root/x.csv};
\addplot[b6color, mark=*,mark size=.5pt] table [x=t, y=B6, col sep=comma, forget plot] {\root/x.csv};
\addplot[b7color, mark=*,mark size=.5pt] table [x=t, y=B7, col sep=comma, forget plot] {\root/x.csv};
\addplot[b8color, mark=*,mark size=.5pt] table [x=t, y=B8, col sep=comma, forget plot] {\root/x.csv};
\addplot[b8Acolor, mark=*,mark size=.5pt] table [x=t, y=B8A, col sep=comma] {\root/x.csv};

\addplot[b2color, mark=*,mark size=.5pt] table [x=t, y=B2, col sep=comma, forget plot] {\root/x.csv};
\addplot[b3color, mark=*,mark size=.5pt] table [x=t, y=B3, col sep=comma, forget plot] {\root/x.csv};
\addplot[b4color, mark=*,mark size=.5pt] table [x=t, y=B4, col sep=comma] {\root/x.csv};

\nextgroupplot[cycle list name=featurecolorlist, draw opacity=.1, smooth=0.01, thin, ylabel=$\V{i}$]

\addplot table [x=t,  y=0, col sep=comma] {\root/i.csv};
\addplot table [x=t,  y=1, col sep=comma] {\root/i.csv};
\addplot table [x=t,  y=2, col sep=comma] {\root/i.csv};
\addplot table [x=t,  y=3, col sep=comma] {\root/i.csv};
\addplot table [x=t,  y=4, col sep=comma] {\root/i.csv};
\addplot table [x=t,  y=5, col sep=comma] {\root/i.csv};
\addplot table [x=t,  y=6, col sep=comma] {\root/i.csv};
\addplot table [x=t,  y=7, col sep=comma] {\root/i.csv};
\addplot table [x=t,  y=8, col sep=comma] {\root/i.csv};
\addplot table [x=t,  y=9, col sep=comma] {\root/i.csv};
\addplot table [x=t, y=10, col sep=comma] {\root/i.csv};
\addplot table [x=t, y=11, col sep=comma] {\root/i.csv};
\addplot table [x=t, y=12, col sep=comma] {\root/i.csv};
\addplot table [x=t, y=13, col sep=comma] {\root/i.csv};
\addplot table [x=t, y=14, col sep=comma] {\root/i.csv};
\addplot table [x=t, y=15, col sep=comma] {\root/i.csv};
\addplot table [x=t, y=16, col sep=comma] {\root/i.csv};
\addplot table [x=t, y=17, col sep=comma] {\root/i.csv};
\addplot table [x=t, y=18, col sep=comma] {\root/i.csv};
\addplot table [x=t, y=19, col sep=comma] {\root/i.csv};
\addplot table [x=t, y=20, col sep=comma] {\root/i.csv};
\addplot table [x=t, y=21, col sep=comma] {\root/i.csv};
\addplot table [x=t, y=22, col sep=comma] {\root/i.csv};
\addplot table [x=t, y=23, col sep=comma] {\root/i.csv};
\addplot table [x=t, y=24, col sep=comma] {\root/i.csv};
\addplot table [x=t, y=25, col sep=comma] {\root/i.csv};
\addplot table [x=t, y=26, col sep=comma] {\root/i.csv};
\addplot table [x=t, y=27, col sep=comma] {\root/i.csv};
\addplot table [x=t, y=28, col sep=comma] {\root/i.csv};
\addplot[draw opacity=1] table [x=t, y=29, col sep=comma] {\root/i.csv};
\addplot table [x=t, y=30, col sep=comma] {\root/i.csv};
\addplot[draw opacity=1] table [x=t, y=31, col sep=comma] {\root/i.csv};

\nextgroupplot[cycle list name=featurecolorlist, draw opacity=.2, smooth=0.01, thin, ylabel=$\V{f}$]

\addplot table [x=t,  y=0, col sep=comma] {\root/f.csv};
\addplot table [x=t,  y=1, col sep=comma] {\root/f.csv};
\addplot table [x=t,  y=2, col sep=comma] {\root/f.csv};
\addplot table [x=t,  y=3, col sep=comma] {\root/f.csv};
\addplot table [x=t,  y=4, col sep=comma] {\root/f.csv};
\addplot table [x=t,  y=5, col sep=comma] {\root/f.csv};
\addplot table [x=t,  y=6, col sep=comma] {\root/f.csv};
\addplot table [x=t,  y=7, col sep=comma] {\root/f.csv};
\addplot table [x=t,  y=8, col sep=comma] {\root/f.csv};
\addplot table [x=t,  y=9, col sep=comma] {\root/f.csv};
\addplot table [x=t, y=10, col sep=comma] {\root/f.csv};
\addplot table [x=t, y=11, col sep=comma] {\root/f.csv};
\addplot table [x=t, y=12, col sep=comma] {\root/f.csv};
\addplot table [x=t, y=13, col sep=comma] {\root/f.csv};
\addplot table [x=t, y=14, col sep=comma] {\root/f.csv};
\addplot table [x=t, y=15, col sep=comma] {\root/f.csv};
\addplot table [x=t, y=16, col sep=comma] {\root/f.csv};
\addplot table [x=t, y=17, col sep=comma] {\root/f.csv};
\addplot table [x=t, y=18, col sep=comma] {\root/f.csv};
\addplot table [x=t, y=19, col sep=comma] {\root/f.csv};
\addplot table [x=t, y=20, col sep=comma] {\root/f.csv};
\addplot table [x=t, y=21, col sep=comma] {\root/f.csv};
\addplot table [x=t, y=22, col sep=comma] {\root/f.csv};
\addplot table [x=t, y=23, col sep=comma] {\root/f.csv};
\addplot table [x=t, y=24, col sep=comma] {\root/f.csv};
\addplot table [x=t, y=25, col sep=comma] {\root/f.csv};
\addplot table [x=t, y=26, col sep=comma] {\root/f.csv};
\addplot table [x=t, y=27, col sep=comma] {\root/f.csv};
\addplot table [x=t, y=28, col sep=comma] {\root/f.csv};
\addplot[draw opacity=1] table [x=t, y=29, col sep=comma] {\root/f.csv};
\addplot table [x=t, y=30, col sep=comma] {\root/f.csv};
\addplot[draw opacity=1] table [x=t, y=31, col sep=comma] {\root/f.csv};

\nextgroupplot[cycle list name=featurecolorlist, draw opacity=.2, smooth=0.01, thin, ylabel=$\V{g}$]

\addplot table [x=t,  y=0, col sep=comma] {\root/g.csv};
\addplot table [x=t,  y=1, col sep=comma] {\root/g.csv};
\addplot table [x=t,  y=2, col sep=comma] {\root/g.csv};
\addplot table [x=t,  y=3, col sep=comma] {\root/g.csv};
\addplot table [x=t,  y=4, col sep=comma] {\root/g.csv};
\addplot table [x=t,  y=5, col sep=comma] {\root/g.csv};
\addplot table [x=t,  y=6, col sep=comma] {\root/g.csv};
\addplot table [x=t,  y=7, col sep=comma] {\root/g.csv};
\addplot table [x=t,  y=8, col sep=comma] {\root/g.csv};
\addplot table [x=t,  y=9, col sep=comma] {\root/g.csv};
\addplot table [x=t, y=10, col sep=comma] {\root/g.csv};
\addplot table [x=t, y=11, col sep=comma] {\root/g.csv};
\addplot table [x=t, y=12, col sep=comma] {\root/g.csv};
\addplot table [x=t, y=13, col sep=comma] {\root/g.csv};
\addplot table [x=t, y=14, col sep=comma] {\root/g.csv};
\addplot table [x=t, y=15, col sep=comma] {\root/g.csv};
\addplot table [x=t, y=16, col sep=comma] {\root/g.csv};
\addplot table [x=t, y=17, col sep=comma] {\root/g.csv};
\addplot table [x=t, y=18, col sep=comma] {\root/g.csv};
\addplot table [x=t, y=19, col sep=comma] {\root/g.csv};
\addplot table [x=t, y=20, col sep=comma] {\root/g.csv};
\addplot table [x=t, y=21, col sep=comma] {\root/g.csv};
\addplot table [x=t, y=22, col sep=comma] {\root/g.csv};
\addplot table [x=t, y=23, col sep=comma] {\root/g.csv};
\addplot table [x=t, y=24, col sep=comma] {\root/g.csv};
\addplot table [x=t, y=25, col sep=comma] {\root/g.csv};
\addplot table [x=t, y=26, col sep=comma] {\root/g.csv};
\addplot table [x=t, y=27, col sep=comma] {\root/g.csv};
\addplot table [x=t, y=28, col sep=comma] {\root/g.csv};
\addplot[draw opacity=1] table [x=t, y=29, col sep=comma] {\root/g.csv};
\addplot table [x=t, y=30, col sep=comma] {\root/g.csv};
\addplot[draw opacity=1] table [x=t, y=31, col sep=comma] {\root/g.csv};

\nextgroupplot[cycle list name=featurecolorlist, draw opacity=.2, smooth=0.01, thin, ylabel=$\V{o}$]

\addplot table [x=t,  y=0, col sep=comma] {\root/o.csv};
\addplot table [x=t,  y=1, col sep=comma] {\root/o.csv};
\addplot table [x=t,  y=2, col sep=comma] {\root/o.csv};
\addplot table [x=t,  y=3, col sep=comma] {\root/o.csv};
\addplot table [x=t,  y=4, col sep=comma] {\root/o.csv};
\addplot table [x=t,  y=5, col sep=comma] {\root/o.csv};
\addplot table [x=t,  y=6, col sep=comma] {\root/o.csv};
\addplot table [x=t,  y=7, col sep=comma] {\root/o.csv};
\addplot table [x=t,  y=8, col sep=comma] {\root/o.csv};
\addplot table [x=t,  y=9, col sep=comma] {\root/o.csv};
\addplot table [x=t, y=10, col sep=comma] {\root/o.csv};
\addplot table [x=t, y=11, col sep=comma] {\root/o.csv};
\addplot table [x=t, y=12, col sep=comma] {\root/o.csv};
\addplot table [x=t, y=13, col sep=comma] {\root/o.csv};
\addplot table [x=t, y=14, col sep=comma] {\root/o.csv};
\addplot table [x=t, y=15, col sep=comma] {\root/o.csv};
\addplot table [x=t, y=16, col sep=comma] {\root/o.csv};
\addplot table [x=t, y=17, col sep=comma] {\root/o.csv};
\addplot table [x=t, y=18, col sep=comma] {\root/o.csv};
\addplot table [x=t, y=19, col sep=comma] {\root/o.csv};
\addplot table [x=t, y=20, col sep=comma] {\root/o.csv};
\addplot table [x=t, y=21, col sep=comma] {\root/o.csv};
\addplot table [x=t, y=22, col sep=comma] {\root/o.csv};
\addplot table [x=t, y=23, col sep=comma] {\root/o.csv};
\addplot table [x=t, y=24, col sep=comma] {\root/o.csv};
\addplot table [x=t, y=25, col sep=comma] {\root/o.csv};
\addplot table [x=t, y=26, col sep=comma] {\root/o.csv};
\addplot table [x=t, y=27, col sep=comma] {\root/o.csv};
\addplot table [x=t, y=28, col sep=comma] {\root/o.csv};
\addplot[draw opacity=1] table [x=t, y=29, col sep=comma] {\root/o.csv};
\addplot table [x=t, y=30, col sep=comma] {\root/o.csv};
\addplot[draw opacity=1] table [x=t, y=31, col sep=comma] {\root/o.csv};

\nextgroupplot[cycle list name=featurecolorlist, draw opacity=.2, smooth=0.01, thin, ylabel=$\V{h}$]

\addplot table [x=t,  y=0, col sep=comma] {\root/h.csv};
\addplot table [x=t,  y=1, col sep=comma] {\root/h.csv};
\addplot table [x=t,  y=2, col sep=comma] {\root/h.csv};
\addplot table [x=t,  y=3, col sep=comma] {\root/h.csv};
\addplot table [x=t,  y=4, col sep=comma] {\root/h.csv};
\addplot table [x=t,  y=5, col sep=comma] {\root/h.csv};
\addplot table [x=t,  y=6, col sep=comma] {\root/h.csv};
\addplot table [x=t,  y=7, col sep=comma] {\root/h.csv};
\addplot table [x=t,  y=8, col sep=comma] {\root/h.csv};
\addplot table [x=t,  y=9, col sep=comma] {\root/h.csv};
\addplot table [x=t, y=10, col sep=comma] {\root/h.csv};
\addplot table [x=t, y=11, col sep=comma] {\root/h.csv};
\addplot table [x=t, y=12, col sep=comma] {\root/h.csv};
\addplot table [x=t, y=13, col sep=comma] {\root/h.csv};
\addplot table [x=t, y=14, col sep=comma] {\root/h.csv};
\addplot table [x=t, y=15, col sep=comma] {\root/h.csv};
\addplot table [x=t, y=16, col sep=comma] {\root/h.csv};
\addplot table [x=t, y=17, col sep=comma] {\root/h.csv};
\addplot table [x=t, y=18, col sep=comma] {\root/h.csv};
\addplot table [x=t, y=19, col sep=comma] {\root/h.csv};
\addplot table [x=t, y=20, col sep=comma] {\root/h.csv};
\addplot table [x=t, y=21, col sep=comma] {\root/h.csv};
\addplot table [x=t, y=22, col sep=comma] {\root/h.csv};
\addplot table [x=t, y=23, col sep=comma] {\root/h.csv};
\addplot table [x=t, y=24, col sep=comma] {\root/h.csv};
\addplot table [x=t, y=25, col sep=comma] {\root/h.csv};
\addplot table [x=t, y=26, col sep=comma] {\root/h.csv};
\addplot table [x=t, y=27, col sep=comma] {\root/h.csv};
\addplot table [x=t, y=28, col sep=comma] {\root/h.csv};
\addplot[draw opacity=1] table [x=t, y=29, col sep=comma] {\root/h.csv};
\addplot table [x=t, y=30, col sep=comma] {\root/h.csv};
\addplot[draw opacity=1] table [x=t, y=31, col sep=comma] {\root/h.csv};

\nextgroupplot[cycle list name=featurecolorlist, draw opacity=.2, smooth=0.01, thin, ylabel=$\V{c}$]
%
\addplot table [x=t,  y=0, col sep=comma] {\root/c.csv};
\addplot table [x=t,  y=1, col sep=comma] {\root/c.csv};
\addplot table [x=t,  y=2, col sep=comma] {\root/c.csv};
\addplot table [x=t,  y=3, col sep=comma] {\root/c.csv};
\addplot table [x=t,  y=4, col sep=comma] {\root/c.csv};
\addplot table [x=t,  y=5, col sep=comma] {\root/c.csv};
\addplot table [x=t,  y=6, col sep=comma] {\root/c.csv};
\addplot table [x=t,  y=7, col sep=comma] {\root/c.csv};
\addplot table [x=t,  y=8, col sep=comma] {\root/c.csv};
\addplot table [x=t,  y=9, col sep=comma] {\root/c.csv};
\addplot table [x=t, y=10, col sep=comma] {\root/c.csv};
\addplot table [x=t, y=11, col sep=comma] {\root/c.csv};
\addplot table [x=t, y=12, col sep=comma] {\root/c.csv};
\addplot table [x=t, y=13, col sep=comma] {\root/c.csv};
\addplot table [x=t, y=14, col sep=comma] {\root/c.csv};
\addplot table [x=t, y=15, col sep=comma] {\root/c.csv};
\addplot table [x=t, y=16, col sep=comma] {\root/c.csv};
\addplot table [x=t, y=17, col sep=comma] {\root/c.csv};
\addplot table [x=t, y=18, col sep=comma] {\root/c.csv};
\addplot table [x=t, y=19, col sep=comma] {\root/c.csv};
\addplot table [x=t, y=20, col sep=comma] {\root/c.csv};
\addplot table [x=t, y=21, col sep=comma] {\root/c.csv};
\addplot table [x=t, y=22, col sep=comma] {\root/c.csv};
\addplot table [x=t, y=23, col sep=comma] {\root/c.csv};
\addplot table [x=t, y=24, col sep=comma] {\root/c.csv};
\addplot table [x=t, y=25, col sep=comma] {\root/c.csv};
\addplot table [x=t, y=26, col sep=comma] {\root/c.csv};
\addplot table [x=t, y=27, col sep=comma] {\root/c.csv};
\addplot table [x=t, y=28, col sep=comma] {\root/c.csv};
\addplot[draw opacity=1] table [x=t, y=29, col sep=comma] {\root/c.csv};
\addplot table [x=t, y=30, col sep=comma] {\root/c.csv};
\addplot[draw opacity=1] table [x=t, y=31, col sep=comma] {\root/c.csv};

\end{groupplot}

\end{tikzpicture}

\end{frame}

\begin{frame}
\begin{tikzpicture}
%
%	\frametitle{Example}
%		
\def\i{20}
\def\root{images/rnn_examples/5}


\begin{groupplot}[
group style = {
group size = 1 by 7,
xlabels at=edge bottom,
xticklabels at=edge bottom,
vertical sep=0pt,
},
width=\textwidth,
%		hide axis,
enlargelimits=.1,
height=2.5cm,
%		ymin=0, ymax=1.4,
no marks,
]
\nextgroupplot[draw opacity=.8, smooth=0.01, thin]
\addplot[b11color, mark=*,mark size=.5pt] table [x=t, y=B11, col sep=comma, forget plot] {\root/x.csv};
\addplot[b12color, mark=*,mark size=.5pt] table [x=t, y=B12, col sep=comma] {\root/x.csv};

\addplot[b5color, mark=*,mark size=.5pt] table [x=t, y=B5, col sep=comma, forget plot] {\root/x.csv};
\addplot[b6color, mark=*,mark size=.5pt] table [x=t, y=B6, col sep=comma, forget plot] {\root/x.csv};
\addplot[b7color, mark=*,mark size=.5pt] table [x=t, y=B7, col sep=comma, forget plot] {\root/x.csv};
\addplot[b8color, mark=*,mark size=.5pt] table [x=t, y=B8, col sep=comma, forget plot] {\root/x.csv};
\addplot[b8Acolor, mark=*,mark size=.5pt] table [x=t, y=B8A, col sep=comma] {\root/x.csv};

\addplot[b2color, mark=*,mark size=.5pt] table [x=t, y=B2, col sep=comma, forget plot] {\root/x.csv};
\addplot[b3color, mark=*,mark size=.5pt] table [x=t, y=B3, col sep=comma, forget plot] {\root/x.csv};
\addplot[b4color, mark=*,mark size=.5pt] table [x=t, y=B4, col sep=comma] {\root/x.csv};

\nextgroupplot[cycle list name=featurecolorlist, draw opacity=.1, smooth=0.01, thin, ylabel=$\V{i}$]

\addplot table [x=t,  y=0, col sep=comma] {\root/i.csv};
\addplot table [x=t,  y=1, col sep=comma] {\root/i.csv};
\addplot table [x=t,  y=2, col sep=comma] {\root/i.csv};
\addplot table [x=t,  y=3, col sep=comma] {\root/i.csv};
\addplot table [x=t,  y=4, col sep=comma] {\root/i.csv};
\addplot table [x=t,  y=5, col sep=comma] {\root/i.csv};
\addplot table [x=t,  y=6, col sep=comma] {\root/i.csv};
\addplot table [x=t,  y=7, col sep=comma] {\root/i.csv};
\addplot table [x=t,  y=8, col sep=comma] {\root/i.csv};
\addplot table [x=t,  y=9, col sep=comma] {\root/i.csv};
\addplot table [x=t, y=10, col sep=comma] {\root/i.csv};
\addplot table [x=t, y=11, col sep=comma] {\root/i.csv};
\addplot table [x=t, y=12, col sep=comma] {\root/i.csv};
\addplot table [x=t, y=13, col sep=comma] {\root/i.csv};
\addplot table [x=t, y=14, col sep=comma] {\root/i.csv};
\addplot table [x=t, y=15, col sep=comma] {\root/i.csv};
\addplot table [x=t, y=16, col sep=comma] {\root/i.csv};
\addplot table [x=t, y=17, col sep=comma] {\root/i.csv};
\addplot table [x=t, y=18, col sep=comma] {\root/i.csv};
\addplot[draw opacity=1] table [x=t, y=19, col sep=comma] {\root/i.csv};
\addplot table [x=t, y=20, col sep=comma] {\root/i.csv};
\addplot[draw opacity=1] table [x=t, y=21, col sep=comma] {\root/i.csv};
\addplot table [x=t, y=22, col sep=comma] {\root/i.csv};
\addplot table [x=t, y=23, col sep=comma] {\root/i.csv};
\addplot table [x=t, y=24, col sep=comma] {\root/i.csv};
\addplot table [x=t, y=25, col sep=comma] {\root/i.csv};
\addplot table [x=t, y=26, col sep=comma] {\root/i.csv};
\addplot table [x=t, y=27, col sep=comma] {\root/i.csv};
\addplot table [x=t, y=28, col sep=comma] {\root/i.csv};
\addplot table [x=t, y=29, col sep=comma] {\root/i.csv};
\addplot table [x=t, y=30, col sep=comma] {\root/i.csv};
\addplot table [x=t, y=31, col sep=comma] {\root/i.csv};

\nextgroupplot[cycle list name=featurecolorlist, draw opacity=.2, smooth=0.01, thin, ylabel=$\V{f}$]

\addplot table [x=t,  y=0, col sep=comma] {\root/f.csv};
\addplot table [x=t,  y=1, col sep=comma] {\root/f.csv};
\addplot table [x=t,  y=2, col sep=comma] {\root/f.csv};
\addplot table [x=t,  y=3, col sep=comma] {\root/f.csv};
\addplot table [x=t,  y=4, col sep=comma] {\root/f.csv};
\addplot table [x=t,  y=5, col sep=comma] {\root/f.csv};
\addplot table [x=t,  y=6, col sep=comma] {\root/f.csv};
\addplot table [x=t,  y=7, col sep=comma] {\root/f.csv};
\addplot table [x=t,  y=8, col sep=comma] {\root/f.csv};
\addplot table [x=t,  y=9, col sep=comma] {\root/f.csv};
\addplot table [x=t, y=10, col sep=comma] {\root/f.csv};
\addplot table [x=t, y=11, col sep=comma] {\root/f.csv};
\addplot table [x=t, y=12, col sep=comma] {\root/f.csv};
\addplot table [x=t, y=13, col sep=comma] {\root/f.csv};
\addplot table [x=t, y=14, col sep=comma] {\root/f.csv};
\addplot table [x=t, y=15, col sep=comma] {\root/f.csv};
\addplot table [x=t, y=16, col sep=comma] {\root/f.csv};
\addplot table [x=t, y=17, col sep=comma] {\root/f.csv};
\addplot table [x=t, y=18, col sep=comma] {\root/f.csv};
\addplot[draw opacity=1] table [x=t, y=19, col sep=comma] {\root/f.csv};
\addplot table [x=t, y=20, col sep=comma] {\root/f.csv};
\addplot[draw opacity=1] table [x=t, y=21, col sep=comma] {\root/f.csv};
\addplot table [x=t, y=22, col sep=comma] {\root/f.csv};
\addplot table [x=t, y=23, col sep=comma] {\root/f.csv};
\addplot table [x=t, y=24, col sep=comma] {\root/f.csv};
\addplot table [x=t, y=25, col sep=comma] {\root/f.csv};
\addplot table [x=t, y=26, col sep=comma] {\root/f.csv};
\addplot table [x=t, y=27, col sep=comma] {\root/f.csv};
\addplot table [x=t, y=28, col sep=comma] {\root/f.csv};
\addplot table [x=t, y=29, col sep=comma] {\root/f.csv};
\addplot table [x=t, y=30, col sep=comma] {\root/f.csv};
\addplot table [x=t, y=31, col sep=comma] {\root/f.csv};

\nextgroupplot[cycle list name=featurecolorlist, draw opacity=.2, smooth=0.01, thin, ylabel=$\V{g}$]

\addplot table [x=t,  y=0, col sep=comma] {\root/g.csv};
\addplot table [x=t,  y=1, col sep=comma] {\root/g.csv};
\addplot table [x=t,  y=2, col sep=comma] {\root/g.csv};
\addplot table [x=t,  y=3, col sep=comma] {\root/g.csv};
\addplot table [x=t,  y=4, col sep=comma] {\root/g.csv};
\addplot table [x=t,  y=5, col sep=comma] {\root/g.csv};
\addplot table [x=t,  y=6, col sep=comma] {\root/g.csv};
\addplot table [x=t,  y=7, col sep=comma] {\root/g.csv};
\addplot table [x=t,  y=8, col sep=comma] {\root/g.csv};
\addplot table [x=t,  y=9, col sep=comma] {\root/g.csv};
\addplot table [x=t, y=10, col sep=comma] {\root/g.csv};
\addplot table [x=t, y=11, col sep=comma] {\root/g.csv};
\addplot table [x=t, y=12, col sep=comma] {\root/g.csv};
\addplot table [x=t, y=13, col sep=comma] {\root/g.csv};
\addplot table [x=t, y=14, col sep=comma] {\root/g.csv};
\addplot table [x=t, y=15, col sep=comma] {\root/g.csv};
\addplot table [x=t, y=16, col sep=comma] {\root/g.csv};
\addplot table [x=t, y=17, col sep=comma] {\root/g.csv};
\addplot table [x=t, y=18, col sep=comma] {\root/g.csv};
\addplot[draw opacity=1] table [x=t, y=19, col sep=comma] {\root/g.csv};
\addplot table [x=t, y=20, col sep=comma] {\root/g.csv};
\addplot[draw opacity=1] table [x=t, y=21, col sep=comma] {\root/g.csv};
\addplot table [x=t, y=22, col sep=comma] {\root/g.csv};
\addplot table [x=t, y=23, col sep=comma] {\root/g.csv};
\addplot table [x=t, y=24, col sep=comma] {\root/g.csv};
\addplot table [x=t, y=25, col sep=comma] {\root/g.csv};
\addplot table [x=t, y=26, col sep=comma] {\root/g.csv};
\addplot table [x=t, y=27, col sep=comma] {\root/g.csv};
\addplot table [x=t, y=28, col sep=comma] {\root/g.csv};
\addplot table [x=t, y=29, col sep=comma] {\root/g.csv};
\addplot table [x=t, y=30, col sep=comma] {\root/g.csv};
\addplot table [x=t, y=31, col sep=comma] {\root/g.csv};

\nextgroupplot[cycle list name=featurecolorlist, draw opacity=.2, smooth=0.01, thin, ylabel=$\V{o}$]

\addplot table [x=t,  y=0, col sep=comma] {\root/o.csv};
\addplot table [x=t,  y=1, col sep=comma] {\root/o.csv};
\addplot table [x=t,  y=2, col sep=comma] {\root/o.csv};
\addplot table [x=t,  y=3, col sep=comma] {\root/o.csv};
\addplot table [x=t,  y=4, col sep=comma] {\root/o.csv};
\addplot table [x=t,  y=5, col sep=comma] {\root/o.csv};
\addplot table [x=t,  y=6, col sep=comma] {\root/o.csv};
\addplot table [x=t,  y=7, col sep=comma] {\root/o.csv};
\addplot table [x=t,  y=8, col sep=comma] {\root/o.csv};
\addplot table [x=t,  y=9, col sep=comma] {\root/o.csv};
\addplot table [x=t, y=10, col sep=comma] {\root/o.csv};
\addplot table [x=t, y=11, col sep=comma] {\root/o.csv};
\addplot table [x=t, y=12, col sep=comma] {\root/o.csv};
\addplot table [x=t, y=13, col sep=comma] {\root/o.csv};
\addplot table [x=t, y=14, col sep=comma] {\root/o.csv};
\addplot table [x=t, y=15, col sep=comma] {\root/o.csv};
\addplot table [x=t, y=16, col sep=comma] {\root/o.csv};
\addplot table [x=t, y=17, col sep=comma] {\root/o.csv};
\addplot table [x=t, y=18, col sep=comma] {\root/o.csv};
\addplot[draw opacity=1] table [x=t, y=19, col sep=comma] {\root/o.csv};
\addplot table [x=t, y=20, col sep=comma] {\root/o.csv};
\addplot[draw opacity=1] table [x=t, y=21, col sep=comma] {\root/o.csv};
\addplot table [x=t, y=22, col sep=comma] {\root/o.csv};
\addplot table [x=t, y=23, col sep=comma] {\root/o.csv};
\addplot table [x=t, y=24, col sep=comma] {\root/o.csv};
\addplot table [x=t, y=25, col sep=comma] {\root/o.csv};
\addplot table [x=t, y=26, col sep=comma] {\root/o.csv};
\addplot table [x=t, y=27, col sep=comma] {\root/o.csv};
\addplot table [x=t, y=28, col sep=comma] {\root/o.csv};
\addplot table [x=t, y=29, col sep=comma] {\root/o.csv};
\addplot table [x=t, y=30, col sep=comma] {\root/o.csv};
\addplot table [x=t, y=31, col sep=comma] {\root/o.csv};

\nextgroupplot[cycle list name=featurecolorlist, draw opacity=.2, smooth=0.01, thin, ylabel=$\V{h}$]

\addplot table [x=t,  y=0, col sep=comma] {\root/h.csv};
\addplot table [x=t,  y=1, col sep=comma] {\root/h.csv};
\addplot table [x=t,  y=2, col sep=comma] {\root/h.csv};
\addplot table [x=t,  y=3, col sep=comma] {\root/h.csv};
\addplot table [x=t,  y=4, col sep=comma] {\root/h.csv};
\addplot table [x=t,  y=5, col sep=comma] {\root/h.csv};
\addplot table [x=t,  y=6, col sep=comma] {\root/h.csv};
\addplot table [x=t,  y=7, col sep=comma] {\root/h.csv};
\addplot table [x=t,  y=8, col sep=comma] {\root/h.csv};
\addplot table [x=t,  y=9, col sep=comma] {\root/h.csv};
\addplot table [x=t, y=10, col sep=comma] {\root/h.csv};
\addplot table [x=t, y=11, col sep=comma] {\root/h.csv};
\addplot table [x=t, y=12, col sep=comma] {\root/h.csv};
\addplot table [x=t, y=13, col sep=comma] {\root/h.csv};
\addplot table [x=t, y=14, col sep=comma] {\root/h.csv};
\addplot table [x=t, y=15, col sep=comma] {\root/h.csv};
\addplot table [x=t, y=16, col sep=comma] {\root/h.csv};
\addplot table [x=t, y=17, col sep=comma] {\root/h.csv};
\addplot table [x=t, y=18, col sep=comma] {\root/h.csv};
\addplot[draw opacity=1] table [x=t, y=19, col sep=comma] {\root/h.csv};
\addplot table [x=t, y=20, col sep=comma] {\root/h.csv};
\addplot[draw opacity=1] table [x=t, y=21, col sep=comma] {\root/h.csv};
\addplot table [x=t, y=22, col sep=comma] {\root/h.csv};
\addplot table [x=t, y=23, col sep=comma] {\root/h.csv};
\addplot table [x=t, y=24, col sep=comma] {\root/h.csv};
\addplot table [x=t, y=25, col sep=comma] {\root/h.csv};
\addplot table [x=t, y=26, col sep=comma] {\root/h.csv};
\addplot table [x=t, y=27, col sep=comma] {\root/h.csv};
\addplot table [x=t, y=28, col sep=comma] {\root/h.csv};
\addplot table [x=t, y=29, col sep=comma] {\root/h.csv};
\addplot table [x=t, y=30, col sep=comma] {\root/h.csv};
\addplot table [x=t, y=31, col sep=comma] {\root/h.csv};

\nextgroupplot[cycle list name=featurecolorlist, draw opacity=.2, smooth=0.01, thin, ylabel=$\V{c}$]
%
\addplot table [x=t,  y=0, col sep=comma] {\root/c.csv};
\addplot table [x=t,  y=1, col sep=comma] {\root/c.csv};
\addplot table [x=t,  y=2, col sep=comma] {\root/c.csv};
\addplot table [x=t,  y=3, col sep=comma] {\root/c.csv};
\addplot table [x=t,  y=4, col sep=comma] {\root/c.csv};
\addplot table [x=t,  y=5, col sep=comma] {\root/c.csv};
\addplot table [x=t,  y=6, col sep=comma] {\root/c.csv};
\addplot table [x=t,  y=7, col sep=comma] {\root/c.csv};
\addplot table [x=t,  y=8, col sep=comma] {\root/c.csv};
\addplot table [x=t,  y=9, col sep=comma] {\root/c.csv};
\addplot table [x=t, y=10, col sep=comma] {\root/c.csv};
\addplot table [x=t, y=11, col sep=comma] {\root/c.csv};
\addplot table [x=t, y=12, col sep=comma] {\root/c.csv};
\addplot table [x=t, y=13, col sep=comma] {\root/c.csv};
\addplot table [x=t, y=14, col sep=comma] {\root/c.csv};
\addplot table [x=t, y=15, col sep=comma] {\root/c.csv};
\addplot table [x=t, y=16, col sep=comma] {\root/c.csv};
\addplot table [x=t, y=17, col sep=comma] {\root/c.csv};
\addplot table [x=t, y=18, col sep=comma] {\root/c.csv};
\addplot[draw opacity=1] table [x=t, y=19, col sep=comma] {\root/c.csv};
\addplot table [x=t, y=20, col sep=comma] {\root/c.csv};
\addplot[draw opacity=1] table [x=t, y=21, col sep=comma] {\root/c.csv};
\addplot table [x=t, y=22, col sep=comma] {\root/c.csv};
\addplot table [x=t, y=23, col sep=comma] {\root/c.csv};
\addplot table [x=t, y=24, col sep=comma] {\root/c.csv};
\addplot table [x=t, y=25, col sep=comma] {\root/c.csv};
\addplot table [x=t, y=26, col sep=comma] {\root/c.csv};
\addplot table [x=t, y=27, col sep=comma] {\root/c.csv};
\addplot table [x=t, y=28, col sep=comma] {\root/c.csv};
\addplot table [x=t, y=29, col sep=comma] {\root/c.csv};
\addplot table [x=t, y=30, col sep=comma] {\root/c.csv};
\addplot table [x=t, y=31, col sep=comma] {\root/c.csv};

\end{groupplot}

\end{tikzpicture}

\end{frame}

\begin{frame}
\begin{tikzpicture}
%
%	\frametitle{Example}
%		

\def\root{images/rnn_examples/26}


\begin{groupplot}[
group style = {
group size = 1 by 3,
xlabels at=edge bottom,
xticklabels at=edge bottom,
vertical sep=0pt,
},
width=\textwidth,
%		hide axis,
enlargelimits=.1,
height=4cm,
ymin=-.2, ymax=.2,
no marks,
]
\nextgroupplot[draw opacity=.8, smooth=0.01, thin, ymin=0, ymax=1]
\addplot[b11color, mark=*,mark size=.5pt] table [x=t, y=B11, col sep=comma, forget plot] {\root/x.csv};
\addplot[b12color, mark=*,mark size=.5pt] table [x=t, y=B12, col sep=comma] {\root/x.csv};

\addplot[b5color, mark=*,mark size=.5pt] table [x=t, y=B5, col sep=comma, forget plot] {\root/x.csv};
\addplot[b6color, mark=*,mark size=.5pt] table [x=t, y=B6, col sep=comma, forget plot] {\root/x.csv};
\addplot[b7color, mark=*,mark size=.5pt] table [x=t, y=B7, col sep=comma, forget plot] {\root/x.csv};
\addplot[b8color, mark=*,mark size=.5pt] table [x=t, y=B8, col sep=comma, forget plot] {\root/x.csv};
\addplot[b8Acolor, mark=*,mark size=.5pt] table [x=t, y=B8A, col sep=comma] {\root/x.csv};

\addplot[b2color, mark=*,mark size=.5pt] table [x=t, y=B2, col sep=comma, forget plot] {\root/x.csv};
\addplot[b3color, mark=*,mark size=.5pt] table [x=t, y=B3, col sep=comma, forget plot] {\root/x.csv};
\addplot[b4color, mark=*,mark size=.5pt] table [x=t, y=B4, col sep=comma] {\root/x.csv};

\nextgroupplot[cycle list name=featurecolorlist, draw opacity=.1, smooth=0.01, thin, ylabel=$\V{h}$]

\addplot table [x=t,  y=0, col sep=comma] {\root/h.csv};
\addplot table [x=t,  y=1, col sep=comma] {\root/h.csv};
\addplot table [x=t,  y=2, col sep=comma] {\root/h.csv};
\addplot table [x=t,  y=3, col sep=comma] {\root/h.csv};
\addplot table [x=t,  y=4, col sep=comma] {\root/h.csv};
\addplot table [x=t,  y=5, col sep=comma] {\root/h.csv};
\addplot table [x=t,  y=6, col sep=comma] {\root/h.csv};
\addplot table [x=t,  y=7, col sep=comma] {\root/h.csv};
\addplot table [x=t,  y=8, col sep=comma] {\root/h.csv};
\addplot table [x=t,  y=9, col sep=comma] {\root/h.csv};
\addplot table [x=t, y=10, col sep=comma] {\root/h.csv};
\addplot[draw opacity=1] table [x=t, y=11, col sep=comma] {\root/h.csv};
\addplot table [x=t, y=12, col sep=comma] {\root/h.csv};
\addplot table [x=t, y=13, col sep=comma] {\root/h.csv};
\addplot table [x=t, y=14, col sep=comma] {\root/h.csv};
\addplot table [x=t, y=15, col sep=comma] {\root/h.csv};
\addplot table [x=t, y=16, col sep=comma] {\root/h.csv};
\addplot table [x=t, y=17, col sep=comma] {\root/h.csv};
\addplot table [x=t, y=18, col sep=comma] {\root/h.csv};
\addplot table [x=t, y=19, col sep=comma] {\root/h.csv};
\addplot table [x=t, y=20, col sep=comma] {\root/h.csv};
\addplot table [x=t, y=21, col sep=comma] {\root/h.csv};
\addplot table [x=t, y=22, col sep=comma] {\root/h.csv};
\addplot[draw opacity=1] table [x=t, y=23, col sep=comma] {\root/h.csv};
\addplot table [x=t, y=24, col sep=comma] {\root/h.csv};
\addplot table [x=t, y=25, col sep=comma] {\root/h.csv};
\addplot table [x=t, y=26, col sep=comma] {\root/h.csv};
\addplot table [x=t, y=27, col sep=comma] {\root/h.csv};
\addplot table [x=t, y=28, col sep=comma] {\root/h.csv};
\addplot table [x=t, y=29, col sep=comma] {\root/h.csv};
\addplot table [x=t, y=30, col sep=comma] {\root/h.csv};
\addplot table [x=t, y=31, col sep=comma] {\root/h.csv};

\nextgroupplot[cycle list name=featurecolorlist, draw opacity=.1, smooth=0.01, thin, ylabel=$\V{c}$]
%
\addplot table [x=t,  y=0, col sep=comma] {\root/c.csv};
\addplot table [x=t,  y=1, col sep=comma] {\root/c.csv};
\addplot table [x=t,  y=2, col sep=comma] {\root/c.csv};
\addplot table [x=t,  y=3, col sep=comma] {\root/c.csv};
\addplot table [x=t,  y=4, col sep=comma] {\root/c.csv};
\addplot table [x=t,  y=5, col sep=comma] {\root/c.csv};
\addplot table [x=t,  y=6, col sep=comma] {\root/c.csv};
\addplot table [x=t,  y=7, col sep=comma] {\root/c.csv};
\addplot table [x=t,  y=8, col sep=comma] {\root/c.csv};
\addplot table [x=t,  y=9, col sep=comma] {\root/c.csv};
\addplot table [x=t, y=10, col sep=comma] {\root/c.csv};
\addplot[draw opacity=1] table [x=t, y=11, col sep=comma] {\root/c.csv};
\addplot table [x=t, y=12, col sep=comma] {\root/c.csv};
\addplot table [x=t, y=13, col sep=comma] {\root/c.csv};
\addplot table [x=t, y=14, col sep=comma] {\root/c.csv};
\addplot table [x=t, y=15, col sep=comma] {\root/c.csv};
\addplot table [x=t, y=16, col sep=comma] {\root/c.csv};
\addplot table [x=t, y=17, col sep=comma] {\root/c.csv};
\addplot table [x=t, y=18, col sep=comma] {\root/c.csv};
\addplot table [x=t, y=19, col sep=comma] {\root/c.csv};
\addplot table [x=t, y=20, col sep=comma] {\root/c.csv};
\addplot table [x=t, y=21, col sep=comma] {\root/c.csv};
\addplot table [x=t, y=22, col sep=comma] {\root/c.csv};
\addplot[draw opacity=1] table [x=t, y=23, col sep=comma] {\root/c.csv};
\addplot table [x=t, y=24, col sep=comma] {\root/c.csv};
\addplot table [x=t, y=25, col sep=comma] {\root/c.csv};
\addplot table [x=t, y=26, col sep=comma] {\root/c.csv};
\addplot table [x=t, y=27, col sep=comma] {\root/c.csv};
\addplot table [x=t, y=28, col sep=comma] {\root/c.csv};
\addplot table [x=t, y=29, col sep=comma] {\root/c.csv};
\addplot table [x=t, y=30, col sep=comma] {\root/c.csv};
\addplot table [x=t, y=31, col sep=comma] {\root/c.csv};

\end{groupplot}

\end{tikzpicture}

\end{frame}

%\begin{frame}
%\frametitle{Ablation Experiment on Cloudy data}
%\input{images/scl.tikz}
%\input{images/clouds.tikz}
%\end{frame}
%
%
%\begin{frame}
%\frametitle{Remembering Karpathy's "Unreasonable Effectiveness of RNNs"}
%
%\includegraphics[width=\textwidth]{images/karpathy}
%
%\url{http://karpathy.github.io/2015/05/21/rnn-effectiveness/}
%\end{frame}


\newcounter{tcounter}

\newcommand{\drawcbar}[5]{
	\def\tnode{#1} % iGate0
	\def\cmap{#2} % inferno
	\def\cbarheight{#3} % 1cm
	%	\def\vmin{#4} % 0
	%	\def\vmax{#5} % 1
	
	\node[xshift=1mm] (cbar\tnode) at (\tnode.east){\includegraphics[height=2mm, width=\cbarheight, angle=90,origin=c]{\cmap}};
	\node[anchor=south] at (cbar\tnode.south east) {\tiny #4};
	\node[anchor=north] at (cbar\tnode.north east) {\tiny #5};
}
\newcommand{\imagecolumns}[2]{
		\foreach \t in {#1,...,#2}{
			\node(times) at (\thetcounter*\hdist,0*\vdist){$t_{\the\numexpr\t+1\relax}$};
			\node[activation](xlast) at (\thetcounter*\hdist,-1*\vdist){\includegraphics[width=\imagewidth]{\folder/sample\s/time\t_x}};
			
			\node[activation, yshift=-1mm](flast1) at (\thetcounter*\hdist,-2*\vdist){\includegraphics[width=\imagewidth]{\folder/sample\s/time\t/\done_fGate}};
			\node[activation, yshift=-1mm](ilast1) at (\thetcounter*\hdist,-3*\vdist){\includegraphics[width=\imagewidth]{\folder/sample\s/time\t/\done_iGate}};
			\node[activation, yshift=-1mm](jlast1) at (\thetcounter*\hdist,-4*\vdist){\includegraphics[width=\imagewidth]{\folder/sample\s/time\t/\done_jGate}};
			\node[activation, yshift=-1mm](clast1) at (\thetcounter*\hdist,-5*\vdist){\includegraphics[width=\imagewidth]{\folder/sample\s/time\t/\done_state}};
			
			\stepcounter{tcounter}
		}
	}
	
\newcommand{\dotscolumn}{
	\node[activation] at (\thetcounter*\hdist,-1*\vdist){$\dots$};
	
	\node[activation,yshift=-1mm] at (\thetcounter*\hdist,-2*\vdist){$\dots$};
	\node[activation,yshift=-1mm] at (\thetcounter*\hdist,-3*\vdist){$\dots$};
	\node[activation,yshift=-1mm] at (\thetcounter*\hdist,-4*\vdist){$\dots$};
	\node[activation,yshift=-1mm] at (\thetcounter*\hdist,-5*\vdist){$\dots$};
	\stepcounter{tcounter}
	}

\newcommand{\figactivations}[2]{
	% 1: number of sample
	% 2. number of activation
	
	\setcounter{tcounter}{0}
	
	\begin{tikzpicture}
	%\figGates{0}{18}{1}{3}
	\def\folder{images/activations/48px}
	
	\tikzstyle{activation}=[]
	\def\s{#1}
	
	\def\done{#2}
%	\def\dtwo{22}
%	\def\dthree{47}
	
	%% grid distance vertical and horizontal
	\def\vdist{8.5mm}
	\def\hdist{8.5mm}
	
	\def\imagewidth{8mm}
	
	
	%% descriptions top row
	\node[anchor=west] at (-1.5*\hdist,-1*\vdist){$\V{x}$};
	\node[anchor=west, yshift=-1mm](f1) at (-1.5*\hdist,-2*\vdist){$\V{f}^{(\done)}$};
	\node[anchor=west, yshift=-1mm](i1) at (-1.5*\hdist,-3*\vdist){$\V{i}^{(\done)}$};
	\node[anchor=west, yshift=-1mm](j1) at (-1.5*\hdist,-4*\vdist){$\V{j}^{(\done)}$};
	\node[anchor=west, yshift=-1mm](c1) at (-1.5*\hdist,-5*\vdist){$\V{c}^{(\done)}$};


	\imagecolumns{0}{1}
	\dotscolumn
	\imagecolumns{9}{14}
	\dotscolumn
	\imagecolumns{30}{34}
	
	
	\drawcbar{flast1}{images/activations/inferno}{\imagewidth}{0}{1}
	\drawcbar{ilast1}{images/activations/inferno}{\imagewidth}{0}{1}
	\drawcbar{jlast1}{images/activations/RdBu_r}{\imagewidth}{-1}{1}
	\drawcbar{clast1}{images/activations/RdBu_r}{\imagewidth}{-1}{1}
	
	
	
	\end{tikzpicture}
}
\begin{frame}
\frametitle{Internal States Encode increasingly Classification Features}
\framesubtitle{LSTM cell \textbf{47} of 256}
\figactivations{1}{3}
\end{frame}
%%
%
\begin{frame}
\frametitle{Found Cloud Masking Cells in the RNN}
\framesubtitle{LSTM cell \textbf{47} of 256}
\figactivations{1}{47}
\end{frame}

\begin{frame}
\frametitle{Gate Activations are complicated}

\begin{itemize}
\item something seems to happen with the gates given the input
\item hard to visually interpret
\item still classification accuracy circa 80\% $\leftarrow$ gates seem to follow a purpose
\item information likely encoded in a linear combination in multiple dimensions over multiple layers
\item the designed purpose of the gates \emph{input},\emph{forget},\emph{output}, etc. may not be very meaningful
\end{itemize}
\end{frame}

{\setbeamercolor{background canvas}{bg=tumbluedark}
\begin{frame}[plain]

\vspace{8em}
\begin{center}
\Huge\color{tumwhite}
One more tool in our Deep Learning toolbox....
\end{center}\color{white}

\end{frame}
}

{\setbeamercolor{background canvas}{bg=white}
\begin{frame}[plain]

\vspace{8em}
\begin{center}
\Huge\color{tumbluedark}
Gradients
\end{center}\color{white}

\end{frame}
}

\begin{frame}
\only<1-3>{\frametitle{Usually we use gradients to adjust $\Mweight$...}}
\only<4->{\frametitle{We can also backprop to $\M{X}$}}

\begin{tikzpicture}
\node[font=\Huge](grad){
\only<1-3>{
$\frac{\partial \mathcal{L}(\V{y},f_\Mweight(\M{X}))}{\partial \Mweight}$
}
\only<4->{
$\frac{\partial \max(\yhat)}{\partial \M{X}}$
}
};

\node[above left=of grad, text width=12em](annotdx){
\only<2,3>{how do we have to change the network weights $\Mweight$...}
\only<4,5>{what changes in the input $\M{X}$...} 
};
\node[below right=of grad, text width=12em](annotdy){
\only<3>{... to change (minimize) the loss $\mathcal{L}$?}
\only<5>{... would change the predicted score $\max(\yhat) = \max(\f_\Mweight(\VInput))$?}
};

\visible<2,3,4,5>{\draw[-stealth, shorten >= 1em, rounded corners] (annotdx) -| ($ (annotdx)!0.5!(grad) $) |- ($ (grad)+(-.5em, -1em) $);}
\visible<3,5>{\draw[-stealth, shorten >= 1em, rounded corners] (annotdy) -| ($ (annotdy)!0.5!(grad) $) |- ($ (grad)+(2.5em, 1em) $);}

\end{tikzpicture}
\end{frame}


\begin{frame}
\begin{tikzpicture}

\frametitle{Gradients from $\max(\yhat)$ to \M{X}}
\framesubtitle{Example 1}


\def\root{images/rnn_examples/5}


\begin{groupplot}[
group style = {
group size = 1 by 3,
xlabels at=edge bottom,
xticklabels at=edge bottom,
vertical sep=0pt,
},
width=\textwidth,
%		hide axis,
enlargelimits=.1,
height=4cm,
%ymin=-.2, ymax=.2,
%no marks,
]
\nextgroupplot[draw opacity=.8, smooth=0.01, ylabel=$\M{X}$]
\addplot[b11color, mark=*,mark size=.5pt] table [x=t, y=B11, col sep=comma, forget plot] {\root/x.csv};
\addplot[b12color, mark=*,mark size=.5pt] table [x=t, y=B12, col sep=comma] {\root/x.csv};

\addplot[b5color, mark=*,mark size=.5pt] table [x=t, y=B5, col sep=comma, forget plot] {\root/x.csv};
\addplot[b6color, mark=*,mark size=.5pt] table [x=t, y=B6, col sep=comma, forget plot] {\root/x.csv};
\addplot[b7color, mark=*,mark size=.5pt] table [x=t, y=B7, col sep=comma, forget plot] {\root/x.csv};
\addplot[b8color, mark=*,mark size=.5pt] table [x=t, y=B8, col sep=comma, forget plot] {\root/x.csv};
\addplot[b8Acolor, mark=*,mark size=.5pt] table [x=t, y=B8A, col sep=comma] {\root/x.csv};

\addplot[b2color, mark=*,mark size=.5pt] table [x=t, y=B2, col sep=comma, forget plot] {\root/x.csv};
\addplot[b3color, mark=*,mark size=.5pt] table [x=t, y=B3, col sep=comma, forget plot] {\root/x.csv};
\addplot[b4color, mark=*,mark size=.5pt] table [x=t, y=B4, col sep=comma] {\root/x.csv};

\nextgroupplot[draw opacity=.8, smooth=0.01, ylabel=$\frac{\partial \max(\yhat)}{\partial \V{X}}$]
\addplot[b11color, mark=*,mark size=.5pt] table [x=t, y=B11, col sep=comma, forget plot] {\root/dydx.csv};
\addplot[b12color, mark=*,mark size=.5pt] table [x=t, y=B12, col sep=comma] {\root/dydx.csv};

\addplot[b5color, mark=*,mark size=.5pt] table [x=t, y=B5, col sep=comma, forget plot] {\root/dydx.csv};
\addplot[b6color, mark=*,mark size=.5pt] table [x=t, y=B6, col sep=comma, forget plot] {\root/dydx.csv};
\addplot[b7color, mark=*,mark size=.5pt] table [x=t, y=B7, col sep=comma, forget plot] {\root/dydx.csv};
\addplot[b8color, mark=*,mark size=.5pt] table [x=t, y=B8, col sep=comma, forget plot] {\root/dydx.csv};
\addplot[b8Acolor, mark=*,mark size=.5pt] table [x=t, y=B8A, col sep=comma] {\root/dydx.csv};

\addplot[b2color, mark=*,mark size=.5pt] table [x=t, y=B2, col sep=comma, forget plot] {\root/dydx.csv};
\addplot[b3color, mark=*,mark size=.5pt] table [x=t, y=B3, col sep=comma, forget plot] {\root/dydx.csv};
\addplot[b4color, mark=*,mark size=.5pt] table [x=t, y=B4, col sep=comma] {\root/dydx.csv};

\end{groupplot}
\end{tikzpicture}
\end{frame}


\begin{frame}
\begin{tikzpicture}

\frametitle{Gradients from $\max(\yhat)$ to \M{X}}
\framesubtitle{Example 2}

\def\root{images/rnn_examples/6}


\begin{groupplot}[
group style = {
group size = 1 by 3,
xlabels at=edge bottom,
xticklabels at=edge bottom,
vertical sep=0pt,
},
width=\textwidth,
%		hide axis,
enlargelimits=.1,
height=4cm,
%ymin=-.2, ymax=.2,
%no marks,
]
\nextgroupplot[draw opacity=.8, smooth=0.01, ylabel=$\M{X}$]
\addplot[b11color, mark=*,mark size=.5pt] table [x=t, y=B11, col sep=comma, forget plot] {\root/x.csv};
\addplot[b12color, mark=*,mark size=.5pt] table [x=t, y=B12, col sep=comma] {\root/x.csv};

\addplot[b5color, mark=*,mark size=.5pt] table [x=t, y=B5, col sep=comma, forget plot] {\root/x.csv};
\addplot[b6color, mark=*,mark size=.5pt] table [x=t, y=B6, col sep=comma, forget plot] {\root/x.csv};
\addplot[b7color, mark=*,mark size=.5pt] table [x=t, y=B7, col sep=comma, forget plot] {\root/x.csv};
\addplot[b8color, mark=*,mark size=.5pt] table [x=t, y=B8, col sep=comma, forget plot] {\root/x.csv};
\addplot[b8Acolor, mark=*,mark size=.5pt] table [x=t, y=B8A, col sep=comma] {\root/x.csv};

\addplot[b2color, mark=*,mark size=.5pt] table [x=t, y=B2, col sep=comma, forget plot] {\root/x.csv};
\addplot[b3color, mark=*,mark size=.5pt] table [x=t, y=B3, col sep=comma, forget plot] {\root/x.csv};
\addplot[b4color, mark=*,mark size=.5pt] table [x=t, y=B4, col sep=comma] {\root/x.csv};

\nextgroupplot[draw opacity=.8, smooth=0.01, ylabel=$\frac{\partial \max(\yhat)}{\partial \V{X}}$]
\addplot[b11color, mark=*,mark size=.5pt] table [x=t, y=B11, col sep=comma, forget plot] {\root/dydx.csv};
\addplot[b12color, mark=*,mark size=.5pt] table [x=t, y=B12, col sep=comma] {\root/dydx.csv};

\addplot[b5color, mark=*,mark size=.5pt] table [x=t, y=B5, col sep=comma, forget plot] {\root/dydx.csv};
\addplot[b6color, mark=*,mark size=.5pt] table [x=t, y=B6, col sep=comma, forget plot] {\root/dydx.csv};
\addplot[b7color, mark=*,mark size=.5pt] table [x=t, y=B7, col sep=comma, forget plot] {\root/dydx.csv};
\addplot[b8color, mark=*,mark size=.5pt] table [x=t, y=B8, col sep=comma, forget plot] {\root/dydx.csv};
\addplot[b8Acolor, mark=*,mark size=.5pt] table [x=t, y=B8A, col sep=comma] {\root/dydx.csv};

\addplot[b2color, mark=*,mark size=.5pt] table [x=t, y=B2, col sep=comma, forget plot] {\root/dydx.csv};
\addplot[b3color, mark=*,mark size=.5pt] table [x=t, y=B3, col sep=comma, forget plot] {\root/dydx.csv};
\addplot[b4color, mark=*,mark size=.5pt] table [x=t, y=B4, col sep=comma] {\root/dydx.csv};

\end{groupplot}
\end{tikzpicture}
\end{frame}


\begin{frame}
\begin{tikzpicture}

\frametitle{Gradients from $\max(\yhat)$ to \M{X}}
\framesubtitle{Example 3}

\def\root{images/rnn_examples/7}


\begin{groupplot}[
group style = {
group size = 1 by 3,
xlabels at=edge bottom,
xticklabels at=edge bottom,
vertical sep=0pt,
},
width=\textwidth,
%		hide axis,
enlargelimits=.1,
height=4cm,
%ymin=-.2, ymax=.2,
%no marks,
]
\nextgroupplot[draw opacity=.8, smooth=0.01, ylabel=$\M{X}$]
\addplot[b11color, mark=*,mark size=.5pt] table [x=t, y=B11, col sep=comma, forget plot] {\root/x.csv};
\addplot[b12color, mark=*,mark size=.5pt] table [x=t, y=B12, col sep=comma] {\root/x.csv};

\addplot[b5color, mark=*,mark size=.5pt] table [x=t, y=B5, col sep=comma, forget plot] {\root/x.csv};
\addplot[b6color, mark=*,mark size=.5pt] table [x=t, y=B6, col sep=comma, forget plot] {\root/x.csv};
\addplot[b7color, mark=*,mark size=.5pt] table [x=t, y=B7, col sep=comma, forget plot] {\root/x.csv};
\addplot[b8color, mark=*,mark size=.5pt] table [x=t, y=B8, col sep=comma, forget plot] {\root/x.csv};
\addplot[b8Acolor, mark=*,mark size=.5pt] table [x=t, y=B8A, col sep=comma] {\root/x.csv};

\addplot[b2color, mark=*,mark size=.5pt] table [x=t, y=B2, col sep=comma, forget plot] {\root/x.csv};
\addplot[b3color, mark=*,mark size=.5pt] table [x=t, y=B3, col sep=comma, forget plot] {\root/x.csv};
\addplot[b4color, mark=*,mark size=.5pt] table [x=t, y=B4, col sep=comma] {\root/x.csv};

\nextgroupplot[draw opacity=.8, smooth=0.01, ylabel=$\frac{\partial \max(\yhat)}{\partial \V{X}}$]
\addplot[b11color, mark=*,mark size=.5pt] table [x=t, y=B11, col sep=comma, forget plot] {\root/dydx.csv};
\addplot[b12color, mark=*,mark size=.5pt] table [x=t, y=B12, col sep=comma] {\root/dydx.csv};

\addplot[b5color, mark=*,mark size=.5pt] table [x=t, y=B5, col sep=comma, forget plot] {\root/dydx.csv};
\addplot[b6color, mark=*,mark size=.5pt] table [x=t, y=B6, col sep=comma, forget plot] {\root/dydx.csv};
\addplot[b7color, mark=*,mark size=.5pt] table [x=t, y=B7, col sep=comma, forget plot] {\root/dydx.csv};
\addplot[b8color, mark=*,mark size=.5pt] table [x=t, y=B8, col sep=comma, forget plot] {\root/dydx.csv};
\addplot[b8Acolor, mark=*,mark size=.5pt] table [x=t, y=B8A, col sep=comma] {\root/dydx.csv};

\addplot[b2color, mark=*,mark size=.5pt] table [x=t, y=B2, col sep=comma, forget plot] {\root/dydx.csv};
\addplot[b3color, mark=*,mark size=.5pt] table [x=t, y=B3, col sep=comma, forget plot] {\root/dydx.csv};
\addplot[b4color, mark=*,mark size=.5pt] table [x=t, y=B4, col sep=comma] {\root/dydx.csv};

\end{groupplot}
\end{tikzpicture}
\end{frame}

\begin{frame}
	\frametitle{Jupyter Notebook}
	
	\Large
	\url{github.com/marccoru/phiweek-notebook/recurrence.ipynb}
\end{frame}
%
%\def\fps{3}
%%\tikzsetnextfilename{lstm}

\tikzstyle{operator} = [draw, circle, fill=tumbluemedium, draw=tumbluemedium, inner sep=0, text=white]
%\tikzstyle{function} = [draw, rectangle, fill=tumbluemedium, draw=tumbluemedium, text=white]
\tikzstyle{gate} = [fill=tumivory,draw,rounded corners=1pt, inner sep=2pt, minimum width=11mm, minimum height=11mm]
\tikzstyle{io} = [fill=tumwhite,draw,rounded corners=1pt, inner sep=2pt, minimum width=11mm, minimum height=11mm]

\tikzstyle{dummy} = [inner sep=0]
\tikzstyle{flow} = [rounded corners]
\tikzstyle{endflow} = [-stealth,flow]

\colorlet{boxcolor}{tumgraylight}
\tikzstyle{bigbox} = [rectangle, draw=tumivory, thick, fill=boxcolor, rounded corners, 
inner xsep=0ex, inner ysep=2ex]

\tikzset{pic shift/.store in=\shiftcoord,
	pic shift={(0,0)},
	lstmexplain/.pic = {
		\begin{scope}[shift={\shiftcoord},xscale=6,yscale=2.5]
			
			\node[dummy] (bl) at (0,0){}; % bottom left
			\node[dummy] (tr) at (1,1){}; % top right
			
			\node[dummy] (br) at ($ (bl -| tr) $){}; % bottom right
			\node[dummy] (tl) at ($ (bl |- tr) $){}; % top left
			
			\node[fit=(bl) (tr),bigbox] (-C) {};
			
			% input coordinate for rounded draw lines -> slightly right of tl
			\coordinate (-input) at (0.1,1); % top left
			
			% output coordinate for rounded draw lines -> slightly left of br
			\coordinate (-coutput) at (0.98,0); % bottom right
			\coordinate (-houtput) at (0.98,1); % bottom right
			
%			% gate distance
			\def\d{1/5}
			
			% gate heights
			\def\h{1/3}
			
			\coordinate (f)  at bl+(0.7*\d,0);
			\coordinate (i)  at bl+(1.8*\d,0);
			\coordinate (j)  at bl+(2.9*\d,0);
			\coordinate (o)  at bl+(4*\d,0);
			\coordinate (out) at bl+(4.7*\d,0);
			
			\coordinate (gates) at (0,2*\h);
			
			%\node[above=of tl](xt){$x_{t}$};
			%\node[left=of tl](htminus1){$h_{t-1}$};
			
			%\node[below=of br](ct){$c_{t}$};
%			\setlength{\thinmuskip}{0pt}
			
			\visible<2->{
			\node[gate](fgate) at ($ (gates -| f) $){
				$\VForgetGate_t$
				};
				\draw[endflow] (-input) -| (fgate);
			}
			
			\visible<3->{
			\node[gate](igate) at ($ (gates -| i) $){
				$\VInputGate_t$
				};
			}
				
			\visible<3->{
			\node[gate](jgate) at ($ (gates -| j) $){$\VModulationGate_t$};
			\node[operator](jmult) at ([shift={(0,-1.3*\h)}]jgate) {$ \odot $};
			\draw[endflow] (-input) -| (jgate);
			\draw[endflow] (jgate) -- (jmult);
			\draw[endflow] (-input) -| (igate); 
			\draw[endflow] (igate) |- (jmult);
			}
				
			\visible<4->{
			\node[gate](ogate) at ($ (gates -| o) $){
				$\VOutputGate_t$
				};
			\draw[endflow] (tl) -| (ogate);
			}
			
%			\coordinate (htminus1) at bl+(-.5,0);
%			\coordinate (ht) at bl+(-.5,0);
%			
			% forget gate 
			\visible<5->{
			\node[operator](fmult) at ($ (bl -| fgate) $) {$ \odot $};
			\draw[endflow] (fgate) -- (fmult); 
			\node[operator](cadd) at ($ (bl -| jgate) $) {$ + $};
			\draw[endflow] (jmult) -- (cadd); 
			\draw[flow] (fmult) -- (cadd) -- (-coutput);		
			}

			\visible<6->{
			\node[operator](outtanh) at ($ (jmult -| out) $) {$\odot$};
			\draw[endflow] (ogate) |- (outtanh);
			\draw[flow] (outtanh) |- (-houtput);
			\draw[endflow] (cadd) -| (outtanh);
			}
		\end{scope}
	}
}

\tikzset{pic shift/.store in=\shiftcoord,
	pic shift={(0,0)},
	lstmanim/.pic = {
		\begin{scope}[shift={\shiftcoord},xscale=6,yscale=2.5]
			
			\node[dummy] (bl) at (0,0){}; % bottom left
			\node[dummy] (tr) at (1,1){}; % top right
			
			\node[dummy] (br) at ($ (bl -| tr) $){}; % bottom right
			\node[dummy] (tl) at ($ (bl |- tr) $){}; % top left
			
			\node[fit=(bl) (tr),bigbox] (-C) {};
			
			% input coordinate for rounded draw lines -> slightly right of tl
			\coordinate (-input) at (0.1,1); % top left
			
			% output coordinate for rounded draw lines -> slightly left of br
			\coordinate (-coutput) at (0.98,0); % bottom right
			\coordinate (-houtput) at (0.98,1); % bottom right
			
			%			% gate distance
			\def\d{1/5}
			
			% gate heights
			\def\h{1/3}
			
			\coordinate (f)  at bl+(0.7*\d,0);
			\coordinate (i)  at bl+(1.8*\d,0);
			\coordinate (j)  at bl+(2.9*\d,0);
			\coordinate (o)  at bl+(4*\d,0);
			\coordinate (out) at bl+(4.7*\d,0);
			
			\coordinate (gates) at (0,2*\h);
			
			%\node[above=of tl](xt){$x_{t}$};
			%\node[left=of tl](htminus1){$h_{t-1}$};
			
			%\node[below=of br](ct){$c_{t}$};
			
			\node[gate, label={[label distance=0ex]265:$\VForgetGate_{t}$}](fgate) at ($ (gates -| f) $){
				\animategraphics[poster=25,width=1cm,autoplay,loop]{\fps}{images/activations/16494/f/3-}{1}{36}
				};
			\node[gate, label={[label distance=0ex]265:$\VInputGate_{t}$}](igate) at ($ (gates -| i) $){
				\animategraphics[poster=25,width=1cm,autoplay,loop]{\fps}{images/activations/16494/i/3-}{1}{36}
				};
			\node[gate, label={[label distance=0ex]265:$\VModulationGate_{t}$}](jgate) at ($ (gates -| j) $){
				\animategraphics[poster=25,width=1cm,autoplay,loop]{\fps}{images/activations/16494/j/3-}{1}{36}
				};
			\node[gate, label={[label distance=0ex]265:$\VOutputGate_{t}$}](ogate) at ($ (gates -| o) $){
				\animategraphics[poster=25,width=1cm,autoplay,loop]{\fps}{images/activations/16494/o/3-}{1}{36}
				};
			
			%			\coordinate (htminus1) at bl+(-.5,0);
			%			\coordinate (ht) at bl+(-.5,0);
			%			
			% forget gate
			\node[operator](fmult) at ($ (bl -| fgate) $) {$ \odot $};
			\draw[endflow] (-input) -| (fgate);
			\draw[endflow] (fgate) -- (fmult); 
			
			%			%j
			\node[operator](jmult) at ([shift={(0,-1.3*\h)}]jgate) {$ \odot $};
			\node[operator](cadd) at ($ (bl -| jgate) $) {$ + $};
			\draw[endflow] (-input) -| (jgate);
			\draw[endflow] (jgate) -- (jmult);
			\draw[endflow] (jmult) -- (cadd); 			
			
			%			%i	
			\draw[endflow] (-input) -| (igate);
			\draw[endflow] (igate) |- (jmult); 
			%
			%%			% outpu
			\node[operator](outtanh) at ($ (jmult -| out) $) {$\odot$};
			%			
			%			%o 
			\draw[endflow] (tl) -| (ogate);
			\draw[endflow] (ogate) |- (outtanh);
			\draw[flow] (outtanh) |- (-houtput);
			%			
			%			% output flow
			\draw[endflow] (cadd) -| (outtanh);
			\draw[flow] (fmult) -- (cadd) -- (-coutput);
			%			
			
		\end{scope}
	}
}

\tikzset{pic shift/.store in=\shiftcoord,
	pic shift={(0,0)},
	simplernn/.pic = {
		\begin{scope}[shift={\shiftcoord},xscale=2,yscale=.1]
			
			\node[dummy] (bl) at (0,0){}; % bottom left
			\node[dummy] (tr) at (1,1){}; % top right
			
			\node[dummy] (br) at ($ (bl -| tr) $){}; % bottom right
			\node[dummy] (tl) at ($ (bl |- tr) $){}; % top left
			
			\node[fit=(bl) (tr),bigbox] (-C) {};
			
			% input coordinate for rounded draw lines -> slightly right of tl
			\coordinate (-input) at (0.1,1); % top left
			
			% output coordinate for rounded draw lines -> slightly left of br
			\coordinate (-coutput) at (0.98,0); % bottom right
			\coordinate (-houtput) at (0.98,1); % bottom right
			
			%			% gate distance
			\def\d{1/5}
			
			% gate heights
			
			\node[] at ($ (-input)!0.5!(-houtput) $)(label){
				RNN
%				$\sigma(\concat{\VInput_t}{\VHidden_{t-1}}\MWeight$
			};
			\draw[endflow] (-input) -- (label);
			\draw[flow] (label) -- (-houtput);
			%\node[above=of tl](xt){$x_{t}$};
			%\node[left=of tl](htminus1){$h_{t-1}$};
			
			%\node[below=of br](ct){$c_{t}$};
			
%			\node[gate, label={[label distance=0ex]265:$\VForgetGate_{t}$}](fgate) at ($ (gates -| f) $){
%				\animategraphics[poster=25,width=1cm,autoplay,loop]{\fps}{images/activations/16494/f/3-}{1}{36}
%			};
%			\node[gate, label={[label distance=0ex]265:$\VInputGate_{t}$}](igate) at ($ (gates -| i) $){
%				\animategraphics[poster=25,width=1cm,autoplay,loop]{\fps}{images/activations/16494/i/3-}{1}{36}
%			};
%			\node[gate, label={[label distance=0ex]265:$\VModulationGate_{t}$}](jgate) at ($ (gates -| j) $){
%				\animategraphics[poster=25,width=1cm,autoplay,loop]{\fps}{images/activations/16494/j/3-}{1}{36}
%			};
%			\node[gate, label={[label distance=0ex]265:$\VOutputGate_{t}$}](ogate) at ($ (gates -| o) $){
%				\animategraphics[poster=25,width=1cm,autoplay,loop]{\fps}{images/activations/16494/o/3-}{1}{36}
%			};
			
			%			\coordinate (htminus1) at bl+(-.5,0);
			%			\coordinate (ht) at bl+(-.5,0);
			%			
			% forget gate
%			\node[operator](fmult) at ($ (bl -| fgate) $) {$ \odot $};
%			\draw[endflow] (-input) -| (fgate);
%			\draw[endflow] (fgate) -- (fmult); 
			
			%			%j
%			\node[operator](jmult) at ([shift={(0,-1.3*\h)}]jgate) {$ \odot $};
%			\node[operator](cadd) at ($ (bl -| jgate) $) {$ + $};
%			\draw[endflow] (-input) -| (jgate);
%			\draw[endflow] (jgate) -- (jmult);
%			\draw[endflow] (jmult) -- (cadd); 			
			
			%			%i	
%			\draw[endflow] (-input) -| (igate);
%			\draw[endflow] (igate) |- (jmult); 
			%
			%%			% outpu
%			\node[operator](outtanh) at ($ (jmult -| out) $) {$\odot$};
			%			
			%			%o 
%			\draw[endflow] (tl) -| (ogate);
%			\draw[endflow] (ogate) |- (outtanh);
%			\draw[flow] (outtanh) |- (-houtput);
			%			
			%			% output flow
%			\draw[endflow] (cadd) -| (outtanh);
%			\draw[flow] (fmult) -- (cadd) -- (-coutput);
			%			
			
		\end{scope}
	}
}

\tikzset{pic shift/.store in=\shiftcoord,
	pic shift={(0,0)},
	gru/.pic = {
		\begin{scope}[shift={\shiftcoord},xscale=6,yscale=2.5]
			
			\node[dummy] (bl) at (0,0){}; % bottom left
			\node[dummy] (tr) at (1,1){}; % top right
			
			\node[dummy] (br) at ($ (bl -| tr) $){}; % bottom right
			\node[dummy] (tl) at ($ (bl |- tr) $){}; % top left
			
			\node[fit=(bl) (tr),bigbox] (-C) {};
			
			%			% gate distance
			\def\d{1/5}
			
			% gate heights
			\def\h{1/4}
			
			
			% input coordinate for rounded draw lines -> slightly right of tl
			\coordinate (-xinput) at (0.3*\d,1); % top left
			\coordinate (-xinputflow) at (0.5*\d,1); % top left
			
			\coordinate (-hinput) at (0.2*\d,1); % top left
			\coordinate (-hinputflow) at (0.2*\d,.9); % top left
			
			% output coordinate for rounded draw lines -> slightly left of br
			\coordinate (-houtput) at (.98,1); % bottom right
			
			\coordinate (r)  at bl+(1*\d,0);
			\coordinate (rgap)at bl+(2*\d,0);
			\coordinate (u)  at bl+(3*\d,0);
			\coordinate (c)  at bl+(4*\d,0);
			
			\coordinate (out) at bl+(4.75*\d,0);
			
			\coordinate (abovegates) at (0,3.5*\h);
			\coordinate (gates) at (0,2.3*\h);
			\coordinate (belowgates) at ($(gates)!0.65!(bl)$);
			
			
%			\node[above=of tl](xt){$x_{t}$};
%			\node[left=of tl](htminus1){$h_{t-1}$};
			
			%\node[below=of br](ct){$c_{t}$};
			
			
			\node[gate](rgate) at ($ (gates -| r) $){r};
			\node[gate](ugate) at ($ (gates -| u) $){u};
			\node[gate](cgate) at ($ (gates -| c) $){$\tilde{\VHidden}$};
			
			\node[operator](cadd) at ($ (cgate |- bl) $) {$+$};
			\node[operator](cmult) at ($ (cgate |- belowgates) $) {$\odot$};
			
			\node[operator](rmult) at ($ (rgap |- belowgates) $) {$\odot$};
			
			\node[operator](umult) at ($ (u |- bl) $) {$\odot$};
			\draw[endflow] (ugate) -- (umult); 
			
			\draw[endflow] (-hinput)++(0,-.1) |- ($ (bl -| rgate) $) -| (rmult); 
			\draw[endflow] (rgate) |- (rmult);
			
			\draw[endflow] (cgate) -- (cmult);
			\draw[endflow] (ugate) |- (cmult);
			\draw[endflow] (cmult) -- (cadd);
			
			
			%%			
			\draw[endflow] (-xinputflow) -| (rgate); 
			\draw[endflow] (-xinputflow) -| ($ (rgate |- abovegates) $) -| (ugate); 
%%			\draw[endflow] (rgate) |- ++(.5\d,-\h) -| (rgate); 
			
			\draw[flow, draw=boxcolor,double=black,double distance=\pgflinewidth,ultra thick] (rmult) |- ($ (ugate |- tl) $);
			
			
			%\draw[flow] (tl) |- ($ (abovegates -| u) $) -| (ugate); 
			\draw[endflow] (-xinputflow) -| (cgate); 
			\draw[flow] (-hinput)++(-0.1, 0) -- (-xinputflow); 
			
			\draw[flow] (-hinputflow) |- (umult) -- (cadd) -| ($ (out) + (0,.5) $) |- (-houtput); %($(br)!0.5!(tr)$) 
			
			
			
			%% debug
%			\node at (gates) {\tiny{gates}};
%			\node at (abovegates) {\tiny{abovegates}};
%			\node at (belowgates) {\tiny{belowgates}};
%			\node at (-input) {\tiny{-input}};
%			\node at (-coutput) {\tiny{-coutput}};
%			\node at (-houtput) {\tiny{-houtput}};
%			\node at (f) {\tiny{f}};
%			\node at (i) {\tiny{i}};
%			\node at (j) {\tiny{j}};
%			\node at (o) {\tiny{o}};
%			\node at (tl) {\tiny{tl}};
%			\node at (br) {\tiny{br}};
%			\node at (bl) {\tiny{bl}};
%			\node at (tr) {\tiny{tr}};
%			\node at (out) {\tiny{out}};
			
		\end{scope}
	}
}

\tikzset{pic shift/.store in=\shiftcoord,
	pic shift={(0,0)},
	concat/.pic = {
		\node[](a) at (0, .5){$\V{a}$};
		\node[](b) at (0, -.5){$\V{b}$};
		\node[](out) at (1, 0){$\concat{ \V{a} }{ \V{b} }$};
		
		\draw[endflow] (a) |- (out);
		\draw[endflow] (b) |- (out);
	}
}

\tikzset{pic shift/.store in=\shiftcoord,
	pic shift={(0,0)},
	copy/.pic = {
		\node[](ain) at (0, 0){$\V{a}$};
		\node[](aout1) at (.5, .5){$\V{a}$};
		\node[](aout2) at (.5, -.5){$\V{a}$};
		\draw[endflow] (ain) -| (aout1);
		\draw[endflow] (ain) -| (aout2);
	}
}

\tikzset{pic shift/.store in=\shiftcoord,
	pic shift={(0,0)},
	fgate/.pic = {
		\begin{scope}[shift={\shiftcoord},xscale=1,yscale=1]
			
			\node[dummy] (tl_a) at (0,0){}; % bottom left
			\node[dummy] (br_a) at (1,1){}; % top right
			
			\node[fit=(br_a) (tr_a),gate,inner sep=0] (-C) {};
			
			\node[draw] (conv) at (0.5,0){$conv$}; % bottom left
			\node[draw] (bn) at (0.5,.5){$bn$}; % bottom left
			\node[draw] (sigmoid) at (0.5,1){$\sigma$}; % bottom left
				
		\end{scope}
	}
}

\newcommand{\gru}{
\begin{tikzpicture}[scale=1, node distance=2em]%,show background rectangle,background rectangle/.style={draw=red}]
\draw pic (GRU) at (0,0) {gru};
\node[io,left=of GRU-hinput](gru_htminus1){$\VHidden_{t-1}$};
\draw[rounded corners] (gru_htminus1) -| (GRU-hinputflow);
\node[io,above=of GRU-xinput](gru_xt){$\VInput_{t}$};

\draw[rounded corners] (gru_xt) |- (GRU-xinputflow);

\node[io,right=of GRU-houtput](gru_ht){$\VHidden_{t}$};
\draw[rounded corners] (GRU-houtput)--(gru_ht);
\end{tikzpicture}
}

\newcommand{\lstmexplain}{
	\begin{tikzpicture}[scale=1, node distance=2em]%,show background rectangle,background rectangle/.style={draw=red}]
	
	
	%\clip(0,0) rectangle (7,7);
	
	%\draw pic (B) at (5,0) {lstm};
	
	%\draw pic (C) at (10,0) {lstm};
	
	%\draw pic (C) at (2,2) {fgate};
	
	\draw pic (LSTM) at (0,0) {lstmexplain};
	\node[io,xshift=1ex,above=3em of LSTMtl, label=above:\phantom{$\VInput_{t}$}](xt){$\VInput_{t}$};%$x_{t}$
		
	\node[io,left=of LSTMtl](htminus1){$\VHidden_{t-1}$};
	
	\node[io,right=of LSTMbr](ct){$\VCellState_{t}$}; % $c_{t}$

	\node[io,left=of LSTMbl](ctminus1){$\VCellState_{t-1}$}; % 
		
	\node[io,right=of LSTMtr](ht){$\VHidden_{t}$};
	
	%% iterative connections
	\visible<7->{
	\draw[endflow] (ct) -- ($ (ct)+(0,-0.8) $) -| (ctminus1);
	\draw[endflow] (ht) -- ($ (ht)+(0,.8) $) -| (htminus1);
	}
		
	\visible<2->{
	\draw[rounded corners] (xt) |- (LSTM-input);
	}
	

	\visible<2->{
	\draw[endflow] (htminus1) -- (LSTM-input);
	}
	
	\visible<5->{
	\draw[endflow] (LSTM-coutput)--(ct);
	}
	
	
	\visible<5->{
	\draw[endflow] (ctminus1)--(LSTMfmult);
	}
	
	\visible<6->{
	\draw[endflow] (LSTM-houtput)--(ht);
	}
	\end{tikzpicture}
}

\newcommand{\lstmanim}{
	\begin{tikzpicture}[scale=1, node distance=2em]%,show background rectangle,background rectangle/.style={draw=red}]
	
	
	\draw pic (LSTM) at (0,0) {lstmanim};
	\node[io,xshift=1ex,above=3em of LSTMtl, ,label=above:$\VInput_{t}$](xt){\animategraphics[poster=25,width=1cm,autoplay,loop]{\fps}{images/activations/16494/x/x-}{1}{36}};%$x_{t}$
	\draw[rounded corners] (xt) |- (LSTM-input);
	\node[io,left=of LSTMtl,label=below:$\VHidden_{t-1}$](htminus1){
		\animategraphics[poster=24,width=1cm,autoplay,loop]{\fps}{images/activations/16494/output/3-}{0}{35}
	};
	\draw[endflow] (htminus1) -- (LSTM-input);
	\node[io,right=of LSTMbr,label=above:$\VCellState_{t}$](ct){\animategraphics[poster=25,width=1cm,autoplay,loop]{\fps}{images/activations/16494/state/3-}{1}{36}}; % $c_{t}$
	\draw[endflow] (LSTM-coutput)--(ct);
	\node[io,left=of LSTMbl,label=above:$\VCellState_{t-1}$](ctminus1){\animategraphics[poster=24,width=1cm,autoplay,loop]{\fps}{images/activations/16494/state/3-}{0}{35}}; % 
	\draw[endflow] (ctminus1)--(LSTMfmult);
	\node[io,right=of LSTMtr,label=below:$\VHidden_{t}$](ht){
		\animategraphics[poster=24,width=1cm,autoplay,loop]{\fps}{images/activations/16494/output/3-}{1}{36}
	};
	\draw[endflow] (LSTM-houtput)--(ht);
	
	\draw[endflow] (ct) -- ($ (ct)+(0,-0.8) $) -| (ctminus1);
	\draw[endflow] (ht) -- ($ (ht)+(0,.8) $) -| (htminus1);
	
	\end{tikzpicture}
}

\newcommand{\rnn}{
	\begin{tikzpicture}[scale=1, node distance=2em]%,show background rectangle,background rectangle/.style={draw=red}]
	
	
	\draw pic (RNN) at (0,0) {simplernn};
	\node[io,xshift=1ex,above=3em of RNNtl](xt){
		$\VInput_{t}$
%		\animategraphics[poster=25,width=1cm,autoplay,loop]{\fps}{images/activations/16494/x/x-}{1}{36}
	};%$x_{t}$
	\draw[rounded corners] (xt) |- (RNN-input);
	\node[io,left=of RNNtl](htminus1){
		$\VHidden_{t-1}$
%		\animategraphics[poster=24,width=1cm,autoplay,loop]{\fps}{images/activations/16494/output/3-}{0}{35}
	};
	\draw[flow] (htminus1) -- (RNN-input);
%	\node[io,right=of LSTMbr,label=above:$\VCellState_{t}$](ct){
%		\animategraphics[poster=25,width=1cm,autoplay,loop]{\fps}{images/activations/16494/state/3-}{1}{36}
%	}; % $c_{t}$
%	\draw[endflow] (LSTM-coutput)--(ct);
%	\node[io,left=of LSTMbl,label=above:$\VCellState_{t-1}$](ctminus1){
%		\animategraphics[poster=24,width=1cm,autoplay,loop]{\fps}{images/activations/16494/state/3-}{0}{35}
%	}; % 
%	\draw[endflow] (ctminus1)--(LSTMfmult);
	\node[io,right=of RNNtr](ht){
		$\VHidden_{t}$
%		\animategraphics[poster=24,width=1cm,autoplay,loop]{\fps}{images/activations/16494/output/3-}{1}{36}
	};
	\draw[endflow] (RNN-houtput)--(ht);
	
%	\draw[endflow] (ct) -- ($ (ct)+(0,-0.8) $) -| (ctminus1);
	\draw[endflow] (ht) -- ($ (ht)+(0,-.8) $) -| (htminus1);
	
	\end{tikzpicture}
}
%
%\begin{frame}[t]
%\frametitle{Extracting features from noisy data with ConvRNNs}
%
%\centering
%%\lstmanim
%\begin{tikzpicture}[scale=1, node distance=2em]%,show background rectangle,background rectangle/.style={draw=red}]
%
%
%\draw pic (LSTM) at (0,0) {lstmanim};
%\node[io,xshift=1ex,above=3em of LSTMtl, ,label=above:$\VInput_{t}$](xt){\animategraphics[poster=25,width=1cm,autoplay,loop]{\fps}{images/activations/16494/x/x-}{1}{36}};%$x_{t}$
%\draw[rounded corners] (xt) |- (LSTM-input);
%\node[io,left=of LSTMtl,label=below:$\VHidden_{t-1}$](htminus1){
%	\animategraphics[poster=24,width=1cm,autoplay,loop]{\fps}{images/activations/16494/output/3-}{0}{35}
%};
%\draw[endflow] (htminus1) -- (LSTM-input);
%\node[io,right=of LSTMbr,label=above:$\VCellState_{t}$](ct){\animategraphics[poster=25,width=1cm,autoplay,loop]{\fps}{images/activations/16494/state/3-}{1}{36}}; % $c_{t}$
%\draw[endflow] (LSTM-coutput)--(ct);
%\node[io,left=of LSTMbl,label=above:$\VCellState_{t-1}$](ctminus1){\animategraphics[poster=24,width=1cm,autoplay,loop]{\fps}{images/activations/16494/state/3-}{0}{35}}; % 
%\draw[endflow] (ctminus1)--(LSTMfmult);
%\node[io,right=of LSTMtr,label=below:$\VHidden_{t}$](ht){
%	\animategraphics[poster=24,width=1cm,autoplay,loop]{\fps}{images/activations/16494/output/3-}{1}{36}
%%};
%\draw[endflow] (LSTM-houtput)--(ht);
%
%\draw[endflow] (ct) -- ($ (ct)+(0,-0.8) $) -| (ctminus1);
%\draw[endflow] (ht) -- ($ (ht)+(0,.8) $) -| (htminus1);
%
%\end{tikzpicture}
%
%\end{frame}


{\setbeamercolor{background canvas}{bg=black}
	\begin{frame}[plain]
	\vfill
	\begin{columns}
		\column{.5\textwidth}
		\color{tumwhite}
		
		\Huge
		\visible<2->{\color{tumgray} Self-}{\color{white}Attention} \visible<2->{in \\ Deep Learning}
		
		\column{.5\textwidth}
		
		\includegraphics[width=6cm]{images/goldfish_zhengtaoTang}
	\end{columns}
	\begin{center}
		\Huge\color{tumwhite}
		\vfill\raggedleft
		{\small \color{tumgray} Photo by zhengtao tang on Unsplash}
		
	\end{center}
	\vfill
\end{frame}
}



\begin{frame}
	\only<1-4>{\frametitle{Attention}}
	\only<5>{\frametitle{Self-Attention}}
	
	\colorlet{querycolor}{tumorange}
	\colorlet{attentioncolor}{tumred}
	\colorlet{valuecolor}{tumblue}
	\colorlet{attentionoutcolor}{tumbluedark}
	\colorlet{keycolor}{tumgreen}
	
	\begin{columns}
		\column{.5\textwidth}
		
		\only<1>{
			\begin{itemize}
				\item let's assume we have a sequence of observations.
				\item each observation is differently important given one objective (defined by query).
				\item we want an output focused only on the important observations
			\end{itemize}
			
			
		}
		\visible<2->{
\newcommand{\attnquery}{%
	\only<3>{%
		\begin{tikzpicture}[scale=0.6]
		\node[draw=querycolor, circle, fill=querycolor, fill opacity=.2, text opacity=1, font=\small, inner sep=.2em](a) at (0, 0){.2};
		\node[draw=querycolor, circle, fill=querycolor, fill opacity=.1, text opacity=1, font=\small, inner sep=.2em](b) at (1, 0){.1};
		\end{tikzpicture}
	}%
	\only<4->{%
		\begin{tikzpicture}[scale=0.4]
		\node[draw=querycolor, circle, fill=querycolor, fill opacity=.2, text opacity=1, font=\small, inner sep=.2em](a) at (0, 0){};
		\node[draw=querycolor, circle, fill=querycolor, fill opacity=.2, text opacity=1, font=\small, inner sep=.2em](a) at (0, 1){};
		\node[draw=querycolor, circle, fill=querycolor, fill opacity=.2, text opacity=1, font=\small, inner sep=.2em](a) at (0, 2){};
		\node[draw=querycolor, circle, fill=querycolor, fill opacity=.2, text opacity=1, font=\small, inner sep=.2em](a) at (0, 3){};
		
		\node[draw=querycolor, circle, fill=querycolor, fill opacity=.1, text opacity=1, font=\small, inner sep=.2em](b) at (1, 0){};
		\node[draw=querycolor, circle, fill=querycolor, fill opacity=.1, text opacity=1, font=\small, inner sep=.2em](b) at (1, 1){};
		\node[draw=querycolor, circle, fill=querycolor, fill opacity=.1, text opacity=1, font=\small, inner sep=.2em](b) at (1, 2){};
		\node[draw=querycolor, circle, fill=querycolor, fill opacity=.1, text opacity=1, font=\small, inner sep=.2em](b) at (1, 3){};
		
		\node[draw=querycolor, circle, fill=querycolor, fill opacity=.1, text opacity=1, font=\small, inner sep=.2em](b) at (2, 0){};
		\node[draw=querycolor, circle, fill=querycolor, fill opacity=.1, text opacity=1, font=\small, inner sep=.2em](b) at (2, 1){};
		\node[draw=querycolor, circle, fill=querycolor, fill opacity=.1, text opacity=1, font=\small, inner sep=.2em](b) at (2, 2){};
		\node[draw=querycolor, circle, fill=querycolor, fill opacity=.1, text opacity=1, font=\small, inner sep=.2em](b) at (2, 3){};
		
		\node[draw=querycolor, circle, fill=querycolor, fill opacity=.1, text opacity=1, font=\small, inner sep=.2em](b) at (3, 0){};
		\node[draw=querycolor, circle, fill=querycolor, fill opacity=.1, text opacity=1, font=\small, inner sep=.2em](b) at (3, 1){};
		\node[draw=querycolor, circle, fill=querycolor, fill opacity=.1, text opacity=1, font=\small, inner sep=.2em](b) at (3, 2){};
		\node[draw=querycolor, circle, fill=querycolor, fill opacity=.1, text opacity=1, font=\small, inner sep=.2em](b) at (3, 3){};
		\end{tikzpicture}
	}%
}


\newcommand{\attention}{%
	\only<2,3>{%
		\begin{tikzpicture}[scale=0.6]
		\node[draw=attentioncolor, circle, fill=attentioncolor, fill opacity=.4, text opacity=1, font=\small, inner sep=.2em](a) at (1,0){.2};
		\node[draw=attentioncolor, circle, fill=attentioncolor, fill opacity=.8, text opacity=1, font=\small, inner sep=.2em](b) at (2,0){.4};
		\node[draw=attentioncolor, circle, fill=attentioncolor, fill opacity=.2, text opacity=1, font=\small, inner sep=.2em](c) at (3,0){.1};
		\node[draw=attentioncolor, circle, fill=attentioncolor, fill opacity=.6, text opacity=1, font=\small, inner sep=.2em](d) at (4,0){.3};
		\end{tikzpicture}
	}%
	\only<4->{%
		\begin{tikzpicture}[scale=0.4]
		\node[draw=attentioncolor, circle, fill=attentioncolor, fill opacity=.4, text opacity=1, font=\small, inner sep=.2em](a) at (1,0){};
		\node[draw=attentioncolor, circle, fill=attentioncolor, fill opacity=.8, text opacity=1, font=\small, inner sep=.2em](b) at (2,0){};
		\node[draw=attentioncolor, circle, fill=attentioncolor, fill opacity=.2, text opacity=1, font=\small, inner sep=.2em](c) at (3,0){};
		\node[draw=attentioncolor, circle, fill=attentioncolor, fill opacity=.6, text opacity=1, font=\small, inner sep=.2em](d) at (4,0){};
		
		\node[draw=attentioncolor, circle, fill=attentioncolor, fill opacity=.4, text opacity=1, font=\small, inner sep=.2em](a) at (1,1){};
		\node[draw=attentioncolor, circle, fill=attentioncolor, fill opacity=.8, text opacity=1, font=\small, inner sep=.2em](b) at (2,1){};
		\node[draw=attentioncolor, circle, fill=attentioncolor, fill opacity=.2, text opacity=1, font=\small, inner sep=.2em](c) at (3,1){};
		\node[draw=attentioncolor, circle, fill=attentioncolor, fill opacity=.6, text opacity=1, font=\small, inner sep=.2em](d) at (4,1){};
		
		\node[draw=attentioncolor, circle, fill=attentioncolor, fill opacity=.4, text opacity=1, font=\small, inner sep=.2em](a) at (1,2){};
		\node[draw=attentioncolor, circle, fill=attentioncolor, fill opacity=.8, text opacity=1, font=\small, inner sep=.2em](b) at (2,2){};
		\node[draw=attentioncolor, circle, fill=attentioncolor, fill opacity=.2, text opacity=1, font=\small, inner sep=.2em](c) at (3,2){};
		\node[draw=attentioncolor, circle, fill=attentioncolor, fill opacity=.6, text opacity=1, font=\small, inner sep=.2em](d) at (4,2){};
		
		\node[draw=attentioncolor, circle, fill=attentioncolor, fill opacity=.4, text opacity=1, font=\small, inner sep=.2em](a) at (1,3){};
		\node[draw=attentioncolor, circle, fill=attentioncolor, fill opacity=.8, text opacity=1, font=\small, inner sep=.2em](b) at (2,3){};
		\node[draw=attentioncolor, circle, fill=attentioncolor, fill opacity=.2, text opacity=1, font=\small, inner sep=.2em](c) at (3,3){};
		\node[draw=attentioncolor, circle, fill=attentioncolor, fill opacity=.6, text opacity=1, font=\small, inner sep=.2em](d) at (4,3){};
		\end{tikzpicture}
	}%
}

\newcommand{\attnv}{%
	\only<2,3>{%
		\begin{tikzpicture}[scale=0.6]
		\node[draw=valuecolor, circle, fill=valuecolor, fill opacity=.2, text opacity=1, font=\small, inner sep=.2em](a) at (0, 1){.2};
		\node[draw=valuecolor, circle, fill=valuecolor, fill opacity=.1, text opacity=1, font=\small, inner sep=.2em](b) at (0, 2){.1};
		\node[draw=valuecolor, circle, fill=valuecolor, fill opacity=.3, text opacity=1, font=\small, inner sep=.2em](c) at (0, 3){.3};
		\node[draw=valuecolor, circle, fill=valuecolor, fill opacity=.4, text opacity=1, font=\small, inner sep=.2em](d) at (0, 4){.4};
		\end{tikzpicture}
	}%
	\only<4->{%
		\begin{tikzpicture}[scale=0.4]
		\node[draw=valuecolor, circle, fill=valuecolor, fill opacity=.2, text opacity=1, font=\small, inner sep=.2em](a) at (0, 1){};
		\node[draw=valuecolor, circle, fill=valuecolor, fill opacity=.1, text opacity=1, font=\small, inner sep=.2em](b) at (0, 2){};
		\node[draw=valuecolor, circle, fill=valuecolor, fill opacity=.3, text opacity=1, font=\small, inner sep=.2em](c) at (0, 3){};
		\node[draw=valuecolor, circle, fill=valuecolor, fill opacity=.4, text opacity=1, font=\small, inner sep=.2em](d) at (0, 4){};
		
		\node[draw=valuecolor, circle, fill=valuecolor, fill opacity=.2, text opacity=1, font=\small, inner sep=.2em](a) at (1, 1){};
		\node[draw=valuecolor, circle, fill=valuecolor, fill opacity=.1, text opacity=1, font=\small, inner sep=.2em](b) at (1, 2){};
		\node[draw=valuecolor, circle, fill=valuecolor, fill opacity=.3, text opacity=1, font=\small, inner sep=.2em](c) at (1, 3){};
		\node[draw=valuecolor, circle, fill=valuecolor, fill opacity=.4, text opacity=1, font=\small, inner sep=.2em](d) at (1, 4){};
		\end{tikzpicture}
	}%
}

\newcommand{\attnout}{%
	\only<2,3>{%
		\begin{tikzpicture}[scale=0.6]
		\node[draw=tumblack, circle, fill=attentionoutcolor, text=white, text opacity=1, font=\small, inner sep=.2em](d) at (0,0){.27};
		\end{tikzpicture}
	}	
	\only<4->{%
		\begin{tikzpicture}[scale=0.4]
		\node[draw=tumblack, circle, fill=attentionoutcolor, text=white, text opacity=1, font=\small, inner sep=.2em](d) at (0,0){};
		\node[draw=tumblack, circle, fill=attentionoutcolor, text=white, text opacity=1, font=\small, inner sep=.2em](d) at (0,1){};
		\node[draw=tumblack, circle, fill=attentionoutcolor, text=white, text opacity=1, font=\small, inner sep=.2em](d) at (0,2){};
		\node[draw=tumblack, circle, fill=attentionoutcolor, text=white, text opacity=1, font=\small, inner sep=.2em](d) at (0,3){};
		
		\node[draw=tumblack, circle, fill=attentionoutcolor, text=white, text opacity=1, font=\small, inner sep=.2em](d) at (1,0){};
		\node[draw=tumblack, circle, fill=attentionoutcolor, text=white, text opacity=1, font=\small, inner sep=.2em](d) at (1,1){};
		\node[draw=tumblack, circle, fill=attentionoutcolor, text=white, text opacity=1, font=\small, inner sep=.2em](d) at (1,2){};
		\node[draw=tumblack, circle, fill=attentionoutcolor, text=white, text opacity=1, font=\small, inner sep=.2em](d) at (1,3){};
		\end{tikzpicture}
	}%
}

\newcommand{\attnkey}{
	\only<3>{%
		\begin{tikzpicture}[scale=0.6]
		\node[draw=keycolor, circle, fill=keycolor, fill opacity=.2, text opacity=1, font=\small, inner sep=.2em](a) at (0, 0){.2};
		\node[draw=keycolor, circle, fill=keycolor, fill opacity=.1, text opacity=1, font=\small, inner sep=.2em](b) at (1, 0){.1};
		\node[draw=keycolor, circle, fill=keycolor, fill opacity=.3, text opacity=1, font=\small, inner sep=.2em](c) at (2, 0){.3};
		\node[draw=keycolor, circle, fill=keycolor, fill opacity=.4, text opacity=1, font=\small, inner sep=.2em](d) at (3, 0){.4};
		
		\node[draw=keycolor, circle, fill=keycolor, fill opacity=.2, text opacity=1, font=\small, inner sep=.2em](a) at (0, 1){.2};
		\node[draw=keycolor, circle, fill=keycolor, fill opacity=.1, text opacity=1, font=\small, inner sep=.2em](b) at (1, 1){.1};
		\node[draw=keycolor, circle, fill=keycolor, fill opacity=.3, text opacity=1, font=\small, inner sep=.2em](c) at (2, 1){.3};
		\node[draw=keycolor, circle, fill=keycolor, fill opacity=.4, text opacity=1, font=\small, inner sep=.2em](d) at (3, 1){.4};
		\end{tikzpicture}
	}
	\only<4->{%
		\begin{tikzpicture}[scale=0.4]
		\node[draw=keycolor, circle, fill=keycolor, fill opacity=.2, text opacity=1, font=\small, inner sep=.2em](a) at (0, 0){};
		\node[draw=keycolor, circle, fill=keycolor, fill opacity=.1, text opacity=1, font=\small, inner sep=.2em](b) at (1, 0){};
		\node[draw=keycolor, circle, fill=keycolor, fill opacity=.3, text opacity=1, font=\small, inner sep=.2em](c) at (2, 0){};
		\node[draw=keycolor, circle, fill=keycolor, fill opacity=.4, text opacity=1, font=\small, inner sep=.2em](d) at (3, 0){};
		
		\node[draw=keycolor, circle, fill=keycolor, fill opacity=.2, text opacity=1, font=\small, inner sep=.2em](a) at (0, 1){};
		\node[draw=keycolor, circle, fill=keycolor, fill opacity=.1, text opacity=1, font=\small, inner sep=.2em](b) at (1, 1){};
		\node[draw=keycolor, circle, fill=keycolor, fill opacity=.3, text opacity=1, font=\small, inner sep=.2em](c) at (2, 1){};
		\node[draw=keycolor, circle, fill=keycolor, fill opacity=.4, text opacity=1, font=\small, inner sep=.2em](d) at (3, 1){};
		
		\node[draw=keycolor, circle, fill=keycolor, fill opacity=.2, text opacity=1, font=\small, inner sep=.2em](a) at (0, 2){};
		\node[draw=keycolor, circle, fill=keycolor, fill opacity=.1, text opacity=1, font=\small, inner sep=.2em](b) at (1, 2){};
		\node[draw=keycolor, circle, fill=keycolor, fill opacity=.3, text opacity=1, font=\small, inner sep=.2em](c) at (2, 2){};
		\node[draw=keycolor, circle, fill=keycolor, fill opacity=.4, text opacity=1, font=\small, inner sep=.2em](d) at (3, 2){};
		
		\node[draw=keycolor, circle, fill=keycolor, fill opacity=.2, text opacity=1, font=\small, inner sep=.2em](a) at (0, 3){};
		\node[draw=keycolor, circle, fill=keycolor, fill opacity=.1, text opacity=1, font=\small, inner sep=.2em](b) at (1, 3){};
		\node[draw=keycolor, circle, fill=keycolor, fill opacity=.3, text opacity=1, font=\small, inner sep=.2em](c) at (2, 3){};
		\node[draw=keycolor, circle, fill=keycolor, fill opacity=.4, text opacity=1, font=\small, inner sep=.2em](d) at (3, 3){};
		\end{tikzpicture}
	}%
}

\begin{tikzpicture}[node distance=.2em]
\visible<2->{
\node[label={below:$\V{\alpha}^T$}, draw=attentioncolor, rounded corners](alpha){\attention};
\node[right=of alpha](out){\attnout};
\node[above=of out, label={above:${\only<2,3>{\V{v}}\only<4>{\M{V}}}$}, draw=valuecolor, rounded corners](v){\attnv};
}
\visible<3->{
\node[above=of alpha, label={above:$\M{K}$}, draw=keycolor, rounded corners](k){\attnkey};
\node[left=of alpha, label={below:${\only<3>{\V{q}}\only<4->{\M{Q}}}^T$}, draw=querycolor, rounded corners](q){\attnquery};
}
\visible<5>{
\node[fill=white, text=black, fill opacity=0.5, text opacity=1, rounded corners] at (v){$\M{X}\Mweight_V$};
\node[fill=white, text=black, fill opacity=0.5, text opacity=1, rounded corners] at (k){$\M{X}\Mweight_K$};
\node[fill=white, text=black, fill opacity=0.5, text opacity=1, rounded corners] at (q){$\M{X}\Mweight_Q$};
}
\end{tikzpicture}}
		
		
		%	\begin{equation*}
		%		\text{Attention}(Q,K,V) = 
		%		\begin{tikzpicture}
		%		\node(alpha){$\underbrace{\attention}_\alpha$};
		%		\node[right=of alpha](out){};
		%		\node[above=of out]{\attnv};
		%		\end{tikzpicture}
		%		 
		%	\end{equation*}
		
		
		\column{.5\textwidth}
		
		\begin{tikzpicture}[yscale=3]
			
			\visible<4->{
				\node[draw=valuecolor, circle, fill=valuecolor, fill opacity=.2, text opacity=1, font=\small, inner sep=.2em](d) at (1,.35){};
				\node[draw=valuecolor, circle, fill=valuecolor, fill opacity=.1, text opacity=1, font=\small, inner sep=.2em](e) at (2,.05){};
				\node[draw=valuecolor, circle, fill=valuecolor, fill opacity=.3, text opacity=1, font=\small, inner sep=.2em](f) at (3,.42){};
				\node[draw=valuecolor, circle, fill=valuecolor, fill opacity=.4, text opacity=1, font=\small, inner sep=.2em](g) at (4,.25){};
				\draw (d) -- (e) -- (f) -- (g);
			}
			
			\node[draw=valuecolor, circle, fill=valuecolor, fill opacity=.2, text opacity=1, font=\small, inner sep=.2em](a) at (1,.3){};
			\node[draw=valuecolor, circle, fill=valuecolor, fill opacity=.1, text opacity=1, font=\small, inner sep=.2em](b) at (2,.1){};
			\node[draw=valuecolor, circle, fill=valuecolor, fill opacity=.3, text opacity=1, font=\small, inner sep=.2em](c) at (3,.3){};
			\node[draw=valuecolor, circle, fill=valuecolor, fill opacity=.4, text opacity=1, font=\small, inner sep=.2em](d) at (4,.4){};
			
			\draw (a) -- (b) -- (c) -- (d);
			
			\foreach \t in {1,2,3,4} {
				\draw[tumgray] (\t,0) -- (\t,-.05) node[at end, below, text=tumgray] {$t_\t$};
			}
			
			%\draw[-stealth] (0,0) -- (0,.5);
			\draw[-stealth, tumgray] (.5,0) -- (4.5,0);
			
			\only<2,3>{
			\node[draw=tumblack, circle, fill=attentionoutcolor, text=white, fill opacity=1, text opacity=1, font=\small, inner sep=.2em, label={right:$\sum_{t=0}^{T} \alpha_t v_t = \V{\alpha}^T \V{v}$}](out) at (2.5,-.5) {};
			}
			\only<4->{
			\node[draw=white, circle, fill=attentionoutcolor, text=white, fill opacity=1, text opacity=1, font=\small, inner sep=.2em](outba) at (1.05,-.48) {};
			\node[draw=white, circle, fill=attentionoutcolor, text=white, fill opacity=1, text opacity=1, font=\small, inner sep=.2em](outbb) at (2.05,-.48) {};
			\node[draw=white, circle, fill=attentionoutcolor, text=white, fill opacity=1, text opacity=1, font=\small, inner sep=.2em](outbc) at (3.05,-.48) {};
			\node[draw=white, circle, fill=attentionoutcolor, text=white, fill opacity=1, text opacity=1, font=\small, inner sep=.2em](outbd) at (4.05,-.48) {};
				
			\node[draw=white, circle, fill=attentionoutcolor, text=white, fill opacity=1, text opacity=1, font=\small, inner sep=.2em](outa) at (1,-.5) {};
			\node[draw=white, circle, fill=attentionoutcolor, text=white, fill opacity=1, text opacity=1, font=\small, inner sep=.2em](out) at (2,-.5) {};
			\node[draw=white, circle, fill=attentionoutcolor, text=white, fill opacity=1, text opacity=1, font=\small, inner sep=.2em](outa) at (3,-.5) {};
			\node[draw=white, circle, fill=attentionoutcolor, text=white, fill opacity=1, text opacity=1, font=\small, inner sep=.2em](outa) at (4,-.5) {};

			}
		
			\node at (2.5, -.25) {};
			
			\visible<2->{
			\draw[-stealth, draw=tumred, opacity=.4, line width=.4] (a) -- (out);
			\draw[-stealth, draw=tumred, opacity=.8, line width=.8] (b) -- (out);
			\draw[-stealth, draw=tumred, opacity=.2, line width=.2] (c) -- (out);
			\draw[-stealth, draw=tumred, opacity=.6, line width=.6] (d) -- (out);
			}
			
			\visible<1>{
				\node(annot1) at (5.5,.1){keep those};
				\draw[-stealth, tumred, shorten <= .3em, , shorten >= .3em](annot1) -- (b);
				\draw[-stealth, tumred, shorten <= .3em, , shorten >= .3em](annot1) -- (d);
				
				\node(annot2) at (3.5,.85){ignore these};
				\draw[-stealth, tumbluelight, shorten <= 1em, shorten >= 1em](annot2) -- (a);
				\draw[-stealth, tumbluelight, shorten <= 1em, shorten >= 1em](annot2) -- (c);
			}
			
		\end{tikzpicture}
%		

	\end{columns}

	
	\Large
	\only<2>{
	\begin{equation*}
	\text{Attention}({\color{tumred}\V{\alpha}}, {\color{tumblue}\V{v}}) = {\color{tumred}\V{\alpha}}^T  {\color{tumblue}\V{v}} = \sum_{t=0}^{T} \alpha_tv_t, \quad \V{\alpha} \in [0,1]^{T=4}, \V{v} \in \mathbb{R}^{T}
	\end{equation*}
	}
	\only<3>{
		\begin{equation*}
		\text{Attention}({\color{tumorange}\V{K}}, {\color{tumgreen}\V{q}}, {\color{tumblue}\V{v}}) = 
		\overbrace{\text{softmax}\left({\color{tumgreen}\V{q}^T}{\color{tumorange}\V{K}}\right)}^{{\color{tumred}\V{\alpha}}^T}
		{\color{tumblue}\V{v}}, \quad \V{v} \in \mathbb{R}^{T}, \V{q} \in \mathbb{R}^{D_k}, \M{K} \in \mathbb{R}^{D_k \times T}
		\end{equation*}
	}
	\only<4>{
		\begin{equation*}
		\text{Attention}({\color{tumorange}\V{K}}, {\color{tumgreen}\V{Q}}, {\color{tumblue}\V{V}}) = 
		\text{softmax}\left({\color{tumgreen}\V{Q}^T}{\color{tumorange}\V{K}}\right)
		{\color{tumblue}\V{V}}, \quad \V{V} \in \mathbb{R}^{T \times D_v}, \V{Q} \in \mathbb{R}^{D_k \times T}, \M{K} \in \mathbb{R}^{D_k \times T}
		\end{equation*}
	}
	\only<5>{
		\begin{equation*}
		\text{Self-Attention}_\Mweight(\M{X}) = \text{Attention}(\M{X}\Mweight_K, \M{X}\Mweight_Q, \M{X}\Mweight_V) = \text{softmax}\left(\left(\M{X}\Mweight_Q\right)\left(\M{X}\Mweight_K\right)\right)\left(\M{X}\Mweight_V\right)
		\end{equation*}
	}
\end{frame}

\begin{frame}
	\frametitle{Visualize the Attention Matrix as Adjacency Matrix}
	\begin{tikzpicture}
		\node[inner sep=0](alpha){\includegraphics[width=5cm]{images/self-attention/transformer/self-attention-1/head0_imshow}};
		\node[left=0em of alpha]{$T_\text{in}$};
		\node[right=of alpha](colorbar){\includegraphics[height=5cm]{images/self-attention/colorbar}};
		\node[below=0em of alpha]{$T_\text{out}$};
		
	\end{tikzpicture}
	
%	\includegraphics[width=4cm]{images/self-attention/transformer/self-attention-1/head1_imshow}
%	\includegraphics[width=4cm]{images/self-attention/transformer/self-attention-1/head2_imshow}
	
\end{frame}

\begin{frame}

\frametitle{Visualize the Attention Matrix as Bipartite Graph}
	\begin{tikzpicture}
		\node[inner sep=0](alpha){\includegraphics[width=\textwidth]{images/self-attention/transformer/self-attention-1/head0_conn}};
		\node[above=0em of alpha]{$T_\text{in}$};
		\node[below=0em of alpha]{$T_\text{out}$};
	\end{tikzpicture}
	
\end{frame}

\begin{frame}
	\includegraphics[width=\textwidth]{images/self-attention}
\end{frame}

\begin{frame}
	
	
	\begin{tikzpicture}[baseline=-2em, inner sep=0]
	
	\def\path{images/self-attention/transformer}
	
	\begin{axis}[
	thin,
	width=6cm,
	hide axis,
	height=3cm,
	ymin=0, ymax=1.4,
	no marks,  
	draw opacity=.8,
	smooth=0.01
	]
	
%	\addplot table [x index = 0, y index = 0, col sep=comma] {\path/self-attention-1/enc-input.txt};
%	
	
	\end{axis}
	
	\end{tikzpicture}
	
\end{frame}


%
\begin{frame}[c]
\frametitle{Publications and Code}
\centering 

\Large



Github + DockerHub

\vspace{1ex}

\includegraphics[width=2cm]{images/github} \hspace{.5ex}
\includegraphics[width=2cm]{images/qr_github} \hspace{.5ex}
\includegraphics[width=2cm]{images/docker}

\vspace{1ex}

\url{https://github.com/TUM-LMF/MTLCC}

\url{https://github.com/TUM-LMF/MTLCC-pytorch}

\vspace{1em}
\small
\textsl{
	Rußwurm M., Körner M. (2018). \textbf{Multi-Temporal Land Cover Classification with Sequential Recurrent Encoders}. ISPRS International Journal of Geo-Information. https://arxiv.org/abs/1802.02080.
}
	
\end{frame}


\begin{frame}
\frametitle{Cloud coverage as spatiotemporal noise}
\centering

\def\imagewidth{1.5cm}


\visible<1>{\includegraphics[width=\imagewidth]{images/activations/16494/x/x-0.png}}
\visible<1>{\includegraphics[width=\imagewidth]{images/activations/16494/x/x-1.png}}
\visible<1>{\includegraphics[width=\imagewidth]{images/activations/16494/x/x-2.png}}
\visible<1>{\includegraphics[width=\imagewidth]{images/activations/16494/x/x-3.png}}
\visible<1>{\includegraphics[width=\imagewidth]{images/activations/16494/x/x-4.png}}
\visible<1,2>{\includegraphics[width=\imagewidth]{images/activations/16494/x/x-5.png}}
\visible<1>{\includegraphics[width=\imagewidth]{images/activations/16494/x/x-6.png}}
\visible<1>{\includegraphics[width=\imagewidth]{images/activations/16494/x/x-7.png}}
\visible<1,2>{\includegraphics[width=\imagewidth]{images/activations/16494/x/x-8.png}}
\visible<1>{\includegraphics[width=\imagewidth]{images/activations/16494/x/x-9.png}}
\visible<1>{\includegraphics[width=\imagewidth]{images/activations/16494/x/x-10.png}}
\visible<1>{\includegraphics[width=\imagewidth]{images/activations/16494/x/x-11.png}}
\visible<1,2>{\includegraphics[width=\imagewidth]{images/activations/16494/x/x-12.png}}
\visible<1,2>{\includegraphics[width=\imagewidth]{images/activations/16494/x/x-13.png}}
\visible<1,2>{\includegraphics[width=\imagewidth]{images/activations/16494/x/x-14.png}}
\visible<1,2>{\includegraphics[width=\imagewidth]{images/activations/16494/x/x-15.png}}
\visible<1,2>{\includegraphics[width=\imagewidth]{images/activations/16494/x/x-16.png}}
\visible<1>{\includegraphics[width=\imagewidth]{images/activations/16494/x/x-18.png}}
\visible<1>{\includegraphics[width=\imagewidth]{images/activations/16494/x/x-19.png}}
\visible<1,2>{\includegraphics[width=\imagewidth]{images/activations/16494/x/x-20.png}}
\visible<1,2>{\includegraphics[width=\imagewidth]{images/activations/16494/x/x-21.png}}
\visible<1>{\includegraphics[width=\imagewidth]{images/activations/16494/x/x-22.png}}
\visible<1>{\includegraphics[width=\imagewidth]{images/activations/16494/x/x-23.png}}
\visible<1>{\includegraphics[width=\imagewidth]{images/activations/16494/x/x-24.png}}
\visible<1>{\includegraphics[width=\imagewidth]{images/activations/16494/x/x-25.png}}
\visible<1>{\includegraphics[width=\imagewidth]{images/activations/16494/x/x-26.png}}
\visible<1,2>{\includegraphics[width=\imagewidth]{images/activations/16494/x/x-27.png}}
\visible<1>{\includegraphics[width=\imagewidth]{images/activations/16494/x/x-28.png}}
\visible<1,2>{\includegraphics[width=\imagewidth]{images/activations/16494/x/x-29.png}}
\visible<1>{\includegraphics[width=\imagewidth]{images/activations/16494/x/x-30.png}}
\visible<1>{\includegraphics[width=\imagewidth]{images/activations/16494/x/x-31.png}}
\visible<1,2>{\includegraphics[width=\imagewidth]{images/activations/16494/x/x-32.png}}
\visible<1>{\includegraphics[width=\imagewidth]{images/activations/16494/x/x-33.png}}
%	
\end{frame}


\begin{frame}
\frametitle{Common Filtering/Preprocessing Algorithms work quite well}

\begin{columns}
	\column{.5\textwidth}
	\begin{itemize}
		\item FMask
		\item MAJA
		\item Sen2Cor (F-Mask)
		\item supervised cloud classification e.g., CNNs
		\item unsupervised cloud clustering? {\small Go FDL!}
	\end{itemize}
	\column{.5\textwidth}
	
\end{columns}
\end{frame}


\begin{frame}
\frametitle{test}

\tikzstyle{conn} = [-stealth, rounded corners, tumbluedark, thick]
\tikzstyle{module} = [draw=none, fill=tumgraylight, rounded corners]
\tikzstyle{layer} = [draw=none, fill=tumbluelight, rounded corners]

\visible<2>{
	\begin{tikzpicture}[node distance=2em]
	\node(input){inputs};
	\node[below of=input](plus){+};
	\node[left of=plus](posenc){p};
	
	\node[module, below=2em of plus](attn){Multi-Head-Attention};
	\node[module, below=.5em of attn](addnorm){Add \& Norm};
	
	\node[module, below=2em of addnorm](ff){Feed Forward};
	\node[module, below=.5em of ff](addnorm2){Add \& Norm};
	
	
	\coordinate[above=of attn](in);
	\coordinate[below=of addnorm2](out);
	\draw[conn] (in) -- (attn);
	\draw[conn] (in) -- ($ (attn)!.5!(in) $) -| ($ (attn.north)+(2em,0) $);
	\draw[conn] (in) -- ($ (attn)!.5!(in) $) -| ($ (attn.north)-(2em,0) $);  
	%		
	\draw[conn] (in) -- ($ (attn)!.6!(in) $) -| ($ (attn.east)+(.5em,0) $) |- (addnorm.east);
	
	\draw[conn] (attn) -- (addnorm);
	
	\draw[conn] (addnorm) -- (ff);
	\draw[conn] (addnorm) -- ($ (addnorm)!.5!(ff) $) -| ($ (ff.east)+(.5em,0) $) |- (addnorm2.east);
	\draw[conn] (ff) -- (addnorm2);
	\draw[conn] (addnorm2) -- (out);
	
	\begin{pgfonlayer}{background}
	\node[layer, draw=black, fit=(attn)(addnorm)(ff)(addnorm2)(in)(out)]{};
	\end{pgfonlayer}
	
	\end{tikzpicture}
}

\end{frame}

\begin{frame}
	\frametitle{Conclusion}
	We looked at two mechanisms how deep neural networks learn to deal with noise in the data.
	We raised the question if extensive region-specific preprocessing is necessary in the context of end-to-end learning
	We provide source code and Notebooks to examples at XXX
\end{frame}


\end{document}


