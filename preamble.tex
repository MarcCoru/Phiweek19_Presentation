

\usepackage{times}
\usepackage{epsfig}
\usepackage{graphicx}
\usepackage{amsmath}
\usepackage{amssymb}
\usepackage[utf8]{inputenc}
\usepackage{booktabs}
\setlength{\tabcolsep}{5pt}
\usepackage{subcaption}

\usepackage{appendix}
\usepackage{tummath}
\usepackage{tumtensors} % experimental
\usepackage{tumcolors}

\usepackage{pgfmath}
\newcommand\randmin{}
\newcommand\randmax{}
\newcommand\randmultof{}
\newcommand\setrand[4]%
{\def\randmin{#1}%
	\def\randmax{#2}%
	\def\randmultof{#3}%
	\pgfmathsetseed{#4}%
}
\newcommand\nextrand
{\pgfmathparse{int(int((rnd*(\randmax-\randmin+1)+\randmin)/\randmultof)*\randmultof)}%
	\xdef\thisrand{\pgfmathresult}%
}

% Include other packages here, before hyperref.

% If you comment hyperref and then uncomment it, you should delete
% egpaper.aux before re-running latex.  (Or just hit 'q' on the first latex
% run, let it finish, and you should be clear).
%\usepackage[breaklinks=true,bookmarks=false]{hyperref}


\usepackage[capitalize]{cleveref}
\usepackage[square,sort,comma,numbers]{natbib}

%% this hack seems to be nececessary due to incompatibilities of cvpr template and tikz... -> https://tex.stackexchange.com/questions/398223/tikz-gives-error-command-everyshipouthook-already-defined
\makeatletter
\@namedef{ver@everyshi.sty}{}
\makeatother
%% hackend

%{r\tikzsetnextfilenameawinput}

\newcommand{\rawtimeseries}[1]{

\begin{tikzpicture}[baseline=-2em, inner sep=0]

	\begin{axis}[
		thin,
		width=6cm,
		hide axis,
		height=3cm,
		ymin=0, ymax=1.4,
		no marks,  
		draw opacity=.8,
		smooth=0.01
	]
		 
	\addplot[b1color] table [x=t, y=B1, col sep=comma, forget plot] {images/example/#1};
	\addplot[b9color] table [x=t, y=B9, col sep=comma, forget plot] {images/example/#1};
	\addplot[b10color] table [x=t, y=B10, col sep=comma] {images/example/input.csv};
	
	\addplot[b11color] table [x=t, y=B11, col sep=comma, forget plot] {images/example/#1};
	\addplot[b12color] table [x=t, y=B12, col sep=comma] {images/example/#1};
	
	\addplot[b5color] table [x=t, y=B5, col sep=comma, forget plot] {images/example/#1};
	\addplot[b6color] table [x=t, y=B6, col sep=comma, forget plot] {images/example/#1};
	\addplot[b7color] table [x=t, y=B7, col sep=comma, forget plot] {images/example/#1};
	\addplot[b8color] table [x=t, y=B8, col sep=comma, forget plot] {images/example/#1};
	\addplot[b8Acolor] table [x=t, y=B8A, col sep=comma] {images/example/#1};
		
	\addplot[b2color] table [x=t, y=B2, col sep=comma, forget plot] {images/example/#1};
	\addplot[b3color] table [x=t, y=B3, col sep=comma, forget plot] {images/example/#1};
	\addplot[b4color] table [x=t, y=B4, col sep=comma] {images/example/#1};

	\end{axis}
	
\end{tikzpicture}
	
}

\usepackage{tikz}
\usepackage{pgfplots}
\usetikzlibrary{positioning, calc,arrows,arrows.meta, fit}
%\usetikzlibrary{arrows.meta,calc,decorations.markings,math,arrows.meta}
\usepgfplotslibrary{groupplots}
\usepgfplotslibrary{fillbetween}
\usepgfplotslibrary{statistics} % provides boxplots
\usepackage{xfrac}

\newcommand{\tp}{tp}
\newcommand{\tn}{tn}
\newcommand{\fp}{fp}
\newcommand{\fn}{fn}


\usepackage{tumcolors}
\usepackage{tummath}
\newcommand{\yhat}{\hat{\V{y}}}
\newcommand{\ycorrect}{\hat{y}^+}
\newcommand{\thetadelta}{\V{\Theta}_\delta}
\newcommand{\biasdelta}{b_\delta}
\newcommand{\biasclass}{\V{b}_\text{c}}
\newcommand{\thetaclass}{\V{\Theta}_\text{c}}
\newcommand{\thetafeat}{\V{\Theta}_\text{feat}}
\newcommand{\fclass}{f_\text{c}}
\newcommand{\fdelta}{f_\delta}
\newcommand{\ffeat}{f_\text{feat}}
\newcommand{\f}{f}

\newcommand{\rvtime}{T_c} 
\newcommand{\xuptot}{\M{X}_{\rightarrow t}} 
\newcommand{\deltauptot}{\delta_{\rightarrow t}} 
\newcommand{\tstop}{\ensuremath{t_\text{stop}}}
\newcommand{\meantstop}{\ensuremath{\bar{t}_\text{stop}}}
\usepackage[super]{nth}
\usepackage{mathtools}

\definecolor{evalcolor}{HTML}{3F3F3F}
\definecolor{traincolor}{HTML}{B98951}
\definecolor{validcolor}{HTML}{3F4BBE}

\colorlet{colortrain}{tumblue}
\colorlet{colorinfer}{tumblack}

\colorlet{earlinesscolor}{tumblue}
\colorlet{accuracycolor}{tumorange}

\colorlet{stdcolor}{tumbluelight}
\colorlet{mediancolor}{tumorange}
\colorlet{meancolor}{tumblue}

%\colorlet{b1color}{tumdiagramaubergine}
%\colorlet{b2color}{tumdiagramnavyblue}
%\colorlet{b3color}{tumdiagramturquoise}
%\colorlet{b4color}{tumdiagramgreen}
%\colorlet{b5color}{tumdiagramlimegreen}
%\colorlet{b6color}{tumdiagramyellow}
%\colorlet{b7color}{tumdiagramsand}
%\colorlet{b8color}{tumdiagramredorange}
%\colorlet{b8Acolor}{tumdiagramred}
%\colorlet{b9color}{tumblack}
%\colorlet{b10color}{tumblue}
%\colorlet{b11color}{tumdiagramdarkred}
%\colorlet{b12color}{tumorange}

% atmospheric bands
%\colorlet{b1color}{tumaubergine}%tumdiagramaubergine
%\colorlet{b9color}{tumblack}%tumblack
%\colorlet{b10color}{tumblack}%tumblue

%visisble bands
\colorlet{b2color}{tumblue}%tumdiagramnavyblue
\colorlet{b3color}{tumgreen}%tumdiagramturquoise
\colorlet{b4color}{tumdarkred}%tumdiagramgreen

% near infrared bands
\colorlet{b5color}{tumlimegreen}%tumdiagramlimegreen
\colorlet{b6color}{tumturquoise}%tumdiagramyellow
\colorlet{b7color}{tumblack}%tumdiagramsand
\colorlet{b8color}{tumredorange}%tumdiagramredorange
\colorlet{b8Acolor}{tumred}%tumdiagramred

% SWIR bands
\colorlet{b11color}{tumaubergine}%tumdiagramdarkred
\colorlet{b12color}{tumorange}%tumorange

\colorlet{epsilon0color}{tumorange}
\colorlet{epsilon1color}{tumblue}
\colorlet{epsilon10color}{tumblack}

\colorlet{meadowcolor}{tumbluemedium}
\colorlet{wbarleycolor}{tumbluedark}
\colorlet{corncolor}{tumorange}
\colorlet{wheatcolor}{tumgreen}
\colorlet{sbarleycolor}{tumred}
\colorlet{clovercolor}{tumturquoise}
\colorlet{triticalecolor}{tumsand}

\tikzstyle{rnn}=[draw,circle, inner sep=.1em]
\tikzstyle{norm}=[rounded corners,draw]
\tikzstyle{annot}=[rounded corners, fill=tumblue!20]
\tikzstyle{infer}=[-stealth, shorten >=.0em, shorten <=.0em, colorinfer]
\tikzstyle{loss}=[fill=tumblue!10, rounded corners, font=\small]
\tikzstyle{grad}=[colortrain]

\newcommand{\ptoffset}{\varepsilon}

\tikzstyle{test} = [thick]
\tikzstyle{train} = [thin, dotted]

\usepackage[inline]{enumitem}
\setenumerate{label=(\roman*),itemsep=3pt,topsep=3pt}

\setlength{\belowcaptionskip}{-10pt}
%\usepackage{titlesec}
%\titlespacing{\section}{0pt}{10pt}{3pt}

\usetikzlibrary{external}
\tikzexternalize[prefix=tikz/]
%\tikzexternalize
\tikzexternaldisable

\usepackage[eulergreek]{sansmath}
\pgfplotsset{
	y tick label style={/pgf/number format/.cd,%
		scaled y ticks = false,
		set thousands separator={},
		fixed},
	x tick label style={/pgf/number format/.cd,%
		scaled x ticks = false,
		set decimal separator={,},
		fixed},
	tick label style = {font=\sansmath\sffamily},
	every axis label = {
		font=\sansmath\sffamily},
	every axis/.append style={
		axis lines=left, 
		enlargelimits, 
		thick},
	legend style = {font=\sansmath\sffamily, draw=none, rounded corners, fill opacity=.5, text opacity=1},
	label style = {font=\sansmath\sffamily},
	grid style={line width=.1pt, draw=gray!10},
	major grid style={line width=.2pt,draw=tumgraylight},
}

%\let\tempone\itemize
%\let\temptwo\enditemize
%\renewenvironment{itemize}{\tempone\addtolength{\itemsep}{-.5\baselineskip}}{\temptwo}

\tikzstyle{circ} = [circle, draw=white, fill=tumblue, inner sep=1pt]
\newcommand{\fcn}{
	\begin{tikzpicture}[scale=0.2, rotate=0, baseline=-.25em, inner sep=1pt]
	\node[circ](a0) at (0,-1){};
	\node[circ](a1) at (0,0){};
	\node[circ](a2) at (0,1){};
	
	\node[circ](b0) at (1,-0.5){};
	\node[circ](b1) at (1,0.5){};
	
	\draw[-] (a0) -- (b0);
	\draw[-] (a1) -- (b0);
	\draw[-] (a2) -- (b0);
	
	\draw[-] (a0) -- (b1);
	\draw[-] (a1) -- (b1);
	\draw[-] (a2) -- (b1);
	
	\end{tikzpicture}
}


\newcommand{\earth}{
	\begin{tikzpicture}[baseline=-.25em, inner sep=0]
	\node{\includegraphics[width=8mm]{images/icons/earth}};
	\end{tikzpicture}
}

\newcommand{\sat}{
	\begin{tikzpicture}[baseline=-.25em, inner sep=0]
	\node[rotate=270,anchor=center]{\includegraphics[width=8mm]{images/icons/sat2}};
	\end{tikzpicture}
}

\newcommand{\hidden}[1]{
	\begin{tikzpicture}[scale=.1, baseline=-.25em]	
	%\draw[step=1.0,black,thin] (0,0) grid (#1,1);
	\foreach \i in {1,...,#1}{
		\node[circle, draw=white, fill=tumbluelight, inner sep=1pt] at (\i,0){};
	}
	\end{tikzpicture}
}

\newcommand{\drawvector}[1]{
	\begin{tikzpicture}[scale=.1, baseline=-.25em]	
	%\draw[step=1.0,black,thin] (0,0) grid (#1,1);
	\foreach \i in {1,...,#1}{
		\node[circ] at (\i,0){};
	}
	\end{tikzpicture}
}
\tikzstyle{proba} = [circle, draw=tumgray, inner sep=2.5pt, fill=tumorange]
\newcommand{\drawprobas}[5]{
	\begin{tikzpicture}[scale=.3, baseline=-.25em]	

	\node[proba, fill=tumblue!#1] at (0,-2){};
	\node[proba, fill=tumblue!#2] at (0,-1){};
	\node[proba, fill=tumblue!#3] at (0,-0){};
	\node[proba, fill=tumblue!#4] at (0,1){};
	\node[proba, fill=tumblue!#5] at (0,2){};
	\end{tikzpicture}
}

\newcommand{\vegetationsmodell}{
	\begin{tikzpicture}[scale=0.5, rotate=0, baseline=-.25em]
	\node[proba](a0) at (0,-1){};
	\node[proba](a1) at (0,0){};
	\node[proba](a2) at (0,1){};
	
	\node[proba](b0) at (1,-0.5){};
	\node[proba](b1) at (1,0.5){};
	
	\draw[-] (a0) -- (b0);
	\draw[-] (a1) -- (b0);
	\draw[-] (a2) -- (b0);
	
	\draw[-] (a0) -- (b1);
	\draw[-] (a1) -- (b1);
	\draw[-] (a2) -- (b1);
	
	\node[fit=(a0)(a2)(b1),draw,rounded corners](node name){};
	
	\end{tikzpicture}
}