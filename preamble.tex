\usepackage{tummath}
\usepackage{tumtensors} % experimental
\usepackage{tumcolors}

\usepackage[utf8]{inputenc}
\usepackage[english]{babel}
\usepackage[]{csquotes}

\usepackage{siunitx}

\sisetup{%
	mode = math,
	detect-family,
	detect-weight,  
	exponent-product = \cdot,
	number-unit-separator=\text{\,},
	output-decimal-marker={\text{,}},
	math-rm=\mathsf,
	text-rm=\sffamily,
}

\usepackage{animate}
\usepackage{glossaries}
\usepackage{graphicx}
\usepackage{nth}
\usepackage{ifthen}
\usepackage{minted}

\PassOptionsToPackage{cmyk}{xcolors}

\usepackage{booktabs}
\usepackage{contour}

\usepackage{tikz}
\usepackage{pgfplots}
\usepgfplotslibrary{groupplots}
\pgfplotsset{compat=1.15} 
%\pgfplotsset{compat=1.9}
%\usepgfplotslibrary{groupplots}
%\usepackage{pgfplotstable}
\usepackage{pgfmath}
\usetikzlibrary{pgfplots.dateplot}
\usepgfplotslibrary{dateplot}

%\newcommand\randmin{}
%\newcommand\randmax{}
%\newcommand\randmultof{}
%\newcommand\setrand[4]%
%{\def\randmin{#1}%
%	\def\randmax{#2}%
%	\def\randmultof{#3}%
%	\pgfmathsetseed{#4}%
%}
%\newcommand\nextrand
%{\pgfmathparse{int(int((rnd*(\randmax-\randmin+1)+\randmin)/\randmultof)*\randmultof)}%
%	\xdef\thisrand{\pgfmathresult}%
%}



\usepackage{booktabs}
\usepackage{array}
\newcolumntype{x}{l}
\newcolumntype{X}{>{}l}
\newcolumntype{v}[1]{>{\raggedright\hspace{0pt}}p{#1}}
\newcolumntype{V}[1]{>{\scriptsize\raggedright\hspace{0pt}}p{#1}}

\definecolor{fusionremovedcolor}{HTML}{00CA43}
\definecolor{fusionaddedcolor}{HTML}{FF00F7}
\definecolor{overlapcolor}{HTML}{FAC843}

\usepgfplotslibrary{external}
%\tikzexternalize
\tikzsetexternalprefix{tikz/}
\tikzexternaldisable

\pgfdeclarelayer{background}
\pgfdeclarelayer{foreground}
\pgfsetlayers{background,main,foreground}

%\usetikzlibrary{mindmap}

\usetikzlibrary{arrows} 
\usetikzlibrary{backgrounds}
\usetikzlibrary{fit}
\usetikzlibrary{shapes}
\usetikzlibrary{calc}
\usetikzlibrary{positioning}
\newcommand\tikzmark[1]{
	\tikz[remember picture,overlay] \coordinate (#1);
}

\usetikzlibrary{matrix}
\usepackage[nodayofweek,level]{datetime}
\usetikzlibrary{spy}


%\usepackage{mwe}
%%
%
\usetikzlibrary{3d}
\tikzstyle{perspective3d}=[
x={(0.5cm,0.5cm)}, y={(1cm,0cm)}, z={(0cm,1cm)}]


\colorlet{traincolor}{tumbluelight}
\colorlet{validcolor}{tumbluedark}
\colorlet{evalcolor}{tumorange}

\colorlet{forwardcolor}{tumblue}
\colorlet{backwardcolor}{tumorange}

\colorlet{activationcolor}{tumblue}
\colorlet{gridcolor}{tumgraylight}
\colorlet{contextonecolor}{tumorange}
\colorlet{contexttwocolor}{tumorange!80}
\colorlet{contextthreecolor}{tumorange!60}
\colorlet{contextfourcolor}{tumorange!40}

% defaultvalue -> might be replaced later
\colorlet{tensorcolor}{forwardcolor}

\colorlet{classcolor}{tumivory}
\colorlet{encodercolor}{tumblue}
\colorlet{encodercolor}{tumred}


\newcommand{\eg}{e.g., }
\newcommand{\ie}{i.e. }

% https://tex.stackexchange.com/questions/167925/how-to-make-maths-equations-start-at-the-left
\newcommand{\mathleft}{\@fleqntrue\@mathmargin0pt}
\newcommand{\mathcenter}{\@fleqnfalse}

% \usepackage[ngerman]{babel} % if you get errors on compile: rm *aux *out *log *nav *snm *toc

\newcommand{\rastergrid}{
	\input{images/rastergrid.tikz}
}
\newcommand{\vectorgrid}{
	\input{images/vectorgrid.tikz}
}

%\setbeamersize{description width=0mm}


\newcommand{\MWeight}{\ensuremath{\M{W}}}
\newcommand{\VBias}{\ensuremath{\V{b}}}
\newcommand{\VInput}{\DataVec}
\newcommand{\VHidden}{\ensuremath{\V{h}}}
\newcommand{\FActivation}{\ensuremath{\sigma}}
\newcommand{\VCellState}{\ensuremath{\V{c}}}
\newcommand{\VForgetGate}{\ensuremath{\V{f}}}
\newcommand{\VModulationGate}{\ensuremath{\V{j}}}
\newcommand{\VInputGate}{\ensuremath{\V{i}}}
\newcommand{\VOutputGate}{\ensuremath{\V{o}}}


\setbeamercovered{transparent}

\usepackage{times}
\usepackage{epsfig}
\usepackage{graphicx}
\usepackage{amsmath}
\usepackage{amssymb}
\usepackage[utf8]{inputenc}
\usepackage{booktabs}
\setlength{\tabcolsep}{5pt}
\usepackage{subcaption}

\usepackage{cancel}

% Include other packages here, before hyperref.

% If you comment hyperref and then uncomment it, you should delete
% egpaper.aux before re-running latex.  (Or just hit 'q' on the first latex
% run, let it finish, and you should be clear).
%\usepackage[breaklinks=true,bookmarks=false]{hyperref}


\usepackage[capitalize]{cleveref}
\usepackage[square,sort,comma,numbers]{natbib}




%\pgfplotsset{compat=1.9}
%\usetikzlibrary{positioning, calc,arrows,arrows.meta, fit}
%\usetikzlibrary{arrows.meta,calc,decorations.markings,math,arrows.meta}
%\usepgfplotslibrary{groupplots}
%\usetikzlibrary{pgfplots.groupplots} 
%%\usepgfplotslibrary{fillbetween}
%\usepgfplotslibrary{statistics} % provides boxplots
%\usepackage{xfrac}
%
%\usepackage{pgfplotstable}
%\usepackage{adjustbox}

%\usetikzlibrary{pgfplots.groupplots}
%\usetikzlibrary{shapes}
%\usetikzlibrary{positioning}
%\usetikzlibrary{decorations.pathreplacing}
%
%\newcommand{\tp}{tp}
%\newcommand{\tn}{tn}
%\newcommand{\fp}{fp}
%\newcommand{\fn}{fn}

%
%\usepackage{tumcolors}
%\usepackage{tummath}
\newcommand{\yhat}{\hat{\V{y}}}
\newcommand{\ycorrect}{\hat{y}^+}
\newcommand{\thetadelta}{\V{\Theta}_\delta}
\newcommand{\biasdelta}{b_\delta}
\newcommand{\biasclass}{\V{b}_\text{c}}
\newcommand{\thetaclass}{\V{\Theta}_\text{c}}
\newcommand{\thetafeat}{\V{\Theta}_\text{feat}}
\newcommand{\fclass}{f_\text{c}}
\newcommand{\fdelta}{f_\delta}
\newcommand{\ffeat}{f_\text{feat}}
\newcommand{\f}{f}
\newcommand{\Mweight}{ {\M{\theta}} }

\newcommand{\rvtime}{T_c} 
\newcommand{\xuptot}{\M{X}_{\rightarrow t}} 
\newcommand{\deltauptot}{\delta_{\rightarrow t}} 
\newcommand{\tstop}{\ensuremath{t_\text{stop}}}
\newcommand{\meantstop}{\ensuremath{\bar{t}_\text{stop}}}

\usepackage{mathtools}

\definecolor{evalcolor}{HTML}{3F3F3F}
\definecolor{traincolor}{HTML}{B98951}
\definecolor{validcolor}{HTML}{3F4BBE}

\colorlet{colortrain}{tumblue}
\colorlet{colorinfer}{tumblack}

\colorlet{earlinesscolor}{tumblue}
\colorlet{accuracycolor}{tumorange}

\colorlet{stdcolor}{tumbluelight}
\colorlet{mediancolor}{tumorange}
\colorlet{meancolor}{tumblue}

%\colorlet{b1color}{tumdiagramaubergine}
%\colorlet{b2color}{tumdiagramnavyblue}
%\colorlet{b3color}{tumdiagramturquoise}
%\colorlet{b4color}{tumdiagramgreen}
%\colorlet{b5color}{tumdiagramlimegreen}
%\colorlet{b6color}{tumdiagramyellow}
%\colorlet{b7color}{tumdiagramsand}
%\colorlet{b8color}{tumdiagramredorange}
%\colorlet{b8Acolor}{tumdiagramred}
%\colorlet{b9color}{tumblack}
%\colorlet{b10color}{tumblue}
%\colorlet{b11color}{tumdiagramdarkred}
%\colorlet{b12color}{tumorange}

% atmospheric bands
\colorlet{b1color}{tumaubergine}%tumdiagramaubergine
\colorlet{b9color}{tumblack}%tumblack
\colorlet{b10color}{tumblack}%tumblue

%visisble bands
\colorlet{b2color}{tumblue}%tumdiagramnavyblue
\colorlet{b3color}{tumgreen}%tumdiagramturquoise
\colorlet{b4color}{tumdarkred}%tumdiagramgreen

% near infrared bands
\colorlet{b5color}{tumlimegreen}%tumdiagramlimegreen
\colorlet{b6color}{tumturquoise}%tumdiagramyellow
\colorlet{b7color}{tumblack}%tumdiagramsand
\colorlet{b8color}{tumredorange}%tumdiagramredorange
\colorlet{b8Acolor}{tumred}%tumdiagramred

% SWIR bands
\colorlet{b11color}{tumaubergine}%tumdiagramdarkred
\colorlet{b12color}{tumorange}%tumorange

\colorlet{epsilon0color}{tumorange}
\colorlet{epsilon1color}{tumblue}
\colorlet{epsilon10color}{tumblack}

\colorlet{meadowcolor}{tumbluemedium}
\colorlet{wbarleycolor}{tumbluedark}
\colorlet{corncolor}{tumorange}
\colorlet{wheatcolor}{tumgreen}
\colorlet{sbarleycolor}{tumred}
\colorlet{clovercolor}{tumturquoise}
\colorlet{triticalecolor}{tumsand}

\tikzstyle{rnn}=[draw,circle, inner sep=.1em]
\tikzstyle{norm}=[rounded corners,draw]
\tikzstyle{annot}=[rounded corners, fill=tumblue!20]
\tikzstyle{infer}=[-stealth, shorten >=.0em, shorten <=.0em, colorinfer]
\tikzstyle{loss}=[fill=tumblue!10, rounded corners, font=\small]
\tikzstyle{grad}=[colortrain]

\newcommand{\ptoffset}{\varepsilon}

\tikzstyle{test} = [thick]
\tikzstyle{train} = [thin, dotted]

\usepackage[inline]{enumitem}
\setenumerate{label=(\roman*),itemsep=3pt,topsep=3pt}

\setlength{\belowcaptionskip}{-10pt}
%\usepackage{titlesec}
%\titlespacing{\section}{0pt}{10pt}{3pt}



\usepackage{setspace}
\usepackage{eqparbox, etoolbox}

\newlist{rdescription}{description}{1}

\AtBeginEnvironment{rdescription}{%
	\renewcommand*\descriptionlabel[2][Des]{\hspace\labelsep\eqmakebox[Des][r]{\hfill\normalfont\bfseries #2}}\setlist[rdescription]{leftmargin =\dimexpr\eqboxwidth{Des}+\labelsep}}%


\usepackage[eulergreek]{sansmath}
\pgfplotsset{
	y tick label style={/pgf/number format/.cd,%
		scaled y ticks = false,
		set thousands separator={},
		fixed},
	x tick label style={/pgf/number format/.cd,%
		scaled x ticks = false,
		set decimal separator={,},
		fixed},
	tick label style = {font=\sansmath\sffamily},
	every axis label = {
		font=\sansmath\sffamily},
	every axis/.append style={
		axis lines=left, 
		enlargelimits, 
		thick},
	legend style = {font=\sansmath\sffamily, draw=none, rounded corners, fill opacity=.5, text opacity=1},
	label style = {font=\sansmath\sffamily},
	grid style={line width=.1pt, draw=gray!10},
	major grid style={line width=.2pt,draw=tumgraylight},
}

%\let\tempone\itemize
%\let\temptwo\enditemize
%\renewenvironment{itemize}{\tempone\addtolength{\itemsep}{-.5\baselineskip}}{\temptwo}

\tikzstyle{circ} = [circle, draw=white, fill=tumblue, inner sep=1pt]
\newcommand{\fcn}{
	\begin{tikzpicture}[scale=0.2, rotate=0, baseline=-.25em, inner sep=1pt]
	\node[circ](a0) at (0,-1){};
	\node[circ](a1) at (0,0){};
	\node[circ](a2) at (0,1){};
	
	\node[circ](b0) at (1,-0.5){};
	\node[circ](b1) at (1,0.5){};
	
	\draw[-] (a0) -- (b0);
	\draw[-] (a1) -- (b0);
	\draw[-] (a2) -- (b0);
	
	\draw[-] (a0) -- (b1);
	\draw[-] (a1) -- (b1);
	\draw[-] (a2) -- (b1);
	
	\end{tikzpicture}
}


\newcommand{\earth}{
	\begin{tikzpicture}[baseline=-.25em, inner sep=0]
	\node{\includegraphics[width=8mm]{images/icons/earth}};
	\end{tikzpicture}
}

\newcommand{\sat}{
	\begin{tikzpicture}[baseline=-.25em, inner sep=0]
	\node[rotate=270,anchor=center]{\includegraphics[width=8mm]{images/icons/sat2}};
	\end{tikzpicture}
}

\newcommand{\hidden}[1]{
	\begin{tikzpicture}[scale=.1, baseline=-.25em]	
	%\draw[step=1.0,black,thin] (0,0) grid (#1,1);
	\foreach \i in {1,...,#1}{
		\node[circle, draw=white, fill=tumbluelight, inner sep=1pt] at (\i,0){};
	}
	\end{tikzpicture}
}

\newcommand{\drawvector}[1]{
	\begin{tikzpicture}[scale=.1, baseline=-.25em]	
	%\draw[step=1.0,black,thin] (0,0) grid (#1,1);
	\foreach \i in {1,...,#1}{
		\node[circ] at (\i,0){};
	}
	\end{tikzpicture}
}
\tikzstyle{proba} = [circle, draw=tumgray, inner sep=2.5pt, fill=tumorange]
\newcommand{\drawprobas}[5]{
	\begin{tikzpicture}[scale=.3, baseline=-.25em]	

	\node[proba, fill=tumblue!#1] at (0,-2){};
	\node[proba, fill=tumblue!#2] at (0,-1){};
	\node[proba, fill=tumblue!#3] at (0,-0){};
	\node[proba, fill=tumblue!#4] at (0,1){};
	\node[proba, fill=tumblue!#5] at (0,2){};
	\end{tikzpicture}
}



\newcommand{\vegetationsmodell}{
	\begin{tikzpicture}[scale=0.5, rotate=0, baseline=-.25em, minimum width=0cm, minimum height=0cm]
	\node[proba](a0) at (0,-1){};
	\node[proba](a1) at (0,0){};
	\node[proba](a2) at (0,1){};
	
	\node[proba](b0) at (1,-0.5){};
	\node[proba](b1) at (1,0.5){};
	
	\draw[-] (a0) -- (b0);
	\draw[-] (a1) -- (b0);
	\draw[-] (a2) -- (b0);
	
	\draw[-] (a0) -- (b1);
	\draw[-] (a1) -- (b1);
	\draw[-] (a2) -- (b1);
	
%	\node[fit=(a0)(a2)(b1)](node name){};
	
	\end{tikzpicture}
}

\tikzstyle{tsmark} = [mark=|,mark size=2pt]



\definecolor{s1}{RGB}{228, 26, 28}
\definecolor{s2}{RGB}{55, 126, 184}
\definecolor{s3}{RGB}{77, 175, 74}
\definecolor{s4}{RGB}{152, 78, 163}
\definecolor{s5}{RGB}{255, 127, 0}
\pgfplotscreateplotcyclelist{featurecolorlist}{
	s1,every mark/.append style={fill=s1},mark=*\\
	s2,every mark/.append style={fill=s2},mark=*\\
	s3,every mark/.append style={fill=s3},mark=*\\
	s4,every mark/.append style={fill=s4},mark=*\\
	s5,every mark/.append style={fill=s5},mark=*\\
}

\newcommand{\backupbegin}{
	\newcounter{framenumberappendix}
	\setcounter{framenumberappendix}{\value{framenumber}}
}
\newcommand{\backupend}{
	\addtocounter{framenumberappendix}{-\value{framenumber}}
	\addtocounter{framenumber}{\value{framenumberappendix}} 
}

